\section{Guards and System Predicates}\label{appendix:guards-system}

Guards and system predicates extend GLP programs with access to the GLP runtime state, operating system and hardware capabilities.

\mypara{Guard predicates}
Guards provide read-only access to the runtime state of GLP computation. A guard appears after the clause head, separated by \verb=|=, and must be satisfied for the clause to be selected. The following builtin guards are fundamental for concurrent GLP programming:

\begin{itemize}
\item \verb|ground(X)| succeeds if \verb|X| is ground (contains no variables).
\item \verb|known(X)| succeeds if \verb|X| is not a variable, though it may not be ground.
\item \verb|writer(X)| and \verb|reader(X)| succeed if \verb|X| is an uninstantiated writer or reader respectively. Note that \verb|reader(X)| is non-monotonic.
\item \verb|otherwise| succeeds if all previous clauses for this procedure failed.
\item \verb|X =?= Y| succeeds if both arguments are ground and equal. For example, \verb|f(a,X?) =?= f(b,Z?)| fails (not suspends) because the ground subterms \verb|a| and \verb|b| already differ. Note that the implementation checks terms left-to-right and does not guarantee early failure detection; \verb|f(X?,a) =?= f(Y?,b)| may suspend rather than fail.
\item Arithmetic comparison guards \verb|X < Y|, \verb|X > Y|, \verb|X =< Y|, \verb|X >= Y|, \verb|X =:= Y|, \verb|X =\= Y| succeed if both arguments are ground numbers satisfying the relation.
\end{itemize}

When a guard's success condition implies that a variable is ground, the clause body may contain multiple occurrences of that variable's reader without violating the single-writer requirement, enabling safe replication of ground terms to multiple concurrent consumers.

\mypara{Defined guard predicates}
To support abstract data types and cleaner code organization, GLP provides for user-defined guards via unit clauses. A unit clause \verb|p(T1,...,Tn).| defines a guard predicate; the call \verb|p(S1,...,Sn)| in guard position is unfolded to the term matching of \verb|T1| with \verb|S1|, ..., \verb|Tn| with \verb|Sn|. For example, the equality guard is defined by the clause \verb|X = X.|, so the guard \verb|A = B| unfolds to matching both \verb|A| and \verb|B| against the same variable \verb|X|.

\mypara{System predicates}
System predicates execute atomically with goal/clause reduction and provide access to underlying runtime services:
\begin{itemize}
\item \verb|evaluate(Expr?,Result)| evaluates ground arithmetic expressions.
\item \verb|current_time(T)| provides system timestamps for temporal coordination.
\item \verb|variable_name(X,Name)| returns a unique identifier for variable \verb|X| and its pair.
\end{itemize}

\mypara{Arithmetic evaluation in assignments}
Arithmetic expressions are defined by the following clause:
\begin{verbatim}
X? := E :- ground(E) | evaluate(E?,X).
\end{verbatim}
Ensuring the expression is ground before calling the system evaluator, maintaining program safety whilst providing convenient notation for mathematical computations.

