\section{Deferred Proofs}\label{appendix:proofs}

This appendix contains proofs of propositions from Section~\ref{sec:glp}.

\begin{proposition*}[\ref{prop:glp-deduction}, GLP Computation is Deduction]
Let $/?$ replace every reader by its paired writer. If $(G_0$ \verb|:-| $G_n)\sigma$ is the outcome of a proper GLP run, then $(G_0$ \verb|:-| $G_n)\sigma/?$ is a logical consequence of $P/?$.
\end{proposition*}

\begin{proof}
The $/\!?$ operator replaces every reader $X?$ by its paired writer $X$, transforming GLP terms into LP terms. We show that the GLP run $\rho$ corresponds to an LP run $\rho/?$ of $LP(P/?)$.

Consider a GLP transition $(G, \sigma) \rightarrow (G', \sigma')$:
\begin{itemize}
    \item \emph{Reduce transition}: Goal $A$ reduces with clause $C$ via writer mgu $\hat\sigma$. Applying $/\!?$, the clause $C/?$ is an LP clause, and $A/?$ unifies with the head $H/?$ via the mgu $\hat\sigma/?$ (since writers map to writers). This is a valid LP reduction.
    \item \emph{Communicate transition}: A reader $X? \in G$ is replaced by the value $T$ assigned to its paired writer. Under $/\!?$, both $X?$ and $X$ map to $X$, so this transition becomes the identity---the variable $X$ is already bound to $T$ in the LP view.
\end{itemize}

Thus each GLP transition corresponds to zero or one LP transitions, and the GLP run $\rho$ projects to an LP run $\rho/?$ of $LP(P/?)$. By standard LP soundness, the outcome of $\rho/?$ is a logical consequence of $P/?$.
\end{proof}

\begin{proposition*}[\ref{prop:so-preservation}, SO Preservation]
If the initial goal $G_0$ satisfies SO, then every goal in the GLP run satisfies SO.
\end{proposition*}

\begin{proof}
By induction on the length of the run. The base case is immediate: $G_0$ satisfies SO by assumption.

For the inductive step, assume $G$ satisfies SO and consider a transition $(G, \sigma) \rightarrow (G', \sigma')$:
\begin{itemize}
    \item \emph{Reduce transition}: Goal $A \in G$ reduces with clause $C = (H \mathrel{\mbox{\texttt{:-}}} B)$ via writer mgu $\hat\sigma$, yielding $G' = (G \setminus \{A\} \cup B)\hat\sigma$. Since $C$ satisfies SRSW, it satisfies SO. Since $C$ is renamed apart from $G$, the variables in $B$ are fresh. The writer mgu $\hat\sigma$ maps writers in $A$ to subterms of $H$ and vice versa; by SO of both $G$ and $C$, each variable is assigned at most once. Applying $\hat\sigma$ to $(G \setminus \{A\} \cup B)$ replaces each variable by a term containing fresh variables (from $B$) or ground subterms. Since no variable in $G \setminus \{A\}$ occurs in $A$ (by SO of $G$), and no variable in $B$ occurs in $G$ (by renaming apart), $G'$ satisfies SO.
    
    \item \emph{Communicate transition}: $G' = G\hat\sigma?$ where $\hat\sigma? = \{X? := T\}$. Since $G$ satisfies SO, $X?$ occurs at most once in $G$. Replacing this single occurrence by $T$ (which satisfies SO by Definition~\ref{def:writers-assignment}) preserves SO, provided $T$ shares no variables with the rest of $G$. By the proper run condition, variables in $T$ are fresh, so $G'$ satisfies SO.
\end{itemize}
\end{proof}

\begin{proposition*}[\ref{prop:glp-monotonicity}, Monotonicity]
If unit goal $A$ can reduce with clause $C$ at step $i$, then either $A$ has been reduced by step $j > i$, or an instance of $A$ can still reduce with $C$ at step $j$.
\end{proposition*}

\begin{proof}
Suppose goal $A$ can reduce with clause $C$ at step $i$, meaning the writer mgu of $A$ and the head $H$ of (a renaming of) $C$ succeeds. Consider what can change between steps $i$ and $j > i$:
\begin{itemize}
    \item \emph{Reduce transitions on other goals}: These do not affect $A$ directly. By SO, no other goal shares a writer with $A$, so no other reduction can bind a writer in $A$.
    
    \item \emph{Communicate transitions}: These bind readers, not writers. A communicate transition $X? := T$ may instantiate a reader $X? \in A$, yielding $A' = A\{X? := T\}$. We show $A'$ can still reduce with $C$:
    
    The original writer mgu succeeded, meaning at position $p$ where $X?$ occurred in $A$, either (a) $H$ had a writer $Y$ at position $p$, yielding assignment $Y := X?$, or (b) $H$ had a reader $Y?$ at position $p$, which would have caused failure (reader-reader), contradiction.
    
    In case (a), after the communicate transition, $A'$ has $T$ at position $p$. The clause $C$ (renamed apart for $A'$) has a fresh writer $Y'$ at position $p$. The writer mgu now yields $Y' := T$, which succeeds.
    
    \item \emph{Reduce transition on $A$}: If $A$ itself is reduced at some step $k$ with $i < k \le j$, then an instance of $A$ has been reduced, satisfying the proposition.
\end{itemize}

Thus, if $A$ has not been reduced by step $j$, the (possibly instantiated) goal $A'$ at step $j$ can still reduce with a fresh renaming of $C$.
\end{proof}
