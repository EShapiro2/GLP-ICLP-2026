\section{Introduction}

\mypara{Grassroots platforms}
Grassroots platforms~\cite{shapiro2023grassrootsBA,shapiro2025atomic} are distributed applications designed to restore digital sovereignty to ordinary people using only their smartphones. A distributed system is \emph{grassroots} if it is permissionless and can have autonomous, independently-deployed instances---geographically and over time---that may interoperate once interconnected. This requires that agents can operate within their group without interference from agents outside it, while retaining the ability to interact when desired.

Grassroots platforms are run by people on their networked personal devices, who are identified cryptographically~\cite{rivest1978method}, communicate only with authenticated friends, and can participate in multiple instances of multiple platforms simultaneously. The grassroots social graph~\cite{shapiro2023gsn} serves as both a platform in its own right and the infrastructure layer for all other grassroots platforms: nodes represent people, edges represent authenticated friendships, and connected components arise spontaneously through befriending. Upon this foundation, grassroots social networks~\cite{shapiro2023gsn}, grassroots cryptocurrencies~\cite{shapiro2024gc}, and grassroots democratic federations~\cite{shapiro2025GF} are built.

\mypara{Why existing alternatives fail}
Federated systems like Mastodon~\cite{raman2019challenges} remain server-dependent: users are still at the mercy of server operators who control their accounts autocratically. Global shared data structures---whether replicated (blockchains), distributed (IPFS, DHT), or pub/sub with global directories---prevent true independence, as members cannot ignore changes made by others~\cite{shapiro2023grassrootsBA}. Single-instance architectures create lock-in: two instances would clash over global resources (domain names, boot nodes) hardwired into their code. No community can independently bootstrap and later interoperate---they must join existing platforms on those platforms' terms.

Grassroots platforms require a different architecture: one where small groups can form and operate independently, yet coalesce into larger communities when they interconnect, all without central coordination or global consensus.

\mypara{Programming grassroots platforms}
A key challenge in implementing grassroots platforms is overcoming faulty and malicious participants~\cite{lamport1982byzantine}. Without secure language support, correct participants cannot reliably identify each other, establish secure communication channels, or verify each other's code integrity~\cite{sabt2015trusted,costan2016intel}.
While grassroots platforms have been formally specified and their properties proven~\cite{shapiro2023grassrootsBA,shapiro2023gsn,shapiro2024gc,shapiro2025GF,shapiro2025atomic}, they remain mathematical constructions without actual implementations. To the best of our knowledge, no existing programming language provides the necessary combination of distributed execution, cryptographic security, safety, and liveness guarantees required to realize these specifications. GLP aims to close the gap between mathematical specification and implementation of grassroots platforms.

\mypara{GLP} We present GLP, a secure, multiagent, concurrent, logic programming language designed for implementing grassroots platforms.
GLP extends logic programs~\cite{lloyd1987foundations,sterling1994art} with paired single-reader/single-writer variables (akin to futures and promises~\cite{dauth2019futures,azadbakht2020formal}), each establishing a secure single-message communication channel between the single writer and the single reader, enabling subsequent secure multidirectional communication by sharing readers and writers in messages.

We present GLP and prove its properties in four steps:
\begin{enumerate}
\item \textbf{Mathematical Foundations:} We recall the mathematical framework for grassroots platforms~\cite{shapiro2025atomic}: transition systems, multiagent transition systems with atomic transactions, protocols, and the key theorem that a protocol over interactive transactions is grassroots.

\item \textbf{Logic Programs:} We recall logic programs (LP)~\cite{lloyd1987foundations,sterling1994art} and their operational semantics via transition systems.

\item \textbf{GLP:} We extend LP with reader/writer pairs satisfying the Single-Reader/Single-Writer (SRSW) requirement; extend unification to suspend upon an attempt to bind a reader; extend configurations to include pending assignments to readers; and provide nondeterministic interleaving-based asynchronous operational semantics. We prove safety properties, including that GLP computations are deductions~\cite{kowalski1974predicate,lloyd1987foundations}.

\item \textbf{Multiagent GLP:} We define multiagent GLP (maGLP) as a transactions-based multiagent transition system with three transactions: Reduce (unary, local goal reduction), Communicate (binary, writer-to-reader message transfer between agents), and Network (binary, cold-call channel establishment). We prove maGLP is grassroots and establish that any GLP application using cold calls is grassroots.
\end{enumerate}

\mypara{Paper outline}
Section~\ref{sec:glp} presents Concurrent GLP: transition systems, logic programs as transition systems, GLP extending LP with readers, operational semantics, guards, programming examples, and the grassroots social graph simulation. Section~\ref{sec:maglp} extends to multiagent GLP: multiagent transition systems, the maGLP definition, and the multiagent social graph. Section~\ref{sec:grassroots} proves maGLP is grassroots. Section~\ref{section:related-work} reviews related work, and Section~\ref{section:conclusion} concludes.
