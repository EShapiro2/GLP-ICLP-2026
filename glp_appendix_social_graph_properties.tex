\section{Grassroots Social Graph Protocol Properties}\label{appendix:friend-introductions}

\subsection{Non-blocking Operation Through Variable Synchronization}\label{appendix:social-graph-implementation}

The social graph protocol achieves non-blocking operation through careful use of unbound variables and the \verb|inject| procedure. When initiating connections, agents send offers containing unbound response variables and continue processing other messages while awaiting responses. Similarly, when receiving offers, agents query their users for approval without blocking the main protocol loop. 

The \verb|inject| procedure in Program~\ref{program:social-graph} implements deferred message insertion: when \verb|X| is unbound, \verb|inject| passes input stream messages to its output whilst waiting for \verb|X| to become bound. Once \verb|X| is known, it inserts the message and terminates. This ensures the protocol remains responsive while awaiting responses to connection attempts, preventing any single pending operation from blocking the entire message processing loop.


\subsection{Protocol Properties}

The social graph protocol exhibits several essential properties for grassroots platforms. Non-blocking operation ensures that agents remain responsive during connection establishment, with no single operation capable of indefinitely blocking message processing. Symmetric channel establishment guarantees that successful connections result in bidirectional communication with identical capabilities for both parties. The unified message processing through stream merging provides fair handling of messages from all sources, preventing starvation of any input source.

The protocol's use of unbound variables for response coordination elegantly solves the distributed consensus problem for connection establishment. Both agents must explicitly agree to connect—the offerer by initiating and the receiver by accepting—with the shared response variable serving as the synchronization mechanism. This design ensures that connections only form through mutual consent while avoiding complex state machines or timeout mechanisms.

The friends list serves multiple roles simultaneously: it represents the agent's local view of the social graph, provides the routing table for message sending, and maintains the state needed for friend-mediated introductions. This unified structure simplifies reasoning about the protocol while enabling efficient implementation. The incremental construction of the social graph through individual connections allows multiple disconnected components to form independently and later merge through cross-component connections, embodying the grassroots principle of spontaneous emergence without central coordination.

