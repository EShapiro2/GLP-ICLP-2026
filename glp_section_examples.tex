\section{Programming Examples}\label{section:programming-examples}

We present some basic GLP programming techniques through examples. Additional  techniques appear in Appendix~\ref{appendix:additional-techniques}.

\Program{Concurrent Monitor}\label{program:monitor}
\begin{verbatim}
monitor(Reqs) :- monitor(Reqs?,0).

monitor([add(N)|Reqs],Sum) :- 
    Sum1 := Sum? + N?, monitor(Reqs?,Sum1?).
monitor([subtract(N)|Reqs],Sum) :- 
    Sum1 := Sum? - N?, monitor(Reqs?,Sum1?).
monitor([value(V)|Reqs],Sum) :- 
    ground(Sum?) | V = Sum?, monitor(Reqs?,Sum?).
monitor([],_).
\end{verbatim}
An example initial goal is:
\begin{verbatim}
client1(Xs), client2(Ys), merge(Xs?,Ys?,Zs), monitor(Zs?).
\end{verbatim}
The monitor demonstrates a stateful service handling requests from multiple concurrent clients, serialized through stream merging (Program~\ref{program:merge}) whilst maintaining state through the \verb|Sum| parameter in tail-recursive calls. The \verb|value(V)| request demonstrates incomplete messages—upon receipt the monitor binds the response variable \verb|V| to the current sum. 

A fixed number of clients can be served by a fixed binary merge tree.  A dynamically-changing set of clients can be served by the following dynamic stream merger, where an existing client can onboard a new client with a request stream \verb|Ws| by sending down its own request stream the request \verb|merge(Ws?)|, creating a dynamic merge tree as follows.


\Program{Dynamic Stream Merger}\label{program:dynamic-merge}
\begin{verbatim}
merger(Ws,Xs,Out?) :- merge(Ws?,Xs?,Out).

merge([merge(Ws)|Xs],Ys,Zs?) :- 
   merger(Ws?,Xs?,Xs1), merge(Xs1?,Ys?,Zs).
merge(Xs,[merge(Ws)|Ys],Zs?) :- 
   merger(Ws?,Ys?,Ys1), merge(Xs?,Ys1?,Zs).
merge([X|Xs],Ys,[X?|Zs?]) :- 
    X =\= merge(_) | merge(Ys?,Xs?,Zs).
merge(Xs,[Y|Ys],[Y?|Zs?]) :- 
    Y =\= merge(_) | merge(Xs?,Ys?,Zs).
merge([],[],[]).
\end{verbatim}
The resulting merge tree can be highly imbalanced; standard optimization techniques can be applied~\cite{shapiro1984fair,shapiro1986multiway}.

Broadcasting to multiple concurrent consumers uses the \verb|ground| guard to enable input replication without violating the single-writer constraint:

\Program{Concurrent Stream Distribution}\label{program:distribute}
\begin{verbatim}
distribute([X|Xs],[X|Ys1],...,[X|Ysn]) :- 
    ground(X) | distribute(Xs?,Ys1?,...,Ysn?).
distribute([],[],...,[]).
\end{verbatim}
When \verb|X| is ground, multiple occurrences in the clause body do not violate SRSW.
Additional programming examples appear in Appendix~\ref{appendix:additional-techniques}.

