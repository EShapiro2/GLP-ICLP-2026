\section{Proofs}\label{appendix:proofs}

\LPComputationisDeduction*
\begin{proof}
We prove by induction on the length of the run that each step preserves logical consequence.

\mypara{Base case} For $n=0$, we have $G_0 = G_0$ with empty substitution $\epsilon$. The outcome $(G_0 \verb|:-| G_0)$ is a tautology, hence a logical consequence of any program.

\mypara{Inductive step} Assume the proposition holds for runs of length $k$. Consider a proper run of length $k+1$:
$$\rho: G_0 \xrightarrow{\sigma_1} \cdots \xrightarrow{\sigma_k} G_k \xrightarrow{\sigma_{k+1}} G_{k+1}$$

By the inductive hypothesis, $(G_0 \verb|:-| G_k)\sigma'$ is a logical consequence of $M$, where $\sigma' = \sigma_1 \circ \cdots \circ \sigma_k$.

For the transition $G_k \xrightarrow{\sigma_{k+1}} G_{k+1}$:
\begin{itemize}
   \item There exists atom $A \in G_k$ and clause $(H \verb|:-| B) \in M$ renamed apart
   \item $\sigma_{k+1}$ is the mgu of $A$ and $H$
   \item $G_{k+1} = (G_k \setminus \{A\} \cup B)\sigma_{k+1}$
\end{itemize}

Since $(H \verb|:-| B)$ is a clause in $M$ and $\sigma_{k+1}$ unifies $A$ with $H$, we know that:
\begin{itemize}
   \item The instance $(H \verb|:-| B)\sigma_{k+1}$ is a logical consequence of $M$ (by instantiation of a program clause)
   \item Since $A\sigma_{k+1} = H\sigma_{k+1}$ (by the mgu property), we can replace $A$ with $B$ under substitution $\sigma_{k+1}$
   \item Therefore, the implication $(G_k \verb|:-| G_{k+1})$ is a logical consequence of $M$ when we consider that $G_{k+1}$ was obtained by replacing $A$ in $G_k$ with $B$ and applying $\sigma_{k+1}$
\end{itemize}

By the transitivity of logical consequence, if $(G_0 \verb|:-| G_k)\sigma'$ is a logical consequence of $M$ and $(G_k \verb|:-| G_{k+1})$ follows from $M$ under the additional substitution $\sigma_{k+1}$, then $(G_0 \verb|:-| G_{k+1})(\sigma' \circ \sigma_{k+1})$ is a logical consequence of $M$.

Since $\sigma = \sigma' \circ \sigma_{k+1} = \sigma_1 \circ \cdots \circ \sigma_{k+1}$, we conclude that the outcome $(G_0 \verb|:-| G_{k+1})\sigma$ is a logical consequence of $M$.
\qed\end{proof}

\ReaderOnlyInstantiation*
\begin{proof}
Consider the transition $G_i \rightarrow G_{i+1}$ via reduction of some atom $A' \in G_i$ with clause $C$. Let $(H \verb|:-| B)$ be the renaming of $C$ apart from $A'$, with writer mgu $\sigma$ and reader counterpart $\sigma?$.

By Definition~\ref{definition:GLP-ts}, the Reduce transition specifies that $G_{i+1} = (G_i \setminus \{A'\} \cup B)\sigma$, and the configuration's reader substitution is updated with $\sigma?$.

For any atom $A \in G_{i+1}$ that also appeared in $G_i$, we have:

\begin{enumerate}
\item $A \neq A'$ (A was not the reduced atom). Then $A \in G_i \setminus \{A'\}$. The reduction applies $\sigma$ to all atoms in the resolvent. Since $A$ was in $G_i$ and the clause was renamed apart from the entire goal (including $A$), any writers in $A$ are distinct from $V_\sigma$. Therefore $\sigma$ does not instantiate variables in $A$. Only the reader counterpart $\sigma?$ can affect $A$. Since $\sigma?$ is a reader substitution with $V_{\sigma?} \subset V?$, we have $A$ in $G_{i+1}$ equals $A'\tau$ where $A' \in G_i$ and $\tau = \sigma?$ instantiates only readers.

\item $A = A'$ (A was the reduced atom). This case cannot occur since $A'$ is removed from the resolvent during reduction and thus cannot appear in $G_{i+1}$.
\end{enumerate}

Therefore, any atom persisting from $G_i$ to $G_{i+1}$ is instantiated only by the reader substitution $\sigma?$.
\qed
\end{proof}

\GLPComputationsareDeductions*
\begin{proof}
Follows from the correspondence between GLP reductions and LP reductions on pure logic variants, combined with Proposition~\ref{proposition:LP-computation-deduction}.
\qed
\end{proof}

\SRSWInvariant*
\begin{proof}
By induction on run length. The base case holds by assumption. For the inductive step, consider $G_i \rightarrow G_{i+1}$ via reduction with clause $C$ renamed apart. The renamed clause has fresh variables satisfying the SRSW syntactic constraint. The reduction replaces atom $A$ with body $B$ and applies $\sigma?$. Since $\sigma?$ replaces variables with terms (eliminating variable occurrences rather than duplicating them), and $B$ has fresh variables distinct from $G_i$, the SRSW invariant is preserved in $G_{i+1}$.
\qed\end{proof}

\Acyclicity* 
\begin{proof}
By induction on run length. For the base case, $G_0$ contains no circular terms by assumption. For the inductive step, assume $G_i$ contains no circular terms and consider the transition $G_i \rightarrow G_{i+1}$ via reduction of atom $A$ with clause $C$. Let $(H \verb|:-| B)$ be the renaming of $C$ apart from $A$, with writer mgu $\sigma$ and reader counterpart $\sigma?$. The reader counterpart exists only if for all $X \in V_\sigma$, $X? \notin X\sigma$ (occurs check). This ensures no writer is bound to a term containing its paired reader. Since $G_{i+1} = (G_i \setminus \{A\} \cup B)\sigma?$, and the occurs check prevents circular assignments, $G_{i+1}$ contains no circular terms.
\qed\end{proof}

\Monotonicity*
\begin{proof}
By induction on $j - i$. For the base case ($j = i$), the atom $A \in G_i$ can reduce with $C$ by assumption. For the inductive step, assume the property holds for $j = k$ and consider $j = k + 1$. 

If $A$ was reduced at some step between $i$ and $k$, then case (1) holds. Otherwise, by the inductive hypothesis, there exists $A' \in G_k$ where $A' = A\tau$ for some reader substitution $\tau$, and $A'$ can reduce with $C$.

Consider the transition $G_k \rightarrow G_{k+1}$. If the reduction involves $A'$, then case (1) holds for $j = k + 1$. If the reduction involves a different atom $B \in G_k$, then $A'$ persists in $G_{k+1}$, possibly further instantiated. Specifically, the reduction applies substitution $\sigma?$ where $\sigma?$ instantiates only readers (by definition of reader counterpart). Thus there exists $A'' \in G_{k+1}$ where $A'' = A'\sigma? = A(\tau \circ \sigma?)$, and $\tau \circ \sigma?$ is a reader substitution.

Since $A'$ could reduce with $C$ (renamed apart) via some writer mgu at step $k$, and $\sigma?$ only instantiates readers, the unification of $A''$ with the head of $C$ (appropriately renamed) still succeeds: reader instantiation preserves unifiability and cannot introduce new writer instantiation requirements. Therefore $A''$ can reduce with $C$ at step $k + 1$.
\qed\end{proof}

\maGLPisgrassroots*    
\begin{proof}
We prove that maGLP is oblivious and interactive.
\begin{enumerate}
    \item \textbf{maGLP is Oblivious:}  Follows directly from Proposition~\ref{proposition:oblivious}.
    \item \textbf{maGLP is Interactive:}  We have to show that in any configuration $c$ of a run of maGLP over $P$, if this configuration is in fact  configuration over $P'\supset P$, then members of $P$ have a behaviour not available to them if this was a run over $P$. The answer, of course, is that in such a case any agent $q\in P'\setminus P$ can send a network message to some agent $p\in P$, resulting in the local state of $p$ having an `alien trace'—a variable produced by an agent not in $P$—a behaviour not available to $P$ on their own.
\end{enumerate}
We conclude that maGLP is grassroots.
\qed\end{proof}

