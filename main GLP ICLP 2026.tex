%\documentclass{new_tlp}
\documentclass[sigconf, nonacm]{acmart}

\let\Bbbk\relax
% Essential packages
\usepackage{graphicx}
\usepackage{amsmath,amssymb,amsfonts}
\usepackage{xspace}
\usepackage{hyperref}
\usepackage{cleveref}
\usepackage{xcolor}
\usepackage{enumitem}

% Algorithm packages
\usepackage{algorithm}
\usepackage[noend]{algpseudocode}

% Additional packages
\usepackage{multirow}
\usepackage{mathtools}
\usepackage{relsize}
\usepackage{bm}
\usepackage{verbatimbox}
\usepackage{wrapfig}

% Theorem-like environments (not provided by new_tlp class)
\usepackage{amsthm}
\theoremstyle{definition}
\newtheorem{definition}{Definition}[section]
\newtheorem{example}[definition]{Example}
\theoremstyle{plain}
\newtheorem{theorem}[definition]{Theorem}
\newtheorem{proposition}[definition]{Proposition}
\newtheorem{lemma}[definition]{Lemma}
\newtheorem{corollary}[definition]{Corollary}
\theoremstyle{remark}
\newtheorem{remark}[definition]{Remark}

% Restatable theorems (must be after amsthm)
\usepackage{thmtools,thm-restate}

% Custom formatting commands
\newcommand{\mypara}[1]{\smallskip\noindent\textbf{#1.}}
\newcommand{\temph}[1]{\textbf{#1}}
\newcommand{\remove}[1]{}
\newcommand{\udi}[1]{\textcolor{blue}{[Udi says: #1]}}
\newcommand{\claude}[1]{\textcolor{red}{[Claude: #1]}}

% Abbreviation commands
\newcommand{\GLP}{\textsc{GLP}\xspace}
\newcommand{\lp}{logic programs\xspace}
\newcommand{\cp}{Concurrent Prolog\xspace}
\newcommand{\scl}{\textsc{scl}\xspace}
\newcommand{\gsn}{\textsc{gsn}\xspace}

% Math commands
\newcommand{\calV}{\mathcal{V}}
\newcommand{\calG}{\mathcal{G}}
\newcommand{\calT}{\mathcal{T}}
\newcommand{\calF}{\mathcal{F}}
\newcommand{\calR}{\mathbb{R}}
\newcommand{\calN}{\mathbb{N}}
\newcommand{\calA}{\mathcal{A}}
\newcommand{\V}{\mathcal{V}}
\newcommand{\calS}{\mathcal{S}}
\newcommand{\calL}{\mathcal{L}}
\newcommand{\calP}{\mathcal{P}}
\newcommand{\calM}{\mathcal{M}}
\newcommand{\calE}{\mathcal{E}}
\newcommand{\calB}{\mathcal{B}}
\newcommand{\calC}{\mathcal{C}}
\newcommand{\calX}{\mathcal{X}}
\newcommand{\calD}{\mathcal{D}}

% Roman numeral abbreviations
\newcommand{\ia}{\textit{i}}
\newcommand{\ib}{\textit{ii}}
\newcommand{\ic}{\textit{iii}}
\newcommand{\id}{\textit{iv}}
\newcommand{\iie}{\textit{v}}
\newcommand{\iif}{\textit{vi}}
\newcommand{\iiv}{\textit{iv}}
\newcommand{\iv}{\textit{v}}

% Program counter for examples
\newcounter{pc}
\newcommand\spc{\addtocounter{pc}{1}\thepc}
\newcommand{\Program}[1]{\medskip\noindent\textbf{Program \spc: #1}\vspace{-5pt}}

% Set list spacing
\setlist{nosep, leftmargin=*}
\setlist{itemsep=1pt, topsep=3pt, leftmargin=*}

\newtheorem{observation}{Observation}

\title{GLP: A Grassroots, Multiagent, Concurrent,\\ Logic Programming Language}

\author[Ehud Shapiro]{Ehud Shapiro\\
London School of Economics and Weizmann Institute of Science\\
(e-mail: ehud.shapiro@weizmann.ac.il)}



%\begin{keywords}
%concurrent logic programming, grassroots platforms, multiagent systems, transition systems, implementation hierarchy
%\end{keywords}

\begin{abstract}
A digital platform is \emph{grassroots} if it can have multiple instances that operate independently of any global resource other than the network, and may coalesce into ever larger instances, possibly resulting in a single global instance.

Grassroots Logic Programs (GLP) is a multiagent concurrent logic programming language designed to implement grassroots platforms. Here, we review the mathematical foundation of grassroots platforms, define GLP, prove that it is grassroots, and demonstrate its utility in implementing grassroots platforms via examples of GLP programs for grassroots social networks and grassroots cryptocurrencies.
\end{abstract}



\begin{document}
\maketitle
% Main sections
\section{Introduction}
\mypara{Grassroots platforms} Grassroots platforms~\cite{shapiro2023grassrootsBA} are distributed applications in which multiple disjoint platform instances emerge independently and coalesce when they interoperate. They are run by people on their networked personal devices (today—smartphones), who are identified cryptographically~\cite{rivest1978method}, communicate only with authenticated friends, and can participate in multiple instances of multiple grassroots platforms simultaneously. The grassroots social graph~\cite{shapiro2023gsn} is both a platform in its own right and the infrastructure layer for all other grassroots platforms. In it, nodes represent people, edges—authenticated friendships, and connected components arise spontaneously and interconnect through befriending. The social graph provides grassroots platforms with communication along graph edges, encrypted for ensuring privacy, signed for authenticity and attested for integrity. Upon this foundation, grassroots social networks~\cite{shapiro2023gsn}, grassroots cryptocurrencies~\cite{shapiro2024gc}, and grassroots democratic federations~\cite{shapiro2025GF} are built.


\mypara{Programming grassroots platforms}  A key challenge in implementing grassroots platforms is overcoming faulty and malicious participants~\cite{lamport1982byzantine}. Without secure language support, correct participants cannot reliably identify each other, establish secure communication channels, or verify each other's code integrity~\cite{sabt2015trusted,costan2016intel}. 
While grassroots platforms have been formally specified and their properties  proven~\cite{shapiro2023grassrootsBA,shapiro2023gsn,shapiro2024gc,shapiro2025GF,shapiro2025atomic}, they are so far mathematical constructions without an actual implementation. To the best of our knowledge, no existing programming language provides the necessary combination of distributed execution, cryptographic security, safety, and liveness guarantees required to realize these specifications. GLP aim to close the gap between the mathematical specifications and actual implementation of grassroots platforms.

\mypara{A grassroots programming language} Our goal is to design a high-level, secure, multiagent, concurrent programming language suitable for the implementation of grassroots platforms. To do so, the language should address:
\begin{enumerate}
\item Mutual authentication~\cite{boyd2003protocols} enabling people to identify each other and verify each other's code identity and integrity
\item Grassroots social graph formation through both cold calls (for bootstrap and connecting disconnected components) and friend-mediated introductions
\item Secure communication among friends
\item Multiagent operational semantics~\cite{shapiro2021multiagent} with proven security, safety and liveness
\item Useful abstractions for distributed multiagent programming in general, and metaprogramming support in particular, to enable the development of programming tools and runtime support for the language within the language.
%~\cite{safta1988,shapiro1984systems}.
\end{enumerate}


\mypara{GLP} We present GLP, a secure, multiagent, concurrent, logic programming language designed for implementing grassroots platforms. 
GLP extends logic programs~\cite{lloyd1987foundations,sterling1994art} with paired single-reader/single-writer variables (akin to futures and promises~\cite{dauth2019futures,azadbakht2020formal}), each establishing a secure single-message communication channel between the single writer and the single reader, enabling subsequent secure multidirectional communication by sharing readers and writers in messages.

Through signed attestations at the language level, participants verify each other's identity and code integrity when befriending and communicating. These mechanisms enable both cold calls (for bootstrap and connecting disconnected components) and friend-mediated introductions (the preferred trust propagation method).

We present GLP and prove their properties in five steps, injecting illustrative programming examples along the way: 
\begin{enumerate}
\item \textbf{Logic Programs:} Define a transition system-based operational semantics for logic programs (LP)~\cite{lloyd1987foundations,sterling1994art}, in which a conjunctive goal (resolvent) is transformed by nondeterministic goal/clause reductions. 

\item \textbf{Concurrent GLP:} Extend LP with reader/writer pairs, which must satisfy the Single-Reader/Single-Writer requirement; extend unification to suspend upon an attempt to bind a reader; extend configurations to include pending assignments to readers; extend transitions to include the application of an assignment from a writer to its paired reader, and thus provide nondeterministic interleaving-based asynchronous operational semantics for concurrent GLP. Prove safety properties~\cite{alpern1985defining}, including that GLP computations are deductions~\cite{kowalski1974predicate,lloyd1987foundations}.  Provide GLP with deterministic `workstation implementation-ready' transition system (Appendix~\ref{appendix:irGLP}), based on which a workstation implementation of GLP can be developed to support GLP program development.

\item \textbf{Multiagent, Concurrent GLP:} Employ multiagent transition systems~\cite{shapiro2021multiagent} with atomic transactions~\cite{shapiro2025atomic} to define the operational semantics of multiagent concurrent GLP, in which goal reductions are local and assignments of shared logic variables are realized as writer-to-reader messages among agents, and prove it to be grassroots~\cite{shapiro2023grassrootsBA}. 

\item \textbf{Secure, Multiagent, Concurrent GLP:} Augment agents with self-chosen keypairs and augment cross-agent communication that is encrypted, signed and attested, resulting in secure, multiagent, concurrent GLP. Prove its security as a distributed systems~\cite{coulouris2011distributed} and that GLP streams enjoy the security properties of blockchains~\cite{nakamoto2008peer}.

\item \textbf{Implementation-Ready Specification:} Replace nondeterministic goal selection with deterministic scheduling, and replace abstract push-based shared-variable communication with pull-based message-passing using dynamic shared-variable tables, geared for smartphone deployment.
\end{enumerate}

The remainder of this paper is organized as follows. Section~\ref{section:lp} recalls logic programs. Section~\ref{section:GLP} extends them to concurrent GLP. Section~\ref{section:programming-examples} presents basic GLP programming techniques. Section~\ref{section:maGLP} defines multiagent GLP and proves it grassroots. Section~\ref{section:social-graph} implements the grassroots social graph cold-call and friend-mediated introduction protocols. Section~\ref{section:GLP-security} adds cryptographic security and attestations and presents security properties, including blockchain security properties of streams. Section~\ref{section:implementation} discusses smartphone implementation. Section~\ref{section:related-work} reviews related work, and Section \ref{section:conclusion} concludes.
The appendixes provide~\ref{appendix:LP-synax} LP syntax, ~\ref{appendix:proofs} proofs, ~\ref{appendix:friend-introductions} social-graph protocol properties,  ~\ref{appendix:social-networking} grassroots social networking, 
~\ref{appendix:guards-system} guard and system predicates,
~\ref{appendix:additional-techniques} additional programming and metaprogramming examples, and ~\ref{appendix:irGLP} single-workstation and~\ref{appendix:irmaGLP}  networked-smartphones implementation-ready specifications of GLP.



%==============================================================================
\section{Concurrent GLP}
\label{sec:glp}
%==============================================================================

This section presents Grassroots Logic Programs (GLP), a concurrent logic programming language. We begin with transition systems, recall logic programs (LP) as transition systems, and then extend LP to GLP. We illustrate GLP with programming examples and conclude with the grassroots social graph---the foundational platform that all other grassroots platforms build upon.

%------------------------------------------------------------------------------
\subsection{Transition Systems}
\label{sec:ts}
%------------------------------------------------------------------------------

\begin{definition}[Transition System, Computation, Run]\label{def:ts}
A \temph{transition system} is a tuple $TS=(S,s_0,T)$, where $S$ is a non-empty set of \temph{states}, $s_0\in S$ is the \temph{initial state}, and $T\subset S^2$ is a set of \temph{transitions}, where each transition $t\in T$ is a pair $(s,s')$ of distinct states, written $s\rightarrow s'$.
A \temph{computation} is a sequence of states $s_1,s_2,\ldots$ such that $s_i\rightarrow s_{i+1} \in T$ for each consecutive pair. A \temph{run} is a computation starting from $s_0$.
\end{definition}

%------------------------------------------------------------------------------
\subsection{Logic Programs as Transition Systems}
\label{sec:lp-ts}
%------------------------------------------------------------------------------

We assume familiarity with standard Logic Programs (LP): terms, goals, clauses, substitutions, and most-general unifiers (mgu). We cast LP operationally as a transition system.

\begin{definition}[LP Transition System]
\label{def:lp-ts}
Given a logic program $P$ and initial goal $G_0$, the \temph{LP transition system} $LP(P) = (C, c_0, T)$ has configurations $C = \mathcal{G}(P) \times \Sigma$ (goal--substitution pairs), initial configuration $c_0 = (G_0, \emptyset)$, and transitions $(G,\sigma) \rightarrow (G',\sigma')$ such that for some unit goal $A \in G$ and clause $H$ \verb|:-| $B \in P$ (renamed apart from $G$), $A$ and $H$ have mgu $\hat\sigma$, $G' = (G \setminus \{A\} \cup B)\hat\sigma$, and $\sigma' = \sigma \circ \hat\sigma$.
\end{definition}

LP has two forms of nondeterminism: choice of goal $A \in G$ (\emph{and-nondeterminism}) and choice of clause $C \in P$ (\emph{or-nondeterminism}).

%------------------------------------------------------------------------------
\subsection{GLP: Extending LP with Reader/Writer Variables}
\label{sec:glp-ext}
%------------------------------------------------------------------------------

Grassroots Logic Programs (GLP) extend LP by (1) adding a paired \emph{reader} $X?$ to every ``ordinary'' logic variable $X$, now called a \emph{writer}, (2) restricting variables in goals and clauses to have at most a single occurrence (SO), (3) requiring that a variable occurs in a clause iff its paired variable also occurs in it (single-reader single-writer, SRSW). The result eschews unification in favour of simple term matching, is linear-logic-like, and is futures/promises-like: each assignment $X := T$ is produced at most once via the sole writer (promise) $X$, and consumed at most once via its sole paired reader (future) $X?$.

\begin{definition}[GLP Variables]
\label{def:glp-variables}
Let $\calV$ be the set of LP variables, henceforth called \temph{writers}. Define $\calV? = \{X? \mid X \in \calV\}$, called \temph{readers}. The set of all GLP variables is $\hat\calV = \calV \cup \calV?$. A writer $X$ and its reader $X?$ form a \temph{variable pair}.
\end{definition}

\begin{definition}[Single-Occurrence (SO) and Single-Reader/Single-Writer (SRSW)]
\label{def:so-srsw}
A term, goal, or clause satisfies \temph{SO} if every variable occurs in it at most once.
A clause $C$ satisfies \temph{SRSW} if it satisfies SO and a variable occurs in $C$ iff its paired variable also occurs in $C$.
A \temph{GLP program} is a finite set of clauses satisfying SRSW.
\end{definition}

The SO invariant is preserved: reducing a goal satisfying SO with a clause satisfying SRSW yields a goal satisfying SO (Proposition~\ref{prop:so-preservation}). The SRSW restriction prevents multiple writers racing to bind a variable.

\begin{example}[Fair Merge]
\label{ex:merge}
The quintessential concurrent logic program for fairly merging two streams:
\begin{verbatim}
merge([X|Xs], Ys, [X?|Zs?]) :- merge(Ys?, Xs?, Zs).
merge(Xs, [Y|Ys], [Y?|Zs?]) :- merge(Xs?, Ys?, Zs).
merge(Xs, [], Xs?).
merge([], Ys, Ys?).
\end{verbatim}
Each clause satisfies SRSW. The first two clauses swap inputs in recursive calls, ensuring fairness when both streams are available.
\end{example}

%------------------------------------------------------------------------------
\subsection{GLP Operational Semantics}
\label{sec:glp-operational}
%------------------------------------------------------------------------------

\begin{definition}[Writer and Reader Assignments]
\label{def:assignments}
A \temph{writer assignment} $X := T$ has $X \in \calV$ and $T \notin \calV$ satisfying SO.
A \temph{reader assignment} $X? := T$ has $X? \in \calV?$ and $T \notin \calV$ satisfying SO.
Given a writer assignment $X := T$, its \temph{readers counterpart} is $X? := T$.
\end{definition}

\begin{definition}[GLP Transition System]
\label{def:glp-ts}
Given a GLP program $P$ and initial goal $G_0$ satisfying SO, the \temph{GLP transition system} $GLP(P) = (\calC, c_0, \calT)$ has:
\begin{itemize}
    \item Configurations $\calC$: pairs $(G, \sigma)$ where $G$ is a goal and $\sigma$ is a readers substitution
    \item Initial configuration $c_0 = (G_0, \emptyset)$
    \item Transitions $(G, \sigma) \rightarrow (G', \sigma')$ of two kinds:
    \begin{enumerate}
        \item \textbf{Reduce:} For unit goal $A \in G$, clause $C \in P$ is the \emph{first} clause for which the writer mgu of $A$ and (renamed) head $H$ succeeds with $(B, \hat\sigma)$; then $G' = (G \setminus \{A\} \cup B)\hat\sigma$ and $\sigma' = \sigma \circ \hat\sigma?$
        \item \textbf{Communicate:} $\{X := T\} \in \sigma$, $X? \in G$, $G' = G\{X? := T\}$, $\sigma' = \sigma$
    \end{enumerate}
\end{itemize}
\end{definition}

GLP differs from LP in two key ways: (1) using writer mgu instead of regular mgu---if matching requires binding a reader, the goal \emph{suspends} rather than fails; (2) choosing the \emph{first} applicable clause, enabling fair concurrent programs like merge.

The Communicate transition realizes asynchronous message passing: when a writer $X$ is bound to $T$, the reader assignment $X? := T$ is added to $\sigma$; later, when $X?$ occurs in a goal, the assignment is applied.

%------------------------------------------------------------------------------
\subsection{Term Matching}
\label{sec:term-matching}
%------------------------------------------------------------------------------

The SO invariant allows eschewing unification in favour of \emph{term matching}:

\begin{definition}[Term Matching]
\label{def:term-matching}
Given terms $T_1$ and $T_2$ jointly satisfying SO, their \temph{term matching} traverses both term-trees jointly:
\begin{center}
\begin{tabular}{l|lll}
$T_1 \backslash T_2$ & Writer $X_2$ & Reader $X_2?$ & Term $f_2/n_2$ \\
\hline
Writer $X_1$ & fail & $X_1 := X_2?$ & $X_1 := T_2$ \\
Reader $X_1?$ & $X_2 := X_1?$ & fail & suspend on $X_1?$\\
Term $f_1/n_1$ & $X_2 := T_1$ & fail & fail if $f_1 \ne f_2$ or $n_1 \ne n_2$\\
\end{tabular}
\end{center}
The writer mgu is the union of all writer assignments if no fail occurred and the suspension set is empty.
\end{definition}

%------------------------------------------------------------------------------
\subsection{Guards}
\label{sec:guards}
%------------------------------------------------------------------------------

GLP clauses may include \emph{guards}---tests that determine clause applicability.

\begin{definition}[Guarded Clause]
\label{def:guarded-clause}
A \temph{guarded clause} has the form $H$ \verb|:-| $G$ \verb"|" $B$, where $H$ is the head, $G$ is a guard conjunction, and $B$ is the body.
\end{definition}

Guards have three-valued semantics: \emph{succeed}, \emph{suspend} (may succeed after further instantiation), or \emph{fail} (can never succeed). A clause is applicable only when all guards succeed; if any guard fails, the next clause is tried; if a guard suspends, clause selection suspends.

Guard predicates include type tests (\verb|integer|, \verb|ground|, \verb|known|, etc.), arithmetic comparisons (\verb|<|, \verb|=:=|, etc.), and ground equality (\verb|=?=|).

%------------------------------------------------------------------------------
\subsection{Safety Properties}
\label{sec:glp-safety}
%------------------------------------------------------------------------------

\begin{proposition}[GLP Computation is Deduction]
\label{prop:glp-deduction}
Let $/?$ replace every reader by its paired writer. If $(G_0$ \verb|:-| $G_n)\sigma$ is the outcome of a proper GLP run, then $(G_0$ \verb|:-| $G_n)\sigma/?$ is a logical consequence of $P/?$.
\end{proposition}

\begin{proposition}[SO Preservation]
\label{prop:so-preservation}
If the initial goal $G_0$ satisfies SO, then every goal in the GLP run satisfies SO.
\end{proposition}

\begin{proposition}[Monotonicity]
\label{prop:glp-monotonicity}
If unit goal $A$ can reduce with clause $C$ at step $i$, then either $A$ has been reduced by step $j > i$, or an instance of $A$ can still reduce with $C$ at step $j$.
\end{proposition}

Proofs appear in Appendix~\ref{appendix:proofs}.

%------------------------------------------------------------------------------
\subsection{Programming Examples}
\label{sec:examples}
%------------------------------------------------------------------------------

\begin{example}[Stream Distribution]
\label{ex:distribute}
Broadcasting to multiple consumers uses the \verb|ground| guard to enable safe replication:
\begin{verbatim}
distribute([X|Xs], Ys1, Ys2) :- ground(X?) |
    Ys1 = [X?|Ys1a], Ys2 = [X?|Ys2a], distribute(Xs?, Ys1a?, Ys2a?).
distribute([], [], []).
\end{verbatim}
When \verb|X?| is ground, multiple occurrences in the body do not violate SRSW.
\end{example}

\begin{example}[Lookup in Association List]
\label{ex:lookup}
\begin{verbatim}
lookup(Key, [(K, Value)|_], Value?) :- Key? =?= K? | true.
lookup(Key, [_|Rest], Value?) :- otherwise | lookup(Key?, Rest?, Value).
\end{verbatim}
The \verb|=?=| guard tests ground equality; \verb|otherwise| succeeds when no prior clause applies.
\end{example}

\begin{example}[Deferred Message Injection]
\label{ex:inject}
The \verb|inject| predicate inserts a message into a stream when a trigger variable becomes bound:
\begin{verbatim}
inject(Trigger, Msg, Ys, [Msg?|Ys?]) :- known(Trigger?) | true.
inject(Trigger, Msg, [Y|Ys], [Y?|Ys1?]) :- 
    otherwise | inject(Trigger?, Msg?, Ys?, Ys1).
\end{verbatim}
Until \verb|Trigger| is bound, messages pass through unchanged; when bound, \verb|Msg| is inserted.
\end{example}

%------------------------------------------------------------------------------
\subsection{The Grassroots Social Graph}
\label{sec:social-graph}
%------------------------------------------------------------------------------

The grassroots social graph is the foundational platform upon which all other grassroots platforms are built. Nodes represent cryptographically-identified agents; edges represent authenticated bidirectional channels; connected components arise spontaneously through befriending.

We present the social graph as a single-agent GLP program, using a \emph{network switch} to simulate communication between agents. This demonstrates the program structure before introducing multiagent GLP in Section~\ref{sec:maglp}.

\subsubsection{Agent Structure}

Each agent processes messages from a unified input stream, maintaining a friends list that maps names to output streams:

\begin{verbatim}
agent(Id, ch(UserIn, UserOut), ch(NetIn, NetOut)) :-
    merge(UserIn?, NetIn?, In),
    social_graph(Id?, In?, [(user, UserOut), (net, NetOut)]).
\end{verbatim}

The agent merges user and network input streams, then enters the main event loop \verb|social_graph| with the agent's identity, merged input, and initial friends list containing channels to the user interface and network.

\subsubsection{Cold-Call Befriending Protocol}

The cold-call protocol enables agents to establish friendship without prior shared variables. Figure~\ref{fig:cold-call} shows the protocol clauses.

\begin{figure*}[t]
\begin{verbatim}
%% User requests connection
social_graph(Id, [msg(user, Id?, connect(Target))|In], Fs) :-
    ground(Id?), ground(Target?) |
    lookup_send(net, msg(Id?, Target?, intro(Id?, Id?, Resp)), Fs?, Fs1),
    inject(Resp?, msg(Target?, Id?, response(Resp?)), In?, In1),
    social_graph(Id?, In1?, Fs1?).

%% Receive introduction offer
social_graph(Id, [msg(From, Id?, intro(From?, From?, Resp))|In], Fs) :-
    ground(Id?), ground(From?) |
    lookup_send(user, msg(agent, user, befriend(From?, Resp?)), Fs?, Fs1),
    social_graph(Id?, In?, Fs1?).

%% User accepts/rejects
social_graph(Id, [msg(user, Id?, decision(Dec, From, Resp?))|In], Fs) :-
    ground(Id?) |
    bind_response(Dec?, From?, Resp, Fs?, Fs1, In?, In1),
    social_graph(Id?, In1?, Fs1?).

%% Process response to our introduction
social_graph(Id, [msg(From, Id?, response(Resp))|In], Fs) :-
    ground(Id?) |
    handle_response(Resp?, From?, Fs?, Fs1, In?, In1),
    social_graph(Id?, In1?, Fs1?).
\end{verbatim}
\caption{Cold-call befriending protocol}
\label{fig:cold-call}
\end{figure*}

\subsubsection{Channel Establishment}

When an offer is accepted, both agents establish symmetric channels (Figure~\ref{fig:channel-establish}).

\begin{figure*}[t]
\begin{verbatim}
bind_response(yes, From, accept(ch(FOut?, FIn)), Fs, Fs1?, In, In1?) :-
    handle_response(accept(ch(FIn?, FOut)), From?, Fs?, Fs1, In?, In1).
bind_response(no, _, no, Fs, Fs?, In, In?).

handle_response(accept(ch(FIn, FOut)), From, Fs, [(From?, FOut?)|Fs?], In, In1?) :-
    ground(From?) |
    tag_stream(From?, FIn?, Tagged),
    merge(In?, Tagged?, In1).
handle_response(no, _, Fs, Fs?, In, In?).
\end{verbatim}
\caption{Channel establishment}
\label{fig:channel-establish}
\end{figure*}

The channel pair \verb|ch(FOut?, FIn)| and \verb|ch(FIn?, FOut)| are complementary: each agent's input is the other's output.

\subsubsection{Friend-Mediated Introduction}

Once agents are friends, they can introduce each other to third parties. The introducer creates a fresh channel pair and sends each half to the respective parties (Figure~\ref{fig:friend-intro}).

\begin{figure*}[t]
\begin{verbatim}
%% User requests introduction: introduce(P, Q)
social_graph(Id, [introduce(P, Q)|In], Fs) :-
    ground(Id?), ground(P?), ground(Q?) |
    lookup_send(P?, msg(Id?, P?, intro(Q?, ch(QtoP?, PtoQ))), Fs?, Fs1, _),
    lookup_send(Q?, msg(Id?, Q?, intro(P?, ch(PtoQ?, QtoP))), Fs1?, Fs2, _),
    social_graph(Id?, In?, Fs2?).

%% Receive introduction offer from friend
social_graph(Id, [msg(From, Id?, intro(Other, Ch))|In], Fs) :-
    ground(Id?), ground(From?), ground(Other?) |
    lookup_send(user, intro_offer(From?, Other?, Ch?), Fs?, Fs1, _),
    social_graph(Id?, In?, Fs1?).

%% User accepts introduction
social_graph(Id, [accept_intro(Other, ch(FIn, FOut))|In], Fs) :-
    ground(Id?), ground(Other?) |
    tag_stream(Other?, FIn?, Tagged),
    merge(In?, Tagged?, In1),
    social_graph(Id?, In1?, [(Other?, FOut?)|Fs?]).
\end{verbatim}
\caption{Friend-mediated introduction protocol}
\label{fig:friend-intro}
\end{figure*}

When Bob types \verb|introduce(alice, charlie)|, he creates a channel pair with writers \verb|PtoQ| and \verb|QtoP|. Alice receives \verb|ch(QtoP?, PtoQ)|---she reads from Charlie via \verb|QtoP?| and writes to Charlie via \verb|PtoQ|. Charlie receives the complementary \verb|ch(PtoQ?, QtoP)|. When both accept, they become direct friends without Bob's further involvement.

\subsubsection{Network Switch Simulation}

In deployment, agents communicate through a physical network. In simulation, a \verb|network| process routes messages between agents. Figure~\ref{fig:network-switch} shows switches for two and three agents.

\begin{figure*}[t]
\begin{verbatim}
%% Two-agent network switch
network2((alice, ch([msg(alice, bob, X)|AliceIn], AliceOut?)),
         (bob, ch(BobIn, [msg(alice, bob, X?)|BobOut?]))) :-
    network2((alice, ch(AliceIn?, AliceOut)), (bob, ch(BobIn?, BobOut))).
network2((alice, ch(AliceIn, [msg(bob, alice, X?)|AliceOut?])),
         (bob, ch([msg(bob, alice, X)|BobIn], BobOut?))) :-
    network2((alice, ch(AliceIn?, AliceOut)), (bob, ch(BobIn?, BobOut))).

%% Three-agent network switch (two representative clauses; four more follow the same structure)
network3((alice, ChA), (bob, ChB), (charlie, ChC)) :-
    ChA? = ch([msg(alice, bob, X)|InA], OutA), ChB? = ch(InB, [msg(alice, bob, X?)|OutB]) |
    network3((alice, ch(InA?, OutA?)), (bob, ch(InB?, OutB)), (charlie, ChC?)).
network3((alice, ChA), (bob, ChB), (charlie, ChC)) :-
    ChB? = ch([msg(bob, charlie, X)|InB], OutB), ChC? = ch(InC, [msg(bob, charlie, X?)|OutC]) |
    network3((alice, ChA?), (bob, ch(InB?, OutB?)), (charlie, ch(InC?, OutC))).
\end{verbatim}
\caption{Network switch simulation for two and three agents}
\label{fig:network-switch}
\end{figure*}

The network switch simulates the Network transaction introduced in Section~\ref{sec:maglp}.

\subsubsection{Complete Plays}

A \emph{play} simulates a scenario by spawning agents and driving them with scripted \emph{actors}. Figure~\ref{fig:plays} shows two plays: a cold-call where Alice initiates contact with Bob, and a friend-mediated introduction where Bob introduces Alice to Charlie.

\begin{figure*}[t]
\begin{verbatim}
%% Cold-call play: Alice initiates contact with Bob
play_cold_call :-
    network2((alice, ch(AliceNetOut?, AliceNetIn)), (bob, ch(BobNetOut?, BobNetIn))),
    agent(alice, ch(AliceUserIn?, AliceUserOut), ch(AliceNetIn?, AliceNetOut)),
    agent(bob, ch(BobUserIn?, BobUserOut), ch(BobNetIn?, BobNetOut)),
    alice_actor(AliceUserOut?, AliceUserIn), bob_actor(BobUserOut?, BobUserIn).
alice_actor(_, [msg(user, alice, connect(bob))]).
bob_actor([msg(agent, user, befriend(From, Resp))|_], [msg(user, bob, decision(yes, From?, Resp?))|_]).

%% Friend-mediated introduction play: Bob introduces Alice to Charlie
play_introduction :-
    network3((alice, ch(AliceNetOut?, AliceNetIn)), (bob, ch(BobNetOut?, BobNetIn)),
             (charlie, ch(CharlieNetOut?, CharlieNetIn))),
    agent(alice, ch(AliceUserIn?, AliceUserOut), ch(AliceNetIn?, AliceNetOut)),
    agent(bob, ch(BobUserIn?, BobUserOut), ch(BobNetIn?, BobNetOut)),
    agent(charlie, ch(CharlieUserIn?, CharlieUserOut), ch(CharlieNetIn?, CharlieNetOut)),
    alice_intro_actor(AliceUserOut?, AliceUserIn), bob_intro_actor(BobUserOut?, BobUserIn),
    charlie_intro_actor(CharlieUserOut?, CharlieUserIn).
bob_intro_actor(_, [introduce(alice, charlie)]).
alice_intro_actor([intro_offer(bob, charlie, Ch)|_], [accept_intro(charlie, Ch?)]).
charlie_intro_actor([intro_offer(bob, alice, Ch)|_], [accept_intro(alice, Ch?)]).
\end{verbatim}
\caption{Complete plays for cold-call and friend-mediated introduction}
\label{fig:plays}
\end{figure*}

In the cold-call play, Alice's actor initiates connection; Bob's actor waits for the befriend request and accepts. In the introduction play, Bob's actor initiates the introduction; Alice and Charlie's actors wait for the offer and accept. Both plays terminate with the respective agents as friends.

\input{glp_section_maglp}
\input{glp_section_grassroots}
\section{Related Work}\label{section:related-work}

Grassroots platforms require agents to verify cryptographic identity and protocol compatibility upon contact, form authenticated channels, and coalesce spontaneously without global coordination. The language must support multiple concurrent platform instances and metaprogramming for tooling development. We examine how existing systems address these requirements.


\mypara{Distributed actor and process languages}
Actor-based languages (Erlang/OTP~\cite{armstrong2013programming}, Akka~\cite{akka2022}, Pony~\cite{clebsch2015deny}) and active object languages~\cite{boer2017survey,boer2024active} provide message-passing concurrency and fault isolation. However, their security models operate at the transport layer (TLS in Akka Remote~\cite{akka2022}, Erlang's cookie-based authentication~\cite{armstrong2013programming}) rather than integrating cryptographic identity and code attestation into language primitives. Orleans~\cite{orleans2022} assumes trusted runtime environments, lacking the attestation mechanisms required for grassroots platforms where participants must verify code integrity without central coordination.

\mypara{Capability security}
E~\cite{miller2006robust} provides capability-based security through unforgeable object references with automatic encryption. While ensuring object uniqueness and access control, E does not address verifying real-world identity or protocol implementation attestation—distinct requirements for grassroots platforms.

\mypara{Linear types and session types}
Linear types~\cite{wadler1990linear} ensure single-use of resources, similar to GLP's single-writer constraint. However, GLP's SRSW mechanism provides bidirectional pairing—each writer has exactly one reader—enabling authenticated channels without type-level tracking. Session types~\cite{honda1993types} specify communication protocols statically, with implementations in Links~\cite{cooper2007links,lindley2017lightweight}, Rust~\cite{jespersen2015session}, Scala~\cite{scalas2016lightweight}, and Go~\cite{castro2019distributed}. While these verify protocol conformance at compile time, GLP's reader/writer synchronization enforces protocol dynamically through suspension and resumption, and runtime attestation enables participants to verify protocol compatibility when establishing connections between independently-deployed agents.

\mypara{Concurrent coordination languages}
Concurrent ML~\cite{reppy1999concurrent} provides first-class synchronous channels and events. The Join Calculus~\cite{fournet1996reflexive} offers pattern-based synchronization through join patterns. GLP's SRSW variables provide asynchronous communication through reader/writer pairs with the monotonicity property (Proposition~\ref{proposition:GLP-monotonicity}) ensuring suspended goals remain reducible once readers are instantiated. However, neither provides mechanisms for cryptographic identity verification or authenticated channel establishment required for grassroots platforms.


\mypara{Blockchain programming languages}
Smart contract languages like Solidity~\cite{mukhopadhyay2018ethereum} and Move~\cite{move2022} provide deterministic execution and asset safety but assume blockchain infrastructure for identity and consensus. While Scilla~\cite{scilla2018} separates computation from communication similar to GLP's message-passing model, it targets on-chain state transitions rather than peer-to-peer authenticated channels. GLP achieves blockchain security properties (Section~\ref{subsection:blockchain-security}) through the language-level SRSW invariant and attestations, without requiring global consensus.


\mypara{Authorization languages}
OPA/Rego~\cite{opa2021} and Cedar~\cite{hicks2023cedar} provide declarative policy specification but are specialized for policy evaluation. They consume authentication tokens as inputs but do not integrate attestation as first-class primitives for verifying remote code execution.

\mypara{Concurrent logic programming}
Concurrent logic programming languages~\cite{shapiro1989family} extend logic programming with shared variables for process synchronization. Concurrent Prolog~\cite{shapiro1987concurrent} introduced reader/writer variables, while PARLOG~\cite{clark1986parlog} and GHC~\cite{ueda1986guarded} used mode declarations. Unlike these, GLP enforces Single-Reader/Single-Writer (SRSW) restriction where each variable occurs at most once, establishing secure point-to-point channels through exclusive reader/writer pairs. This enables authenticated messaging while eliminating distributed atomic unification~\cite{kleinman1990distributed}. GLP's homoiconic nature inherits logic programming's metaprogramming capabilities~\cite{safra1988meta,lichtenstein1988concurrent,shapiro1984systems}, essential for platform tooling development.


\section{Conclusion}\label{section:conclusion}
We have presented secure, multiagent, concurrent GLP, argued for its utility for implementing grassroots platforms, and provided workstation and smartphone implementation-ready specifications for it. The next step is to implement it.


\bibliography{bib}

\appendix



\bibliographystyle{acmtrans}
\bibliography{bib}

\appendix

% Appendices
\section{Proofs}\label{appendix:proofs}

\LPComputationisDeduction*
\begin{proof}
We prove by induction on the length of the run that each step preserves logical consequence.

\mypara{Base case} For $n=0$, we have $G_0 = G_0$ with empty substitution $\epsilon$. The outcome $(G_0 \verb|:-| G_0)$ is a tautology, hence a logical consequence of any program.

\mypara{Inductive step} Assume the proposition holds for runs of length $k$. Consider a proper run of length $k+1$:
$$\rho: G_0 \xrightarrow{\sigma_1} \cdots \xrightarrow{\sigma_k} G_k \xrightarrow{\sigma_{k+1}} G_{k+1}$$

By the inductive hypothesis, $(G_0 \verb|:-| G_k)\sigma'$ is a logical consequence of $M$, where $\sigma' = \sigma_1 \circ \cdots \circ \sigma_k$.

For the transition $G_k \xrightarrow{\sigma_{k+1}} G_{k+1}$:
\begin{itemize}
   \item There exists atom $A \in G_k$ and clause $(H \verb|:-| B) \in M$ renamed apart
   \item $\sigma_{k+1}$ is the mgu of $A$ and $H$
   \item $G_{k+1} = (G_k \setminus \{A\} \cup B)\sigma_{k+1}$
\end{itemize}

Since $(H \verb|:-| B)$ is a clause in $M$ and $\sigma_{k+1}$ unifies $A$ with $H$, we know that:
\begin{itemize}
   \item The instance $(H \verb|:-| B)\sigma_{k+1}$ is a logical consequence of $M$ (by instantiation of a program clause)
   \item Since $A\sigma_{k+1} = H\sigma_{k+1}$ (by the mgu property), we can replace $A$ with $B$ under substitution $\sigma_{k+1}$
   \item Therefore, the implication $(G_k \verb|:-| G_{k+1})$ is a logical consequence of $M$ when we consider that $G_{k+1}$ was obtained by replacing $A$ in $G_k$ with $B$ and applying $\sigma_{k+1}$
\end{itemize}

By the transitivity of logical consequence, if $(G_0 \verb|:-| G_k)\sigma'$ is a logical consequence of $M$ and $(G_k \verb|:-| G_{k+1})$ follows from $M$ under the additional substitution $\sigma_{k+1}$, then $(G_0 \verb|:-| G_{k+1})(\sigma' \circ \sigma_{k+1})$ is a logical consequence of $M$.

Since $\sigma = \sigma' \circ \sigma_{k+1} = \sigma_1 \circ \cdots \circ \sigma_{k+1}$, we conclude that the outcome $(G_0 \verb|:-| G_{k+1})\sigma$ is a logical consequence of $M$.
\qed\end{proof}

\ReaderOnlyInstantiation*
\begin{proof}
Consider the transition $G_i \rightarrow G_{i+1}$ via reduction of some atom $A' \in G_i$ with clause $C$. Let $(H \verb|:-| B)$ be the renaming of $C$ apart from $A'$, with writer mgu $\sigma$ and reader counterpart $\sigma?$.

By Definition~\ref{definition:GLP-ts}, the Reduce transition specifies that $G_{i+1} = (G_i \setminus \{A'\} \cup B)\sigma$, and the configuration's reader substitution is updated with $\sigma?$.

For any atom $A \in G_{i+1}$ that also appeared in $G_i$, we have:

\begin{enumerate}
\item $A \neq A'$ (A was not the reduced atom). Then $A \in G_i \setminus \{A'\}$. The reduction applies $\sigma$ to all atoms in the resolvent. Since $A$ was in $G_i$ and the clause was renamed apart from the entire goal (including $A$), any writers in $A$ are distinct from $V_\sigma$. Therefore $\sigma$ does not instantiate variables in $A$. Only the reader counterpart $\sigma?$ can affect $A$. Since $\sigma?$ is a reader substitution with $V_{\sigma?} \subset V?$, we have $A$ in $G_{i+1}$ equals $A'\tau$ where $A' \in G_i$ and $\tau = \sigma?$ instantiates only readers.

\item $A = A'$ (A was the reduced atom). This case cannot occur since $A'$ is removed from the resolvent during reduction and thus cannot appear in $G_{i+1}$.
\end{enumerate}

Therefore, any atom persisting from $G_i$ to $G_{i+1}$ is instantiated only by the reader substitution $\sigma?$.
\qed
\end{proof}

\GLPComputationsareDeductions*
\begin{proof}
Follows from the correspondence between GLP reductions and LP reductions on pure logic variants, combined with Proposition~\ref{proposition:LP-computation-deduction}.
\qed
\end{proof}

\SRSWInvariant*
\begin{proof}
By induction on run length. The base case holds by assumption. For the inductive step, consider $G_i \rightarrow G_{i+1}$ via reduction with clause $C$ renamed apart. The renamed clause has fresh variables satisfying the SRSW syntactic constraint. The reduction replaces atom $A$ with body $B$ and applies $\sigma?$. Since $\sigma?$ replaces variables with terms (eliminating variable occurrences rather than duplicating them), and $B$ has fresh variables distinct from $G_i$, the SRSW invariant is preserved in $G_{i+1}$.
\qed\end{proof}

\Acyclicity* 
\begin{proof}
By induction on run length. For the base case, $G_0$ contains no circular terms by assumption. For the inductive step, assume $G_i$ contains no circular terms and consider the transition $G_i \rightarrow G_{i+1}$ via reduction of atom $A$ with clause $C$. Let $(H \verb|:-| B)$ be the renaming of $C$ apart from $A$, with writer mgu $\sigma$ and reader counterpart $\sigma?$. The reader counterpart exists only if for all $X \in V_\sigma$, $X? \notin X\sigma$ (occurs check). This ensures no writer is bound to a term containing its paired reader. Since $G_{i+1} = (G_i \setminus \{A\} \cup B)\sigma?$, and the occurs check prevents circular assignments, $G_{i+1}$ contains no circular terms.
\qed\end{proof}

\Monotonicity*
\begin{proof}
By induction on $j - i$. For the base case ($j = i$), the atom $A \in G_i$ can reduce with $C$ by assumption. For the inductive step, assume the property holds for $j = k$ and consider $j = k + 1$. 

If $A$ was reduced at some step between $i$ and $k$, then case (1) holds. Otherwise, by the inductive hypothesis, there exists $A' \in G_k$ where $A' = A\tau$ for some reader substitution $\tau$, and $A'$ can reduce with $C$.

Consider the transition $G_k \rightarrow G_{k+1}$. If the reduction involves $A'$, then case (1) holds for $j = k + 1$. If the reduction involves a different atom $B \in G_k$, then $A'$ persists in $G_{k+1}$, possibly further instantiated. Specifically, the reduction applies substitution $\sigma?$ where $\sigma?$ instantiates only readers (by definition of reader counterpart). Thus there exists $A'' \in G_{k+1}$ where $A'' = A'\sigma? = A(\tau \circ \sigma?)$, and $\tau \circ \sigma?$ is a reader substitution.

Since $A'$ could reduce with $C$ (renamed apart) via some writer mgu at step $k$, and $\sigma?$ only instantiates readers, the unification of $A''$ with the head of $C$ (appropriately renamed) still succeeds: reader instantiation preserves unifiability and cannot introduce new writer instantiation requirements. Therefore $A''$ can reduce with $C$ at step $k + 1$.
\qed\end{proof}

\maGLPisgrassroots*    
\begin{proof}
We prove that maGLP is oblivious and interactive.
\begin{enumerate}
    \item \textbf{maGLP is Oblivious:}  Follows directly from Proposition~\ref{proposition:oblivious}.
    \item \textbf{maGLP is Interactive:}  We have to show that in any configuration $c$ of a run of maGLP over $P$, if this configuration is in fact  configuration over $P'\supset P$, then members of $P$ have a behaviour not available to them if this was a run over $P$. The answer, of course, is that in such a case any agent $q\in P'\setminus P$ can send a network message to some agent $p\in P$, resulting in the local state of $p$ having an `alien trace'—a variable produced by an agent not in $P$—a behaviour not available to $P$ on their own.
\end{enumerate}
We conclude that maGLP is grassroots.
\qed\end{proof}


\section{Grassroots Social Graph Protocol Properties}\label{appendix:friend-introductions}

\subsection{Non-blocking Operation Through Variable Synchronization}\label{appendix:social-graph-implementation}

The social graph protocol achieves non-blocking operation through careful use of unbound variables and the \verb|inject| procedure. When initiating connections, agents send offers containing unbound response variables and continue processing other messages while awaiting responses. Similarly, when receiving offers, agents query their users for approval without blocking the main protocol loop. 

The \verb|inject| procedure in Program~\ref{program:social-graph} implements deferred message insertion: when \verb|X| is unbound, \verb|inject| passes input stream messages to its output whilst waiting for \verb|X| to become bound. Once \verb|X| is known, it inserts the message and terminates. This ensures the protocol remains responsive while awaiting responses to connection attempts, preventing any single pending operation from blocking the entire message processing loop.


\subsection{Protocol Properties}

The social graph protocol exhibits several essential properties for grassroots platforms. Non-blocking operation ensures that agents remain responsive during connection establishment, with no single operation capable of indefinitely blocking message processing. Symmetric channel establishment guarantees that successful connections result in bidirectional communication with identical capabilities for both parties. The unified message processing through stream merging provides fair handling of messages from all sources, preventing starvation of any input source.

The protocol's use of unbound variables for response coordination elegantly solves the distributed consensus problem for connection establishment. Both agents must explicitly agree to connect—the offerer by initiating and the receiver by accepting—with the shared response variable serving as the synchronization mechanism. This design ensures that connections only form through mutual consent while avoiding complex state machines or timeout mechanisms.

The friends list serves multiple roles simultaneously: it represents the agent's local view of the social graph, provides the routing table for message sending, and maintains the state needed for friend-mediated introductions. This unified structure simplifies reasoning about the protocol while enabling efficient implementation. The incremental construction of the social graph through individual connections allows multiple disconnected components to form independently and later merge through cross-component connections, embodying the grassroots principle of spontaneous emergence without central coordination.


\section{Social Networking Applications}\label{appendix:social-networking}

Building upon the authenticated social graph, this section demonstrates how GLP enables secure social networking applications. The established friend channels and attestation mechanisms provide verifiable content authorship and provenance guarantees impossible in centralised platforms.

\subsection{Direct Messaging}

Direct messaging establishes dedicated conversation channels between friends, separate from the protocol control channels. When accepting friendship, the acceptor creates a messaging channel and includes it in the acceptance response:

\Program{Direct Messaging Channel Establishment}\label{program:direct-messaging}
\begin{verbatim}
% Modified establishment for direct messaging
% Secure version - verifies DM channel attestation
establish(yes, From, Resp, Fs, Fs1, In, In1) :-
    new_channel(ch(FIn, FOut), FCh),
    new_channel(ch(DMIn, DMOut), DMCh),
    Resp = accept(FCh, DMCh),
    attestation(DMCh, att(From, _)) |  % Verify DM channel from authenticated friend
    handle_friend(From?, FIn?, FOut?, DMIn?, DMOut?, Fs?, Fs1, In?, In1).

handle_friend(From, FIn, FOut, DMIn, DMOut, Fs, 
             [(From, FOut), (dm(From), DMOut)|Fs], In, In1) :-
    tag_stream(From?, FIn?, Tagged),
    merge(In?, Tagged?, In1),
    forward_to_app(dm_channel(From?, DMIn?)).
\end{verbatim}

The protocol maintains separation between control and messaging channels. The friend channel handles protocol messages whilst the direct messaging channel carries conversation data. Each message through the DM channel carries attestation, ensuring non-repudiation and authenticity of the conversation history.

\subsection{Feed Distribution with Verified Authorship}

Content feeds leverage the \verb|ground| guard's relaxation of SRSW constraints to broadcast to multiple followers whilst maintaining cryptographic proof of authorship:

\Program{Authenticated Feed Distribution}\label{program:feed}
\begin{verbatim}
% Post distribution with attestation preservation
post(Content, Followers, Followers1) :-
    ground(Content), current_time(Time) |
    create_post(Content?, Time?, Post),
    broadcast(Post?, Followers?, Followers1).

broadcast(_, [], []).
broadcast(Post, [(Name,Out)|Fs], [(Name,[Post|Out1?])|Fs1]) :-
    broadcast(Post?, Fs?, Fs1).

% Defined guard for attestation preservation  
preserve_attestation(Post, Author, forward(Author?, Post)).

% Forward with attestation verification
forward(Post, Followers, Followers1) :-
    ground(Post), attestation(Post, att(Author, _)),
    preserve_attestation(Post?, Author?, Forward) |
    broadcast(Forward?, Followers?, Followers1).
\end{verbatim}

Each post carries the creator's attestation $(Post)_{M,p,q}$. When forwarding, the original attestation is preserved whilst adding the forwarder's attestation, creating a cryptographically verifiable provenance chain. Recipients can verify both the original author and the complete forwarding path, preventing misattribution and enabling accountability for content distribution.

\subsection{Group Communication}

Groups in GLP follow a founder-administered model where users create groups with selected friends. The founder invites authenticated friends who decide whether to join. Group messages use interlaced streams, creating natural causal ordering without consensus.

\mypara{Group Formation}
Users initiate groups with a name and friend list. The globally unique group identifier is (founder, name), preventing naming conflicts:

\Program{Group Formation Protocol}\label{program:group-formation}
\begin{verbatim}
% User creates group with friend list
social_graph(Id, [msg(user, Id, create_group(Name, Friends))|In], Fs) :-
    create_group_streams([Id|Friends]?, Streams),
    send_invitations(Friends?, Id?, Name?, Streams?, Fs?, Fs1),
    social_graph(Id, In?, [((Id,Name), group(admin, Streams?))|Fs1?]).

% Send invitations through friend channels
send_invitations([], _, _, _, Fs, Fs).
send_invitations([Friend|Friends], Founder, Name, Streams, Fs, Fs1) :-
    lookup(Friend, Fs?, Ch),
    Ch = [inv(Founder?, Name?, Streams?)|Ch1?],
    send_invitations(Friends?, Founder?, Name?, Streams?, [(Friend,Ch1?)|Fs2?], Fs1).

% Receive invitation from friend
social_graph(Id, [msg(From, Id, inv(Founder, Name, Streams))|In], Fs) :-
    attestation(inv(Founder, Name, Streams), att(From, _)) |
    lookup_send(user, msg(agent, user, 
                join_group(From?, Founder?, Name?)), Fs?, Fs1),
    social_graph(Id, In?, Fs1?).

% User decision on invitation
social_graph(Id, [msg(user, Id, join(yes, Founder, Name, Streams))|In], Fs) :-
    social_graph(Id, In?, [((Founder,Name), group(member, Streams?))|Fs?]).
social_graph(Id, [msg(user, Id, join(no, _, _, _))|In], Fs) :-
    social_graph(Id, In?, Fs?).
\end{verbatim}

The founder creates interlaced stream structures for all members and sends invitations through authenticated friend channels. Recipients verify the invitation's attestation before consulting their user. Accepted groups are stored with key (Founder, Name), ensuring uniqueness whilst clarifying ownership.

\mypara{Group Messaging via Interlaced Streams}
Group members maintain independent message streams whilst observing others' messages, creating causal ordering through the interlaced streams mechanism:

\Program{Group Messaging}\label{program:group-messaging}
\begin{verbatim}
% Member participates in group
group_member(Id, (Founder, Name), Streams) :-
    lookup((Founder,Name), Fs?, group(Role, Streams)),
    compose_messages(Id?, Name?, Messages),
    find_my_stream(Id?, Streams?, MyStream),
    interlace(Messages?, MyStream?, [], Streams?).

compose_messages(Id, Name, [Msg|Msgs]) :-
    await_user_input(Id?, Name?, Input),
    format_message(Input?, Id?, Msg),
    compose_messages(Id?, Name?, Msgs?).
compose_messages(_, _, []).

format_message(reply(Text), Id, msg(Id, reply, Text)).
format_message(post(Text), Id, msg(Id, post, Text)).
\end{verbatim}

Members post independently without control tokens. The interlaced streams mechanism (Program~\ref{program:interlaced-streams}) ensures each member's block references all observed messages. When member p replies to message m, the reply appears in p's stream only after p has observed m, creating natural causality where replies follow what they reply to whilst independent messages remain unordered.

Security derives from authenticated friend channels—all group communication occurs through channels established via the social graph protocol, with automatic attestation on every message. Byzantine agents outside the group cannot inject messages as they lack authenticated channels to members. The interlaced structure provides causal consistency equivalent to consensus protocols whilst eliminating their overhead, demonstrating how authenticated channels combined with GLP's concurrent programming primitives enable efficient group communication without centralisation or Byzantine agreement.

\subsection{Content Authenticity and Provenance}

Content authenticity in GLP derives from the attestation mechanism applied recursively through forwarding operations. When agent p creates post P, it carries attestation $(P)_{M,p,*}$. When agent q forwards this post, the forward operation wraps the entire attested post: `forward(p, P)`, which receives attestation $(forward(p,P))_{M,q,*}$. Recipients can verify both q's forwarding attestation and p's original creation attestation, with the nesting depth revealing the complete forwarding chain.

This mechanism addresses three vulnerabilities in conventional social networks. First, impersonation becomes cryptographically impossible—agents cannot forge attestations for other agents' keys. Second, misattribution is prevented—the original author's attestation remains embedded regardless of forwarding depth. Third, conversation manipulation is detectable—group messages through interlaced streams create a tamper-evident partial order where altered histories fail attestation verification. These properties emerge from the language-level integration of attestations with GLP's communication primitives, requiring no external trust infrastructure or consensus protocols.


\section{Guards and System Predicates}\label{appendix:guards-system}

Guards and system predicates extend GLP programs with access to the GLP runtime state, operating system and hardware capabilities.

\mypara{Guard predicates}
Guards provide read-only access to the runtime state of GLP computation. A guard appears after the clause head, separated by \verb=|=, and must be satisfied for the clause to be selected. The following guards are fundamental for concurrent GLP programming:

\begin{itemize}
\item \verb|ground(X)| succeeds if \verb|X| contains no variables. With this guard, the clause body may contain multiple occurrences of \verb|X?| without violating the single-writer requirement, enabling safe replication of ground terms to multiple concurrent consumers.
\item \verb|known(X)| succeeds if \verb|X| is not a variable, though it may not be ground.
\item \verb|writer(X)| and \verb|reader(X)| succeed if  \verb|X| is an uninstantiated writer or reader respectively.  Note that \verb|reader(X)| is non-monotonic.
\item \verb|otherwise| succeeds if all previous clauses for this procedure failed.
\item \verb|X=Y| succeed if $X$ and $Y$ are identical
\item \verb|X=\=Y| succeeds if the unification of $X$ and $Y$ fails. 
\end{itemize}

\mypara{Defined guard predicates}
To support abstract data types and cleaner code organization, GLP provides for user-defined guards, defined unit clauses \verb|p(T1,...,Tn)|.  The call \verb|p(S1,...,Sn)| in the guard is folded to the equalities \verb|T1=S1,...,Tn=Sn| for each unit goal. This mechanism is demonstrated in the channel abstractions below.

\mypara{System predicates}
System predicates execute atomically with goal/clause reduction and provide access to underlying runtime services:
\begin{itemize}
\item \verb|evaluate(Expr?,Result)| evaluates ground arithmetic expressions.
\item \verb|current_time(T)| provides system timestamps for temporal coordination.
\item \verb|variable_name(X,Name)| returns a unique identifier for variable \verb|X| and its pair.
\end{itemize}

\mypara{Arithmetic evaluation in assignments}
Arithmetic expressions are defined by the following clause:
\begin{verbatim}
X? := E :- ground(E) | evaluate(E?,X).
\end{verbatim}
Ensuring the expression is ground before calling the system evaluator, maintaining program safety whilst providing convenient notation for mathematical computations.


\section{Additional Programming Techniques}\label{appendix:additional-techniques}

This appendix presents GLP programs that were referenced in the main text, as well as additional programs that demonstrate the language's capabilities.


\subsection{Channel Abstractions}

Bidirectional channels are fundamental to concurrent communication in GLP. We represent a channel as the term \verb|ch(In?,Out)| where \verb|In?| is the input stream reader and \verb|Out| is the output stream writer. The following predicates encapsulate channel operations and are defined as guard predicates through unit clauses:

\Program{Channel Operations}\label{program:channel-operations}
\begin{verbatim}
send(X,ch(In,[X?|Out?]),ch(In?,Out)).
receive(X?,ch([X|In],Out?),ch(In?,Out)).
new_channel(ch(Xs?,Ys),ch(Ys?,Xs)).
\end{verbatim}

The \verb|send| predicate adds a message to the output stream, \verb|receive| removes a message from the input stream, and \verb|new_channel| creates a pair of channels where each channel's input is paired with the other's output. When used as guards in clause heads, these predicates enable readable code that abstracts the underlying stream mechanics:

\Program{Stream-Channel Relay}\label{program:relay}
\begin{verbatim}
relay(In,Out?,Ch) :- 
    In?=[X|In1], send(X?,Ch?,Ch1) | relay(In1?,Out,Ch1?).
relay(In,Out?,Ch) :- 
    receive(X,Ch?,Ch1), Out=[X?|Out1?] | relay(In?,Out1,Ch1?).
\end{verbatim}

The relay reads from its input stream and sends to the channel in the first clause, while the second clause receives from the channel and writes to the output stream. The channel state threads through the recursive calls, maintaining the bidirectional communication link.

\subsection{Stream Tagging for Source Identification}

When multiple input streams merge into a single stream, the source identity of each message is lost. Stream tagging preserves this information by wrapping each message with its source identifier:

\Program{Stream Tagging}\label{program:tag-stream}
\begin{verbatim}
tag_stream(Name, [M|In], [msg(Name?, M?)|Out]) :-
    tag_stream(Name?, In?, Out?).
tag_stream(_, [], []).
\end{verbatim}

The procedure recursively processes the input stream, wrapping each message \verb|M| in a \verb|msg(Name, M)| term that includes the source name. The tagged stream can then be safely merged with other tagged streams while preserving source information, essential for multiplexed message processing where receivers must determine message origin.

\subsection{Stream Observation}

For non-ground data requiring observation without consumption, the observer technique forwards communication bidirectionally while producing a replicable audit stream:

\Program{Concurrent Observer}\label{program:observer}
\begin{verbatim}
observe(X?, Y, Z) :- observe(Y?, X, Z).
observe(X, X?, X?) :- ground(X) | true.
observe(Xs, [Y1?|Ys1?], [Y2?|Ys2?]) :-
    Xs? = [X|Xs1] |
    observe(X?, Y1, Y2),
    observe(Xs1?, Ys1, Ys2).
\end{verbatim}

\subsection{Cooperative Stream Production}

While the single-writer constraint prevents competitive concurrent updates, GLP enables sophisticated cooperative techniques where multiple producers coordinate through explicit handover:

\Program{Cooperative Producers}\label{program:cooperative}
\begin{verbatim}
producer_a(control(Xs,Next)) :-
    produce_batch_a(Xs,Xs1,Done),
    handover(Done?,Xs1,Next).

producer_b(control(Xs,Next)) :-
    produce_batch_b(Xs,Xs1,Done),
    handover(Done?,Xs1,Next).

handover(done,Xs,control(Xs,Next)).

produce_batch_a([a,b,c|Xs],Xs,done).
produce_batch_b([d,e,f|Xs],Xs,done).
\end{verbatim}

The \verb|control(Xs,Next)| term encapsulates both the stream tail writer and the continuation for transferring control, enabling round-robin production, priority-based handover, or dynamic producer pools.

These examples demonstrate GLP as a powerful concurrent programming language where reader/writer pairs provide natural synchronization, the single-writer constraint ensures race-free concurrent updates, and stream-based communication enables scalable concurrent architectures.

\subsection{Network Switch}

For three agents \verb|p, q ,r| and three channels  with them \verb|Chp, Chq, Chr|, it is initialized with 
\verb|network((p,Chp?),(q,Chq?),(r,Chr?))|.

\Program{3-Way Network Switch}\label{program:3-way-network-switch}
\begin{verbatim}
% P to Q forwarding
network((P,ChP),(Q,ChQ),(R,ChR)) :-
    ground(Q), receive(ChP?,msg(Q,X),ChP1), send(ChQ?,X?,ChQ1) |
    network((P,ChP1?),(Q,ChQ1?),(R,ChR?)).

% P to R forwarding  
network((P,ChP),(Q,ChQ),(R,ChR)) :-
    ground(R), receive(ChP?,msg(R,X),ChP1), send(ChR?,X?,ChR1) |
    network((P,ChP1?),(Q,ChQ?),(R,ChR1?)).

% Q to P forwarding
network((P,ChP),(Q,ChQ),(R,ChR)) :-
    ground(P), receive(ChQ?,msg(P,X),ChQ1), send(ChP?,X?,ChP1) |
    network((P,ChP1?),(Q,ChQ1?),(R,ChR?)).

% Q to R forwarding
network((P,ChP),(Q,ChQ),(R,ChR)) :-
    ground(R), receive(ChQ?,msg(R,X),ChQ1), send(ChR?,X?,ChR1) |
    network((P,ChP?),(Q,ChQ1?),(R,ChR1?)).

% R to P forwarding
network((P,ChP),(Q,ChQ),(R,ChR)) :-
    ground(P), receive(ChR?,msg(P,X),ChR1), send(ChP?,X?,ChP1) |
    network((P,ChP1?),(Q,ChQ?),(R,ChR1?)).

% R to Q forwarding
network((P,ChP),(Q,ChQ),(R,ChR)) :-
    ground(Q), receive(ChR?,msg(Q,X),ChR1), send(ChQ?,X?,ChQ1) |
    network((P,ChP?),(Q,ChQ1?),(R,ChR1?)).
\end{verbatim}

\subsection{Implementation Correctness Properties}

\begin{proposition}[Goal State Integrity]\label{proposition:goal-integrity}
For any configuration $(R_p, V_p, M_p)$ where $R_p = (A_p, S_p, F_p)$ in an IRmaGLP run, every goal of agent $p$ appears in exactly one of $A_p$, $S_p$, or $F_p$. Furthermore, $F_p$ is monotonically increasing: once a goal enters $F_p$, it remains there.
\end{proposition}

\begin{proof}
By induction on transition steps. Initially all goals are in $A_p$. The Reduce transaction (Definition~\ref{definition:IRmaGLP-reduce}) moves goals between sets atomically: from $A_p$ to $S_p$ on suspension, from $S_p$ to $A_p$ on reactivation, and to $F_p$ on failure. No transition removes goals from $F_p$.
\end{proof}

\begin{proposition}[SRSW Preservation in Implementation]\label{proposition:impl-srsw}
If the initial configuration of IRmaGLP satisfies SRSW, then for any reachable configuration and any variable $Y$, at most one agent holds $Y$ locally (in their resolvent) and at most one agent holds $Y'$ locally.
\end{proposition}

\begin{proof}
The variable table $V_p$ tracks all non-local variable references. When agent $p$ exports a variable $Y$ through the export helper (Definition~\ref{definition:IRmaGLP-local-state}), $Y$ is added to $V_p$ marking it as created by $p$ but referenced externally. The Communicate and Network transactions maintain exclusivity by transferring variables between agents rather than duplicating them. The export helper's relay mechanism for requested readers preserves the single-reader property through fresh variable pairs.
\end{proof}

\begin{proposition}[Suspension Correctness]\label{proposition:suspension-correct}
If goal $G$ is suspended on reader set $W$ at agent $p$, then $G$ transitions to active exactly when either: (1) some $X? \in W$ receives a value through a Communicate transaction, or (2) some $X? \in W$ is abandoned.
\end{proposition}

\begin{proof}
The reactivate helper (Definition~\ref{definition:IRmaGLP-local-state}) is called precisely when assignments arrive or abandonment occurs. It removes $(G, W)$ from $S_p$ if $X? \in W$, adding $G$ to the tail of $A_p$. No other operation modifies suspended goals.
\end{proof}

\subsection{Replication of Non-Ground Terms}

While the main text demonstrated distribution of ground terms to multiple consumers, many applications require replicating incrementally-constructed terms that may contain uninstantiated readers. The following replicator procedure handles nested lists and other structured terms, provided the input contains no writers. This technique suspends when encountering readers and resumes as values become available, enabling incremental replication of partially instantiated data structures.

\Program{Non-Ground Term Replicator}\label{program:replicator}
\begin{verbatim}
replicate(X, X?,..., X?) :- 
    ground(X) | true.                          % Ground terms can be shared
replicate(Xs, [Y1?|Ys1?],..., [Yn?|Ysn?]) :-   % List recursion on both parts
    Xs? = [X|Xs1] |
    replicate(X?, Y1,..., Yn),
    replicate(Xs1?, Ys1,..., Ysn).
\end{verbatim}

The replicator operates recursively on list structures, creating multiple copies that maintain the same incremental construction behavior as the original. When the input list head becomes available, all replica heads receive the replicated value simultaneously. This technique extends naturally to tuples through conversion to lists of arguments, enabling replication of arbitrary term structures that contain readers but no writers.

\subsection{Interlaced Streams as Distributed Blocklace}

A blocklace represents a partially-ordered generalization of the blockchain where each block contains references to multiple preceding blocks, forming a directed acyclic graph. This structure maintains the essential properties of blockchains while enabling concurrent block creation without consensus. GLP's concurrent programming model naturally realizes blocklace structures through interlaced streams, where multiple concurrent processes maintain individual streams while observing and referencing each other's progress.

\Program{Interlaced Streams (Blocklace)}\label{program:interlaced-streams}
\begin{verbatim}
% Three agents maintaining interlaced streams
% Initial goal:
%   p(streams(P_stream, [Q_stream?, R_stream?])),
%   q(streams(Q_stream, [P_stream?, R_stream?])),
%   r(streams(R_stream, [P_stream?, Q_stream?]))

streams(MyStream, Others) :-
    produce_payloads(Payloads),
    interlace(Payloads?, MyStream, [], Others?).

interlace([Payload|Payloads], [block(Payload?,Tips?)|Stream?], PrevTips, Others) :-
    collect_new_tips(Others?, Tips, Others1),
    interlace(Payloads?, Stream, Tips?, Others1?).
interlace([], [], _, _).

% Using reader(X) to identify fresh tips not yet incorporated
collect_new_tips([[Block|Bs]|Others], [Block?|Tips?], [Bs?|Others1?]) :-
    reader(Bs) |  % Bs unbound means Block is the current tip
    collect_new_tips(Others?, Tips, Others1).
collect_new_tips([[B|Bs]|Others], Tips?, [[Bs]?|Others1?]) :-
    % Skip B as it's already been referenced
    collect_new_tips([[Bs]?|Others?], Tips, Others1).
collect_new_tips([], [], []).
\end{verbatim}

Each concurrent process maintains its own stream of blocks containing application payloads and references to the most recent blocks observed from other processes. The `reader(X)` guard predicate identifies unprocessed blocks by detecting unbound tail variables, enabling each process to reference exactly those blocks it has not previously incorporated. This creates a distributed acyclic graph structure where the partial ordering reflects the causal relationships between blocks produced by different processes.

The interlaced streams technique demonstrates how GLP's reader/writer synchronization mechanism naturally implements sophisticated distributed data structures. The resulting blocklace provides eventual consistency guarantees similar to CRDTs while maintaining the integrity and non-repudiation properties of blockchain structures. This technique has applications in distributed consensus protocols, collaborative editing systems, and Byzantine fault-tolerant dissemination networks.


\subsection{Metainterpreters}

Program development is essentially a single-agent endeavour:  The programmer trying to write and debug a GLP program.  As in Concurrent Prolog, a key strength of GLP is metainterpretation:  The ability to write GLP interpreters with various functions in GLP.  This allows writing a GLP program development environment and a GLP operating system within GLP itself~\cite{sterling1994art,safra1988meta,shapiro1982algorithmic,lichtenstein1988concurrent,silverman1988logix}, as well as writing a GLP operating system in GLP~\cite{shapiro1984systems}. These two scenarios are the focus of this section:
a programmer developing a program and running it with enhanced metainterpreters that support the various needs of program development, and an operating system written in GLP that supports the execution, monitoring and and control of GLP programs.

\mypara{Plain metainterpreter}
Next we show a plain \GLP metainterpreter.
It follows the standard granularity of logic programming metainterpreters, using the predicate \verb|reduce| to encode each program clause. This approach avoids the need for explicit renaming and, in the case of concurrent logic programs such as GLP also guard evaluation, while maintaining explicit goal reduction and body evaluation. The encoding is such that if in a call to \verb|reduce| a given goal unifies with its first argument then the body is returned in its second argument. Here we show it together with a \verb|reduce| encoding of \verb|merge|.

\Program{\GLP plain metainterpreter}\label{program:meta}
\begin{small}
\begin{verbatim}
run(true).  % halt 
run((A,B)) :- run(A?), run(B?). % fork 
run(A) :- known(A) | reduce(A?,B), run(B?) %  reduce 

reduce(merge([X|Xs],Ys,[X?|Zs?]),merge(Xs?,Ys?,Zs)). 
reduce(merge(Xs,[Y|Ys],[Y?|Zs?]),merge(Xs?,Ys?,Zs)).
reduce(merge([],[],[]),true). 
\end{verbatim}
\end{small}
 
 
For example, when called with an initial goal:
\begin{small}
\begin{verbatim}
run((merge([1,2,3],[4,5],Xs), merge([a,b],[c,d,e],Ys), merge(Xs?,Ys?,Zs)).
\end{verbatim}
\end{small}
after two forks using the second clause of \verb|run|, its goal would become:
\begin{small}
\begin{verbatim}
run((merge([1,2,3],[4,5],Xs)), run(merge([a,b],[c,d,e],Ys)), run(merge(Xs?,Ys?,Zs)).
\end{verbatim}
\end{small}
and its finite run would produce some merge of the four input lists.


\mypara{Fail-safe metainterpreter}
%
The operational semantics of Logic Programs and GLP specifies that a run is aborted once a goal fails.  Following this rule would make impossible the writing in GLP of a metainterpreter that identifies and diagnoses failure.  The following metainterpreter addresses this by assuming that the  representation of the interpreted program ends with the clause:
\begin{verbatim}
reduce(A,failed(A)) :- otherwise | true.
\end{verbatim}
Returning the failed goal \verb|A| as the term \verb|failed(A)| for further processing, the simplest being just reporting the failure, as in the following metainterpreter:

\Program{\GLP fail-safe metainterpreter}\label{program:meta-failsafe}
\begin{small}
\begin{verbatim}
run(true,[]).  % halt 
run((A,B),Zs?) :- run(A?,Xs), run(B?,Ys), merge(Xs?,Ys?,Zs). % fork
run(fail(A),[fail(A?)]).    % report failure 
run(A,Xs?) :- known(A) | reduce(A?,B), run(B?,Xs) %  reduce 
\end{verbatim}
\end{small}

Failure reports can be used to debug a program, but do not prevent a faulty run from running forever.  

\mypara{Metainterpreter with run control}
%
Here we augment the metainterpreter with run control, via which a run can be suspended, resumed, and aborted. 
As control messages are intended to be ground, the control stream of a run can be distributed to all metainterpreter instances that participate in its execution.

\Program{\GLP metainterpreter with run control}\label{program:meta-control}
\begin{small}
\begin{verbatim}
run(true,_).  % halt 
run((A,B),Cs) :- distribute(Cs?,Cs1,Cs2), run(A?,Cs1?), run(B?,Cs1). % fork
run(A,[suspend|Cs]) :- suspended_run(A,Cs?).   % suspend
run(A,Cs) :- known(A) |        % reduce
        distribute(Cs?,Cs1,Cs2), reduce(A?,B,Cs1?), run(B?,Xs,Cs2?). 

suspended_run(A,[resume|Cs]) :- run(A,Cs?).  
suspended_run(A,[abort|Cs]).
\end{verbatim}
\end{small}
The metainterpreter suspends reductions as soon as the control stream is bound to \verb=[suspend|Cs?]=, upon which the run can be resumed or aborted by binding \verb|Cs| accordingly. 
Combining Programs \ref{program:meta-failsafe} and \ref{program:meta-control} would allow the programmer to abort the run as soon as a goal fails.
But we wish to introduce additional capabilities before integrating them all.


\mypara{Termination detection} The following metainterpreter allows the detection of the termination of a concurrent GLP program.  It uses the `short-circuit' technique, in which a chain of paired variables extends while goals fork, contracts when goals terminate, and closes when all goals have terminated.

\Program{\GLP termination-detecting metainterpreter}\label{program:meta-control}
\begin{small}
\begin{verbatim}
run(true,L,L?).  % halt 
run((A,B),Cs,L,R?) :-  run(A?,Cs1?,L?,M), run(B?,Cs1,M?,R). % fork
run(A,L,R?) :- known(A) |        % reduce
        reduce(A?,B,Cs1?), run(B?,Xs,Cs2?,L?,R). 
\end{verbatim}
\end{small}

When called with \verb|run(A,done,R)|, the reader \verb|R?| will be bound to \verb|done| iff the run terminates.  


\mypara{Collecting a snapshot of an aborted run}
The short-circuit technique can be used to extend the metainterpreter with run control to collect a snapshot of the run, if aborted before termination.  Upon abort,  
the resolvent is passed from left to right in the short circuit, with each metainterpreter instance adding their interpreted goal to the growing resolvent.
We only show the \verb|suspended_run| procedure:

\Program{\GLP metainterpreter with run control and snapshot collection}\label{program:meta-control-abort-snapshot}
\begin{small}
\begin{verbatim}
suspended_run(A,[resume|Cs],L,R?) :- run(A,Cs?,L?,R).
suspended_run(A,[abort|_],L,[A?|L?]).
\end{verbatim}
\end{small}

When called with \verb|run(A,Cs?,[],R)|, if \verb|Cs| is bound to \verb|[suspend,abort]|, the reader \verb|R?| will be bound to the current resolvent of the run (which could be empty if the run has already terminated before

Note that taking a snapshot of a suspended run and then resuming it requires extra effort, as two copies of the goal are needed,  a `frozen' one for the snapshot, and a `live' one to continue the run.  Addressing this is necessary for interactive debugging, to allow a developer to watch a program under development as it runs.  We discuss it below.


\mypara{Producing a trace of a run} Tracing a run of a program and then single-stepping through its critical sections are basic debugging techniques, but applying them to concurrent programs is both difficult and less useful due to their nondeterminism.  Here is a metainterpreter that produces a trace of the run, which can then be used by a retracing  metainterpreter to single-step through the very same run, making the same nondeterministic scheduling choices.
It assumes that each program clause \verb= A:- D | B= is represented by a unit clause
\verb=reduce(A,B,I) :- G | true=, with $I$ being the serial number of the clause in the program.

\Program{\GLP a tracing metainterpreter}\label{program:metatree}
\begin{small}
\begin{verbatim}
run(true,true).  % halt 
run((A,B),(TA?,TB?)) :- run(A?,TA), run(B?,TB). % fork 
run(A,((I?:Time?):-TB?)) :- known(A) | 
    time(Time), reduce(A?,B,I), run(B?, TB).
\end{verbatim}
\end{small}


As another example, here is a \GLP metainterpreter, inspired by~\cite{shapiro1984systems}, that can suspend, resume, and abort a \GLP run and produce a dump of the processes of the aborted run. It employs the guard predicate \verb|otherwise|, which succeeds if and only if all previous clauses in the procedure fail (as opposed to suspend). This enables default case handling when no other clause applies.

\Program{\GLP metainterpreter with runtime control}\label{program:runtime}
\begin{small}
\begin{verbatim}
run(true,Cs,L?,L).  % halt and close the dump
run((A,B),Cs,L?,R) :- run(A?,Cs?,L,M?), run(B?,Cs?,M,R?). % fork 
run(A,Cs,L?,R) :- otherwise, unknown(Cs) | reduce(A?,B), run(B?,Cs,L,R?) %  reduce 
run(A,[abort|Cs],[A?|R?],R). % abort and dump 
run(A,[suspend|Cs],L?,R) :- suspended_run(A?,Cs?,L,R?). % suspend

suspended_run(A,[resume|Cs],L?,R) :-  run(A?,Cs?,L,R?).  % resume
suspended_run(A,[C|Cs],L?,R) :- otherwise | run(A?,[C?|Cs?],L,R?).
\end{verbatim}
\end{small}

Its first argument is the process (goal) to be executed, its second argument \verb|Cs| is the observed interrupt stream, and its last two arguments form a `difference-list', a standard logic programming technique~\cite{sterling1994art} by which a list can be accumulated in a distributed way (the program is not fail-stop resilient; it can be extended to be so). 
 



\input{glp_appendix_security}
\section{Workstation Implementation-Ready Transition System for GLP}\label{appendix:irGLP}

This section specifies a workstation (single-agent) implementation-ready transition system for GLP with deterministic execution. 

\begin{definition}[irGLP Configuration]
An \emph{irGLP configuration} over program $M$ is a triple $R = (Q, S, F)$ where:
\begin{itemize}
\item $Q \in \mathcal{A}^*$ is a sequence of active goals
\item $S \subseteq \mathcal{A} \times 2^{V?}$ contains suspended goals with their suspension sets
\item $F \subseteq \mathcal{A}$ contains failed goals
\end{itemize}
\end{definition}

The irGLP reduction extends GLP reduction by activating goals that were suspended on variables instantiated by the reduction, and explicitly failing goals that do not succeed or suspend.

\begin{definition}[irGLP Goal/Queue Reduction]
Given configuration $(Q, S, F)$ with $Q = A\cdot Q'$ and clause $C \in M$, the \emph{irGLP reduction} of $A$ with $C$:
\begin{itemize}
\item \textbf{succeeds with} $(B, \hat\sigma, R)$ if the GLP reduction of $A$ with $C$ succeeds with $(B, \hat\sigma)$ and $R = \{G : (G, W) \in S \wedge X?\in W \wedge X?\hat\sigma? \neq X?\}$
\item \textbf{suspends with} $W_C$ if GLP reduction of $A$ with $C$ suspends on readers $W_C$
\item \textbf{fails} otherwise
\end{itemize}
\end{definition}

\begin{definition}[Implementation-Ready GLP Transition System]
The transition system $\text{irGLP} = (\mathcal{C}, c_0, \mathcal{T})$ over $M$ and initial goal $G_0$ has configurations $\mathcal{C}$ being all irGLP configurations over $M$, with initial configuration $c_0 = (G_0, \emptyset, \emptyset)$, and transitions $\mathcal{T}$ being all transitions $(Q, S, F) \rightarrow (Q', S', F')$ where $Q = A \cdot Q_r$ and:
    \begin{enumerate}
    \item \textbf{Reduce:} If GLP reduction of $A$ with first applicable clause $C \in M$ succeeds with $(B, \hat\sigma,R)$:
        \begin{itemize}
        \item \textbf{Activate:}  $S' = S \setminus \{(G, W) : G \in R\}$,  $F' = F$
        \item \textbf{Schedule:} $Q' = (Q_r \cdot B \cdot R)\hat\sigma\hat\sigma?$
        \end{itemize}
    \item \textbf{Suspend:} Else if $W = \bigcup_{C \in M} W_{C} \neq \emptyset$ then $Q' = Q_r$, $S' = S \cup \{(A, W)\}$, $F' = F$
    \item \textbf{Fail:} Else,  $Q' = Q_r$, $S' = S$, $F' = F \cup \{A\}$.
    \end{enumerate}
\end{definition}

A key restriction compared to the GLP operational semantics is the immediate application of reader substitutions during reduction rather than through asynchronous communication. This simplification is appropriate for workstation execution where all variables are local.


\section{Smartphone Implementation-ready Multiagent Transition System for GLP}\label{appendix:irmaGLP}

This section combines the implementation-ready structure of irGLP (Section~\ref{appendix:irGLP}) with the multiagent framework of maGLP (Section~\ref{section:maGLP}). While irGLP provides deterministic scheduling and suspension management for single agents, and maGLP defines cross-agent communication through shared variables, irmaGLP specifies the concrete data structures and message-passing mechanisms suitable for multiagent smartphone implementation.

A variable $X$ is \emph{local} to agent $p$ if $X$ occurs in $p$'s resolvent. Non-local variables require coordination through variable tables and explicit message passing, replacing maGLP's abstract shared-variable communication with concrete routing mechanisms.

The fundamental invariant: assignments produced by Reduce transactions are immediately applied if the reader is local, otherwise they become messages routed through the variable tables.

\begin{definition}[Implementation-Ready maGLP Transition System]
The implementation-ready maGLP transition system over agents $P \subset \Pi$ and GLP module $M$ is the multiagent transition system $\text{IRmaGLP} = (C, c_0, T)$ where:
\begin{itemize}
\item $C$ is the set of all configurations where for each $p \in P$, the local state $c_p$ is an implementation-ready resolvent as in Definition~\ref{definition:IRmaGLP-local-state}
\item $c_0$ is the initial configuration where for each $p \in P$:
  \begin{itemize}
  \item $R_p = ([\texttt{agent}(p, \texttt{ch}(_?, _), \texttt{ch}(_?, _))], \emptyset, \emptyset)$
  \item $V_p = \emptyset$  
  \item $M_p = \emptyset$
  \end{itemize}
\item $T$ is the union of all transitions generated by:
  \begin{itemize}
  \item Unary Reduce transactions for each $p \in P$ (Definition~\ref{definition:IRmaGLP-reduce})
  \item Binary Communicate transactions for each $(p, q) \in P \times P, p \neq q$ (Definition~\ref{definition:IRmaGLP-communicate})
  \item Binary Network transactions for each $(p, q) \in P \times P, p \neq q$ (Definition~\ref{definition:IRmaGLP-network})
  \end{itemize}
\end{itemize}
\end{definition}
 
\subsection{Local States}

\begin{definition}[Implementation-Ready maGLP Local State]\label{definition:IRmaGLP-local-state}
The local state of agent $p \in \Pi$ is an \temph{implementation-ready resolvent} $s_p = (R_p, V_p,M_p)$ where:
\begin{enumerate}
\item $R_p = (A_p, S_p, F_p)$ separates the resolvent goals into three types:
    \begin{itemize}
    \item \textbf{Active: }$A_p\in \calA^*$ 
    \item \textbf{Suspended:} $S_p \subseteq \mathcal{A} \times 2^{V?}$ 
    \item \textbf{Failed:} $F_p \subseteq \mathcal{A}$
    \end{itemize}
    
   \item $V_p \subseteq \calV \times \Pi \times (\mathcal{T} \cup \Pi \cup \{\bot\})$ maintains shared variable state as a set of triples where each $(Y, q, s) \in V_p$:
        \begin{itemize}
        \item \textbf{Writer:} $Y \in V$,  $s \in \mathcal{T}$ is the value of $Y$, else $s=\bot$
        \item \textbf{Created Reader:} $Y \in V?$, $q = p$, $s \in \Pi$ is the read-requesting agent, else $s=\bot$
        \item  \textbf{Imported Reader:} $Y \in V?$ (reader),  $q \neq p$, $s =q$ indicates a read request has been sent from $p$ to $q$, else  $s=\bot$ 
        \end{itemize}
        
\item $M_p$ is a set of pending messages as pairs (content, destination) where destination $q \in \Pi$:
    \begin{itemize}
    \item assignments $(X?:=T, q)$ 
    \item read requests $(request(X?, p), q)$ where $p$ requests $X?$ from $q$
    \item abandonment notifications $(abandon(X), q)$
    \end{itemize}
\end{enumerate}
\end{definition}

The resolvent $R_p$ partitions goals into three categories. Active goals $A_p$ contains a queue of goals to be reduced in FIFO order. Suspended goals $S_p$ pairs each atom with the set of readers preventing its reduction—for $(A, W) \in S_p$, the set $W$ contains all readers from the suspension sets across all clause attempts. When any reader $X? \in W$ receives a value or is abandoned, $A$ moves to the tail of $A_p$. Failed goals $F_p$ contains atoms for which every reduction attempt either failed outright or suspended only on abandoned variables.

The variable table $V_p$ maintains shared variables where one element of each reader/writer pair is local to $p$ while its counterpart is non-local. For writers, the table stores the creator and any assignment to enable response to read requests. For created readers, it records which agent has requested the value. For imported readers, it tracks whether a read request has been sent to the creator. This unified structure ensures variables referenced by non-local counterparts are not prematurely garbage collected and provides routing information for cross-agent communication.

The variable table $V_p$ maintains an invariant: it contains exactly those variables whose paired counterparts are non-local. When $p$ receives a term containing a variable from $V_p$, that variable becomes local and must be removed from $V_p$. When $p$ exports a term, the export helper function updates $V_p$ accordingly: variables created by $p$ are added when first exported, while variables created by others are removed (except for requested readers which require relay variables).

\mypara{Helper Routines for Implementation-Ready Transactions, agent $p$}

The \texttt{abandon} helper notifies other agents when variable $Y$ becomes unreachable. For imported variables, it notifies the creator $q$. For created readers with a requester $s$, it notifies that requester. The paired variable $Y'$ is sent in the message to indicate which part of the pair was abandoned.

\begin{definition}[routine abandon(Y)]
\begin{itemize}
\item If $(Y, q, s) \in V_p$ where $q \neq p$: remove from $V'_p$ and add $(abandon(Y'), q)$ to $M'_p$
\item If $(Y, p, s) \in V_p$ and $s \neq \bot$: remove from $V'_p$ and add $(abandon(Y'), s)$ to $M'_p$
\item Otherwise: just remove $(Y, \cdot, \cdot)$ from $V'_p$ if present
\end{itemize}
where $Y' = Y?$ if $Y \in V$, else $Y' = Y$ if $Y \in V?$ (the paired variable)
\end{definition}

The \texttt{request} helper sends a read request for an imported reader that hasn't been requested yet. It updates the table entry from $(X?, q, \bot)$ to $(X?, q, q)$ to record that the request was sent, preventing duplicate requests.

\begin{definition}[routine request(X?)]
If $(X?, q, \bot) \in V'_p$ and $q \neq p$ then:
\begin{itemize}
\item Update to $(X?, q, q)$ in $V'_p$ 
\item Add $(request(X?, p), q)$ to $M'_p$
\end{itemize}
\end{definition}

The \texttt{export} helper updates the variable table when term $T$ is sent outside agent $p$. Variables created by $p$ are added to $V_p$ when first exported. Imported variables are typically removed since they're no longer local, except for requested readers which require special handling: a fresh relay pair $(Z, Z?)$ is created with a forwarding goal to maintain the request relationship while allowing the original reader to leave $p$'s scope.

\begin{definition}[routine export$(T)$ returns $T'$]
\begin{itemize}
Set $T' := T$
\item For each variable $Y$ occurring in $T$:
    \begin{itemize}
    \item \textbf{Local:} If $Y$ created by $p$ and $(Y, p, \cdot) \notin V'_p$: add $(Y, p, \bot)$ to $V'_p$
    \item \textbf{Non-local:} If $Y$ created by $q \neq p$ then
        \begin{itemize}
        \item \textbf{Writer or Non-requested Reader:} If $Y \in V$ or $(Y, q, \bot) \in V'_p$ then remove $(Y, q, \cdot)$ from $V'_p$
        \item \textbf{Requested Reader:} If $(Y, q, q) \in V'_p$  then create fresh pair $(Z, Z?)$, replace $Y$ with $Z?$ in $T'$, add $\text{export\_reader}(Y, Z)$ to $A'_p$, add $(Z?, p, \bot)$ to $V'_p$
        \end{itemize}
    \end{itemize}
\end{itemize}
$T'$ is the result of applying variable replacements (if any) to $T$.
\end{definition}

\begin{definition}[routine reactivate(X?) for agent p returns R]
\begin{itemize}
\item Let $R = \{G : (G, W) \in S'_p, X? \in W\}$
\item $S'_p := S'_p \setminus \{(G, W) : G \in R\}$
\item Return $R$
\end{itemize}
\end{definition}

\subsection{Transactions}

Next, we describe the implementation-ready maGLP transactions one by one:

\mypara{Abandoned variables}
During goal reduction, variables may become abandoned when their paired counterparts disappear from the computation without being instantiated. This happens when a variable that occurs in the reduced atom is neither instantiated by the reduction nor occurring in the resulting body. The implementation should detect such abandonment to prevent indefinite suspension or shared-variable entries for variables that can never receive values. Abandoned variables allow garbage-collection in shared variable tables and cause dependent suspended goals to fail rather than wait indefinitely.

\begin{definition}[Variable Abandonment in Reduction]
When reducing atom $A$ with clause $C$ yielding body $B$ and substitution $\hat\sigma$, a variable $Y$ is \emph{abandoned} if its paired variable $Y'$ satisfies all three conditions: $Y'$ occurs in $A$, $Y'$ is not instantiated by $\hat\sigma$ or $\hat\sigma?$  , and $Y'$ does not occur in $B$.
\end{definition}

\begin{definition}[Implementation-Ready Reduce Transaction]\label{definition:IRmaGLP-reduce}
The unary Reduce transaction for agent $p$ transitions $(R_p, V_p, M_p) \rightarrow (R'_p, V'_p, M'_p)$ where $R_p = (A_p, S_p, F_p)$,  $(R'_p, V'_p, M'_p): = (R_p, V_p, M_p)$ with $A_p = A \cdot A_r$ for head goal $A$:

\begin{enumerate}
\item \textbf{Reduce:} If GLP reduction of $A$ with first applicable clause $C \in M$ succeeds with $(B,\hat\sigma)$:
\begin{itemize}
    \item Let $R = \bigcup_{X? \in V_{\hat\sigma?}} \text{reactivate}(X?)$ (modifies $S'_p$)
    \item $A'_p := (A_r \cdot B \cdot R)\hat\sigma\hat\sigma?$
    \item Update $V'_p$: for each $X? \in W$ where $(X?, q, \bot) \in V'_p$, update to $(X?, q, q)$
    \item Update $M'_p$: add $(X?:=T, r)$ for each $\{X?:=T\} \in \hat\sigma?$ where $(X?, p, r) \in V'_p, r \neq \bot$
    \item Call abandon$(Y)$ for each abandoned variable $Y$
\end{itemize}

\item \textbf{Suspend:} Else if $W = \bigcup_{C \in M} W_{C} \neq \emptyset$:
\begin{itemize}
    \item $A'_p := A_r$
    \item $S'_p := S'_p \cup \{(A, W)\}$
    \item Call request$(X?)$ for each $X? \in W$ (modifies $V'_p$ and $M'_p$)
\end{itemize}

\item \textbf{Fail:} Else:
\begin{itemize}
    \item $A'_p := A_r$
    \item $F'_p := F'_p \cup \{A\}$
    \item Call abandon$(Y)$ for each variable $Y$ in $A$ (modifies $V'_p$ and $M'_p$)
\end{itemize}
\end{enumerate}
Then $R'_p := (A'_p, S'_p, F'_p)$.
\end{definition}

\begin{definition}[Implementation-Ready Communicate Transaction]\label{definition:IRmaGLP-communicate}
The binary Communicate transaction $(c_p,c_q) \rightarrow (c'_p,c'_q)$ where $p \neq q$ and $(m, q) \in M_p$. Set $(c'_p, c'_q) := (c_p, c_q)$, remove $(m, q)$ from $M'_p$, and case:
\begin{enumerate}
\item \textbf{Assignment} $m = (X?:=T)$ where $X?$ is local to $q$: 
\begin{itemize}
    \item Let $R = $ reactivate$(X?)$ for agent $q$ (modifies $S'_q$)
    \item If $T \neq \bot$: $A'_q := (A_q \cdot R)\{X?:=T\}$, and apply $\{X?:=T\}$ to $S'_q$ and $F_q$
    \item Else: $A'_q := A_q \cdot R$
    \item Remove $(X?, \cdot, \cdot)$ from $V'_q$
    \item For each variable $Y$ in $T$ not already local to $q$ and created by $r$: add $(Y, r, \bot)$ to $V'_q$
\end{itemize}
 
\item \textbf{Read Request} $m = \text{request}(X?, p)$:
\begin{itemize}
    \item If $p = \bot$ then call abandon$(X?)$ for agent $q$ (modifies $V'_q$ and $M'_q$)
    \item Else if $(X?, q, \bot) \in V'_q$ then update to $(X?, q, p)$ in $V'_q$
    \item Else if $(X, q, T) \in V'_q$ then add $(X?:=T, p)$ to $M'_q$
\end{itemize}
\end{enumerate}
\end{definition}

\begin{definition}[Implementation-Ready Network Transaction]\label{definition:IRmaGLP-network}
The binary Network transaction $(c_p,c_q) \rightarrow (c'_p,c'_q)$ where $p \neq q$ and a new \verb|msg|$(q,X)$ appears in $p$'s network output stream. Set $(c'_p, c'_q) := (c_p, c_q)$:
\begin{itemize}
\item Let $X' := \text{export}(X)$ for agent $p$ (modifies $V'_p$ and $M'_p$)
\item Add $X'$ to $q$'s network input stream
\item For each variable $Y$ in $X'$ not already local to $q$ and created by $r$: add $(Y, r, \bot)$ to $V'_q$
\end{itemize}
\end{definition}

The scheduler operates deterministically by selecting the head of the active queue $A_p$. When any reader $X? \in W$ for a suspended goal $(A, W) \in S_p$ receives a value or is marked abandoned, the goal $A$ is moved from $S_p$ to $A_p$ for re-evaluation. Goals in $F_p$ remain terminal, preserving logical completeness while enabling runtime fault analysis.

\subsection{Extensions for Secure Multiagent GLP}

To extend the implementation-ready transition system to Secure maGLP, the following cryptographic mechanisms augment the definitions without modifying their structure:

\subsubsection{Agent Identity and Cryptography}

Each agent $p \in \Pi$ is augmented with:
\begin{itemize}
\item A self-chosen keypair $(pk_p, sk_p)$ where the public key $pk_p$ serves as the agent's identity
\item The agent identifier $p$ is synonymous with $pk_p$ throughout the system
\item We assume knowledge of  other agents' public keys through social contacts
\end{itemize}

\subsubsection{Message Authentication and Encryption}

All messages in $M_p$ are cryptographically protected. A message $(m, q) \in M_p$ becomes $(m_{M,p,q}, q)$ where the subscript notation indicates:
\begin{itemize}
\item $M$: Attestation by the GLP runtime proving $m$ resulted from correct execution of module $M$
\item $p$: Digital signature using agent $p$'s private key $sk_p$
\item $q$: Encryption using agent $q$'s public key $pk_q$
\end{itemize}

\subsubsection{Transaction Augmentations}

\paragraph{Reduce Transaction}
When generating messages $(X?:=T, r)$ for remote readers, the implementation creates $(X?:=T)_{M,p,r}$ with attestation proving the assignment resulted from correct goal/clause reduction using module $M$.

\paragraph{Communicate Transaction}
Before processing any received message $(m_{M,p,q}, q)$:
\begin{enumerate}
\item Decrypt using $q$'s private key $sk_q$
\item Verify signature using $p$'s public key $pk_p$
\item Validate attestation for module $M$
\item Discard the message if any verification fails
\item Process according to Definition~\ref{definition:IRmaGLP-communicate} only if all verifications succeed
\end{enumerate}

\paragraph{Network Transaction}
Network messages \verb|msg|$(q,X)$ are similarly protected as (\verb|msg|$(q,X))_{M,p,q}$ ensuring authenticated channel establishment.

\subsubsection{Module Verification}

\begin{itemize}
\item Each agent executes a verified GLP module $M$ with a cryptographic hash identifier
\item Attestations include the module hash, enabling recipients to verify code compatibility
\item Guard predicates \verb|attestation(X, att(Agent, Module))| and \verb|module(M)| provide program-level access to verification results
\end{itemize}

\subsubsection{Security Properties Achieved}

These extensions ensure:
\begin{itemize}
\item \textbf{Integrity}: Messages cannot be modified without detection
\item \textbf{Confidentiality}: Only intended recipients can decrypt messages
\item \textbf{Non-repudiation}: Senders cannot deny authenticated messages
\item \textbf{Authentication}: All inter-agent communication is mutually authenticated
\end{itemize}

The implementation-ready transition system with these cryptographic extensions realizes Secure maGLP while maintaining the same operational behaviour for correctly authenticated participants. Byzantine agents who fail verification are effectively excluded from the computation through message rejection.

\end{document}


\end{document}
