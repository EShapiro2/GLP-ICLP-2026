\documentclass{tlp}

% Packages not already loaded by tlp.cls
% (tlp.cls loads: graphicx, color, amsmath, hyperref, enumitem, natbib)
\usepackage{amssymb,amsfonts}
\usepackage{xspace}
\usepackage{cleveref}
\usepackage{xcolor} % extends color already loaded by tlp

% Algorithm packages
\usepackage{algorithm}
\usepackage[noend]{algpseudocode}

% Additional packages
\usepackage{multicol}
\usepackage{multirow}
\usepackage{mathtools}
\usepackage{relsize}
\usepackage{bm}
\usepackage{verbatimbox}
\usepackage{wrapfig}
\usepackage{fancyvrb}
%\RecustomVerbatimEnvironment{verbatim}{Verbatim}{fontsize=\footnotesize}

% Theorem-like environments (not provided by tlp class)
% Undefine tlp.cls proof environment before loading amsthm
\let\proof\relax
\let\endproof\relax
\usepackage{amsthm}
\theoremstyle{definition}
\newtheorem{definition}{Definition}[section]
\newtheorem{example}[definition]{Example}
\theoremstyle{plain}
\newtheorem{theorem}[definition]{Theorem}
\newtheorem{proposition}[definition]{Proposition}
\newtheorem{lemma}[definition]{Lemma}
\newtheorem{corollary}[definition]{Corollary}
\theoremstyle{remark}
\newtheorem{remark}[definition]{Remark}

% Restatable theorems (must be after amsthm)
\usepackage{thmtools,thm-restate}

% Custom formatting commands
\newcommand{\mypara}[1]{\smallskip\noindent\textbf{#1.}}
%\newcommand{\temph}[1]{\textbf{#1}}
\newcommand{\temph}[1]{\emph{#1}}

\newcommand{\remove}[1]{}
\newcommand{\udi}[1]{\textcolor{blue}{[Udi says: #1]}}
\newcommand{\claude}[1]{\textcolor{red}{[Claude: #1]}}

% Abbreviation commands
\newcommand{\GLP}{\textsc{GLP}\xspace}
\newcommand{\lp}{logic programs\xspace}
\newcommand{\cp}{Concurrent Prolog\xspace}
\newcommand{\scl}{\textsc{scl}\xspace}
\newcommand{\gsn}{\textsc{gsn}\xspace}

% Math commands
\newcommand{\calV}{\mathcal{V}}
\newcommand{\calG}{\mathcal{G}}
\newcommand{\calT}{\mathcal{T}}
\newcommand{\calF}{\mathcal{F}}
\newcommand{\calR}{\mathbb{R}}
\newcommand{\calN}{\mathbb{N}}
\newcommand{\calA}{\mathcal{A}}
\newcommand{\V}{\mathcal{V}}
\newcommand{\calS}{\mathcal{S}}
\newcommand{\calL}{\mathcal{L}}
\newcommand{\calP}{\mathcal{P}}
\newcommand{\calM}{\mathcal{M}}
\newcommand{\calE}{\mathcal{E}}
\newcommand{\calB}{\mathcal{B}}
\newcommand{\calC}{\mathcal{C}}
\newcommand{\calX}{\mathcal{X}}
\newcommand{\calD}{\mathcal{D}}

% Roman numeral abbreviations
\newcommand{\ia}{\textit{i}}
\newcommand{\ib}{\textit{ii}}
\newcommand{\ic}{\textit{iii}}
\newcommand{\id}{\textit{iv}}
\newcommand{\iie}{\textit{v}}
\newcommand{\iif}{\textit{vi}}
\newcommand{\iiv}{\textit{iv}}
\newcommand{\iv}{\textit{v}}

% Program counter for examples
\newcounter{pc}
\newcommand\spc{\addtocounter{pc}{1}\thepc}
\newcommand{\Program}[1]{\medskip\noindent\textbf{Program \spc: #1}\vspace{-5pt}}

% Set list spacing
\setlist{nosep, leftmargin=*}
\setlist{itemsep=1pt, topsep=3pt, leftmargin=*}

\newtheorem{observation}{Observation}

\begin{document}

\lefttitle{E. Shapiro}
\jnlPage{1}{14}
\jnlDoiYr{2026}
\doival{10.1017/xxxxx}

\title{GLP: A Grassroots, Multiagent, Concurrent,\\ Logic Programming Language}

\begin{authgrp}
\author{\gn{Ehud} \sn{Shapiro}}
\affiliation{London School of Economics and Weizmann Institute of Science}
\end{authgrp}

\maketitle

\begin{abstract}
Grassroots platforms are distributed systems with multiple instances that can (1) operate independently of each other and of any global resource other than the network, and (2) coalesce into ever larger instances, possibly resulting in a single global instance.

Here, we present Grassroots Logic Programs (GLP), a multiagent concurrent logic programming language designed for the implementation of grassroots platforms.  
We introduce the language  incrementally: We recall the standard operational semantics of logic programs; introduce the operational semantics of Concurrent (single-agent) GLP as a restriction of that of LP; recall the notion of multiagent transition systems and atomic transactions; introduce the operational semantics of multiagent GLP via a multiagent transition system specified via atomic transactions; and prove multiagent GLP to be grassroots. The accompanying programming example is the grassroots social graph---the fundamental grassroots platform on which all others are based. 

With the mathematical foundations presented here: a workstation-based implementation of Concurrent GLP was developed by AI, based on the operational semantics of Concurrent GLP;  a distributed peer-to-peer smartphone-based implementation of multiagent GLP is being developed by AI, based on the operational semantics of multiagent GLP; a moded type system for GLP was implemented by AI, to facilitate the specification of GLP programs by human and AI designers, for their programming by AI; all reported in detail in companion papers.
\end{abstract}

\begin{keywords}
concurrent logic programming, grassroots platforms, operational semantics, multiagent transition systems
\end{keywords}
% Main sections
\section{Introduction}

A digital platform is \emph{grassroots}~\citep{shapiro2023grassrootsBA} if it can have multiple instances that can (1) operate independently of each other and of any global resource other than the network, and (2) coalesce into ever larger instances, possibly resulting in a single global instance.
Grassroots platforms that have been specified (but not yet implemented) include the grassroots social graph~\citep{shapiro2025atomic}, grassroots social networks~\citep{shapiro2023gsn}, grassroots cryptocurrencies~\citep{shapiro2024gc,lewis2023grassroots}, and grassroots federation~\citep{shapiro2025GF}. The Scuttlebutt protocol and social network~\citep{kermarrec2020gossiping} is perhaps the sole example of a deployed grassroots platform.

Here, we present Grassroots Logic Programs (GLP), a multiagent, concurrent, logic programming language, designed for the implementation of smartphone-based, serverless, grassroots platforms.  First we present Concurrent GLP as a variation on Logic Programs (LP)~\citep{lloyd1987foundations}.  Syntactically, GLP adds to LP (1) \textbf{Readers:} Each logic variable (now referred to as \emph{writer}) is paired with a \emph{reader}~\citep{shapiro1983subset}, which is assigned a value only once its paired writer is assigned that value;  (2) \textbf{Single-Occurrence (SO):} A variable may occur at most once in a goal or a clause; and (3) \textbf{Single-Reader Single-Writer (SRSW):} A writer occurs in a clause iff its paired reader occurs in the clause.  

The operational semantics of GLP is a restriction of that of LP, which preserves the concept of computation-as-deduction~\citep{kowalski1974predicate}, yet conjures both linear logic~\citep{girard1987linear} and futures/promises~\citep{friedman1976impact}: An assignment to a variable may be produced at most once, via the sole occurrence of a writer (promise), and consumed at most once, via the sole occurrence of its paired reader (future). We illustrate Concurrent GLP via a simulation of a grassroots social graph, in which agents may initiate friendships (bidirectional communication channels) via ``cold-calls'' as well as introduce mutual friends to each other, via a simulated network switch.

Then, we present multiagent transition systems and their specification via multiagent atomic transactions, and use them to define multiagent GLP (maGLP), where the key challenge is realizing ``cold-calls'':  How can two agents in disconnected components of the social graph become the owners of paired logic variables?  maGLP extends GLP with (1) \textbf{Multiagents:}  Named agents that operate independently while communicating to remote readers assignments made to paired local writers;  (2) \textbf{Cold-calls:} A means for sending a term with variables to a named agent while retaining their paired variables locally.  We then present an maGLP implementation of the social graph in which the network-simulator is replaced by maGLP's Cold-call transaction, and social cold-calls among agents are realised by maGLP cold-calls among these agents.

Lastly, we recall the definition of grassroots platforms~\citep{shapiro2023grassrootsBA} and how to prove that a platform specified via atomic transactions is grassroots~\citep{shapiro2025atomic}, and apply these to prove that maGLP is indeed grassroots.

The mathematical foundations presented here have enabled three companion efforts, all developed by AI: a workstation-based implementation of Concurrent GLP, based on its operational semantics; a distributed peer-to-peer smartphone-based implementation of multiagent GLP, based on its operational semantics~\citep{shapiro2026implementing}; and a moded type system for GLP~\citep{shapiro2026types}, facilitating the specification of GLP programs by human and AI designers for their programming by AI.
Examples GLP programs in this paper are typed.

\mypara{Paper outline}
Section~\ref{sec:glp} presents Concurrent GLP: transition systems, GLP extending LP with readers, operational semantics, guards, programming examples, and the grassroots social graph simulation. Section~\ref{sec:maglp} extends to multiagent GLP: multiagent transition systems, the maGLP definition, and the multiagent social graph. Section~\ref{sec:grassroots} proves maGLP is grassroots. Section~\ref{section:related-work} reviews related work, and Section~\ref{section:conclusion} concludes. All proofs are deferred to the appendices.

%==============================================================================
\section{Concurrent GLP}
\label{sec:glp}
%==============================================================================

This section presents Grassroots Logic Programs (GLP), a concurrent logic programming language. We begin with transition systems, recall logic programs (LP) and define their operational semantics via transition systems, and then extend LP to GLP. We illustrate GLP with programming examples and conclude with the grassroots social graph---the foundational platform that all other grassroots platforms build upon.

%------------------------------------------------------------------------------
\subsection{Transition Systems}
\label{sec:ts}
%------------------------------------------------------------------------------

\begin{definition}[Transition System]
\label{def:ts}
A \temph{transition system} is a tuple $TS = (C, c_0, T)$ where $C$ is an arbitrary set of \temph{configurations}, $c_0 \in C$ is a designated \temph{initial configuration}, and $T \subseteq C \times C$ is a \temph{transition relation}, with transitions written $c \rightarrow c' \in T$.
A transition $c \rightarrow c' \in T$ is \temph{enabled} from configuration $c$. A configuration $c$ is \temph{terminal} if no transitions are enabled from $c$. A \temph{computation} is a (finite or infinite) sequence of configurations where for each two consecutive configurations $(c,c')$ in the sequence, $c \rightarrow c' \in T$. A \temph{run} is a computation starting from $c_0$, which is \temph{complete} if it is infinite or ends in a terminal configuration.
\end{definition}

%------------------------------------------------------------------------------
\subsection{Logic Programs}
\label{sec:lp-ts}
%------------------------------------------------------------------------------

We recall standard Logic Programs (LP) notions of syntax, most-general unifier (mgu), and semantics via goal reduction.

\begin{definition}[Logic Programs Syntax]
\label{def:lp-syntax}
We employ standard LP notions. Let $\calV$ denote the set of \temph{variables} (identifiers beginning with uppercase). A \temph{term} is a variable, a constant (numbers, strings, or the empty list \verb|[]|), or a compound term $f(T_1,\ldots,T_n)$ with functor $f$ and subterms $T_i$. Let $\calT$ denote the set of all terms. We use standard list notation: \verb=[X|Xs]= for a list cell, \verb|[X1,...,Xn]| for finite lists. A term is \temph{ground} if it contains no variables.

A \temph{unit goal} is a compound term, also commonly referred to as an \temph{atom}. A \temph{goal} is a multiset of unit goals; the empty goal is written \verb|true|. A \temph{clause} $A$~\verb|:-|~$B$ has head $A$ (a unit goal) and body $B$ (a goal); a \temph{unit clause} has empty body. A \temph{logic program} is a finite set of clauses; clauses for the same predicate form a \temph{procedure}. Let $\calG(P)$ denote the set of goals over the vocabulary of the program $P$.
\end{definition}

A \emph{substitution} $\sigma$ is an idempotent function $\sigma: \calV \to \calT$, a mapping from variables to terms applied to a fixed point. By convention, $\sigma(x)=x\sigma$. Let $\Sigma$ denote the set of all substitutions. We assume standard notions of instance, ground, renaming, renaming apart, unifier, and most-general unifier (mgu).

\begin{definition}[LP Goal/Clause Reduction]
\label{def:lp-reduction}
Given an LP unit goal $A$ and clause $C$, with $H$ \verb|:-| $B$ being the result of renaming $C$ apart from $A$, the \temph{LP reduction} of $A$ with $C$ \temph{succeeds with} $(B,\sigma)$ if $A$ and $H$ have an mgu $\sigma$.
\end{definition}

\begin{definition}[Logic Programs Transition System]
\label{def:lp-ts}
A transition system $LP(P) = (C, c_0, T)$ is a \temph{Logic Programs transition system} for a logic program $P$ and initial goal $G_0 \in \mathcal{G}(P)$, if $C=\mathcal{G}(P)\times \Sigma$, $c_0=(G_0,\emptyset)$, and $T$ is the set of all transitions $(G,\sigma) \rightarrow (G',\sigma')$ such that for some unit goal $A \in G$ and clause $C \in P$ the LP reduction of $A$ with $C$ succeeds with $(B,\hat\sigma)$, $G' = (G \setminus \{A\} \cup B)\hat\sigma$, and $\sigma'=\sigma\circ\hat\sigma$.
\end{definition}

LP has two forms of nondeterminism: the choice of $A \in G$, called \emph{and-nondeterminism}, and the choice of $C \in P$, called \emph{or-nondeterminism}, and as such are closely-related to Alternating Turing Machines~\cite{shapiro1984alternation}.

\begin{definition}[Proper and Successful Run, Outcome]
\label{def:proper-run}
A run $\rho: (G_0,\sigma_0) \rightarrow \cdots \rightarrow (G_n, \sigma_n)$ of $LP(P)$ is \temph{proper} if for any $1\le i< n$, a variable that occurs in $G_{i+1}$ but not in $G_i$ also does not occur in any $G_j$, $j<i$. If proper, the \temph{outcome} of $\rho$ is $(G_0$ \verb|:-| $G_n)\sigma_n$. Such a run is \temph{successful} if $G_n=\emptyset$.
\end{definition}
The following proposition justifies the computation-as-deduction view of LP~\cite{kowalski1974predicate},
calling a proper LP run a \emph{derivation} and a complete proper run ending in the empty goal a \emph{successful derivation}.

\begin{proposition}[LP Computation is Deduction]
\label{prop:lp-deduction}
The outcome $(G_0$ \verb|:-| $G_n)\sigma$ of a proper run of $LP(P)$ is a logical consequence of $P$.
\end{proposition}

\mypara{Denotational Semantics}
The $LP(P)$ transition system allows defining several denotational semantic notions for a program $P$: (1) the \emph{clause semantics} is the set of all outcomes of all proper runs with an initial most-general unit goal (arguments are distinct variables), closely related to the fully-abstract compositional semantics of LP~\cite{gaifman1989fully}; (2) the \emph{atom semantics} is the set of all outcomes of all successful derivations with an initial most-general unit goal; (3) the \emph{ground atom semantics} is the standard model-theoretic semantics, the set of ground instances of the atom semantics over the Herbrand universe of $P$~\cite{lloyd1987foundations}.

%------------------------------------------------------------------------------
\subsection{GLP: Extending LP with Readers}
\label{sec:glp-ext}
%------------------------------------------------------------------------------

Grassroots Logic Programs (GLP) extend LP by (1) adding a paired \emph{reader} $X?$ to every ``ordinary'' logic variable $X$, now called a \emph{writer}; (2) restricting variables in goals and clauses to have at most a single occurrence (SO); and (3) requiring that a variable occurs in a clause iff its paired variable also occurs in it (single-reader single-writer, SRSW). The result eschews unification in favour of simple term matching, is linear-logic-like~\cite{girard1987linear}, and is futures/promises-like~\cite{friedman1976impact}: each assignment $X := T$ is produced at most once via the sole occurrence of a writer (promise) $X$, and consumed at most once via the sole occurrence of its paired reader (future) $X?$.

\begin{definition}[GLP Variables]
\label{def:glp-variables}
Recall that $\calV$ is the set of LP variables, henceforth called \temph{writers}. Define $\calV? = \{X? \mid X \in \calV\}$, called \temph{readers}. The set of all GLP variables is $\hat\calV = \calV \cup \calV?$. A writer $X$ and its reader $X?$ form a \temph{variable pair}.
\end{definition}

GLP terms, unit goals, goals, and clauses are as in LP (Definition~\ref{def:lp-syntax}) but defined over the variables in $\hat\calV$.

\begin{definition}[Single-Occurrence (SO) Invariant]
\label{def:so-invariant}
A term, goal, or clause satisfies the \temph{single-occurrence (SO) invariant} if every variable occurs in it at most once.
\end{definition}

\begin{definition}[GLP Program]
\label{def:glp-program}
A clause $C$ satisfies the \temph{single-reader/single-writer (SRSW) restriction} if it satisfies SO and a variable occurs in $C$ iff its paired variable also occurs in $C$.
A \temph{GLP program} is a finite set of clauses satisfying SRSW; clauses for the same predicate form a \temph{procedure}.
Let $\hat\calG(P)$ denote the set of goals over $\hat\calV$ and the vocabulary of $P$.
\end{definition}

\begin{example}[Fair Merge]
\label{ex:merge}
Consider the quintessential concurrent logic program for fairly merging two streams, written in GLP:
\begin{verbatim}
procedure merge(Stream?, Stream?, Stream).
merge([X|Xs], Ys, [X?|Zs?]) :- merge(Ys?, Xs?, Zs).
merge(Xs, [Y|Ys], [Y?|Zs?]) :- merge(Xs?, Ys?, Zs).
merge(Xs, [], Xs?).
merge([], Ys, Ys?).
\end{verbatim}
and the goal \verb=merge([1,2,3|Xs?],[a,b|Ys?],Zs)=. Both the goal and each clause satisfy SO, and each clause satisfies SRSW. The first two clauses swap inputs in recursive calls, ensuring fairness when both streams are available.
\end{example}

\begin{remark}[Typed Programs]
\label{rem:typed-programs}
GLP programs in this paper include \verb|procedure| declarations that specify the moded types of each predicate's arguments. These declarations are part of the moded type system for GLP presented in the companion paper~\cite{shapiro2026types}. The type system itself is not introduced here; the \verb|procedure| declarations and type definitions (e.g., \verb|Stream ::= [] ; [_|Stream].|) should be self-explanatory.
\end{remark}

As we shall see (Proposition~\ref{prop:so-preservation}), the SO invariant is maintained by the SRSW restriction: reducing a goal satisfying SO with a clause satisfying SRSW results in a goal satisfying SO. The purpose of the SRSW restriction is to prevent multiple writer occurrences racing to bind a variable.

%------------------------------------------------------------------------------
\subsection{GLP Operational Semantics}
\label{sec:glp-operational}
%------------------------------------------------------------------------------

\begin{definition}[Writers Substitution, Assignment, Readers Substitution and Counterpart]
\label{def:writers-assignment}
A GLP \temph{writer assignment} is a term of the form $X := T$, $X\in\calV$, $T\notin\calV$, satisfying SO. Similarly, a GLP \temph{reader assignment} is a term of the form $X? := T$, $X?\in\calV?$, $T\notin\calV$, satisfying SO. A \temph{writers (readers) substitution} $\sigma$ is the substitution implied by a set of writer (reader) assignments that jointly satisfy SO. Given a writers assignment $X := T$, its \temph{readers counterpart} is $X? := T$, and given a writers substitution $\sigma$, its \temph{readers counterpart} $\sigma?$ is the readers substitution defined by $X?\sigma? = X\sigma$.
\end{definition}

\begin{definition}[GLP Renaming, Renaming Apart]
\label{def:glp-renaming}
A \temph{GLP renaming} is a substitution $\rho: \hat\calV \to \hat\calV$ such that for each $X \in \calV$: $X\rho \in \calV$ and $X?\rho = (X\rho)?$. Two GLP terms \temph{have a variable in common} if for some writer $X \in \calV$, either $X$ or $X?$ occurs in both. A GLP renaming $\sigma$ \temph{renames $T'$ apart from} $T$ if $T'\sigma$ and $T$ have no variable in common.
\end{definition}

\begin{definition}[Writer MGU]
\label{def:writer-mgu}
Given two GLP unit goals $A$ and $H$, a \temph{writer mgu} is a writers substitution $\sigma$ such that $A\sigma = H\sigma$ and $\sigma$ is most general among such substitutions. Unlike standard unification, writer mgu only binds writers, not readers.
\end{definition}

\begin{definition}[GLP Goal/Clause Reduction]
\label{def:glp-reduction}
Given GLP unit goal $A$ and clause $C$, with $H$ \verb|:-| $B$ being the result of the GLP renaming of $C$ apart from $A$, the \temph{GLP reduction} of $A$ with $C$ \temph{succeeds with result} $(B,\sigma)$ if $A$ and $H$ have a writer mgu.
\end{definition}

\begin{definition}[GLP Transition System]
\label{def:glp-ts}
Given a GLP program $P$, an \temph{asynchronous resolvent} over $P$ is a pair $(G, \sigma)$ where $G \in \hat\calG(P)$ and $\sigma$ is a readers substitution.

A transition system $GLP(P) = (\calC, c_0, \calT)$ is a \temph{GLP transition system} over $P$ and initial goal $G_0$ satisfying SO if:
\begin{enumerate}
    \item $\calC$ is the set of all asynchronous resolvents over $P$
    \item $c_0 = (G_0, \emptyset)$
    \item $\calT$ is the set of all transitions $(G, \sigma) \rightarrow (G', \sigma')$ satisfying either:
    \begin{enumerate}
        \item \textbf{Reduce:} there exists unit goal $A \in G$ such that $C \in P$ is the first clause for which the GLP reduction of $A$ with $C$ succeeds with result $(B, \hat\sigma)$, $G' = (G \setminus \{A\} \cup B)\hat\sigma$, and $\sigma' = \sigma \circ \hat\sigma?$
        \item \textbf{Communicate:} $\{X? := T\} \in \sigma$, $X?\in G$, $G' = G\{X? := T\}$, and $\sigma' = \sigma$
    \end{enumerate}
\end{enumerate}
\end{definition}

GLP Reduce differs from LP in (1) the use of a writer mgu instead of a regular mgu and (2) the choice of the first applicable clause instead of any clause. The first is the fundamental use of GLP readers for synchronization. The second compromises on the or-nondeterminism of LP to allow writing fair concurrent programs, such as fair merge above. Note that or-nondeterminism is not completely eliminated, as different scheduling of arrival of bindings on the two input streams of \verb|merge| may result in different orders in its output stream.

The GLP Communicate rule realizes the use of reader/writer pairs for asynchronous communication: it communicates an assignment from its writer to its paired reader.


\mypara{Monotonicity}
Key differences between LP and GLP relate to monotonicity. In LP, if a goal cannot be reduced, it will never be reduced. In GLP, a goal that cannot be reduced now may be reduced in the future: if $A$ and $H$ have an mgu that writes on a reader $X? \in A$, and therefore have no writer mgu at present,  another goal that has $X$ may reduce, assigning $X$, and later $X?$, to a value that will allow $A$ and $H$ to have a writer mgu. Conversely, in LP, if a goal $A$ can be reduced now with some clause $H$\verb|:-|$B$, with a regular mgu of $A$ and $H$, it may not be reducible in the future due to variables that $A$ shares with other goals being assigned values by reductions of other goals, preventing unification between the instantiated $A$ and $H$. In GLP, if a goal $A$ can be reduced now (with a writers mgu), it can always be reduced in the future, as the SO invariant ensures that no other goal can assign any writer in $A$.

Implementation-wise, if a GLP goal $A$ cannot be reduced now, but there is a readers substitution $\sigma$ such that $A\sigma$ can be reduced, such readers are identified, the goal $A$ \emph{suspends} on these readers, and is rescheduled for another reduction attempt once any of them is assigned.

Despite these differences, GLP adopts the same notions of successful run and outcome of LP (Definition~\ref{def:proper-run}), and has the same notion of logical consequence as LP. Let $/?$ be an operator that replaces every reader by its paired writer.

\begin{proposition}[GLP Computation is Deduction]
\label{prop:glp-deduction}
Let $(G_0$ \verb|:-| $G_n)\sigma$ be the outcome of a proper GLP run $\rho: (G_0,\sigma_0) \rightarrow \cdots \rightarrow (G_n, \sigma_n)$ of $GLP(P)$. Then $(G_0$ \verb|:-| $G_n)\sigma/?$ is a logical consequence of $P/?$.
\end{proposition}

We note two additional safety properties of GLP runs.

\begin{proposition}[SO Preservation]
\label{prop:so-preservation}
If the initial goal $G_0$ satisfies SO, then every goal in the GLP run satisfies SO.
\end{proposition}

\begin{proposition}[Monotonicity]
\label{prop:glp-monotonicity}
In any GLP run, if unit goal $A$ can reduce with clause $C$ at step $i$, then either an instance of $A$ has been reduced by step $j > i$, or an instance of $A$ can still reduce with $C$ at step $j$.
\end{proposition}

\begin{definition}[Persistent Transition System]
\label{def:persistent};//;/;
A transition system is \temph{persistent} if every enabled transition remains enabled until taken.
\end{definition}

\begin{lemma}[GLP Persistence]
\label{lem:persistence}
GLP is persistent. For Reduce: if a Reduce transition is enabled for goal $A$ in configuration $(G, \sigma_r)$, it remains enabled in any configuration $(G, \sigma'_r)$ with $\sigma_r \subseteq \sigma'_r$, since other Reduce transitions bind disjoint writers (by SO) and Communicate transitions only instantiate readers. For Communicate: applying $\{X? := T\} \in \sigma$ to $X? \in G$ remains enabled since Reduce transitions do not remove assignments from $\sigma$ and other Communicate transitions apply different assignments.
\end{lemma}

\begin{definition}[Fair Run]
\label{def:fair-run}
A complete GLP run is \temph{fair} if every transition that becomes enabled is eventually taken.
\end{definition}

Proofs appear in Appendix~\ref{appendix:proofs}.

%------------------------------------------------------------------------------
\subsection{Term Matching Eschews Unification}
\label{sec:term-matching}
%------------------------------------------------------------------------------

If two terms $T_1$ and $T_2$ that jointly satisfy SO are unifiable with an mgu $\sigma$, then $\sigma$ maps any variable in $T_1$ to a subterm of $T_2$ and vice versa. Hence, the SO invariant of GLP allows eschewing unification in favour of \emph{term matching} that performs joint term-tree traversal and collects variable assignments along the way.

\begin{definition}[Term Matching]
\label{def:term-matching}
Given two terms $T_1$ and $T_2$ that jointly satisfy SO, their \temph{term matching} proceeds via the joint traversal of the term-trees of $T_1$ and $T_2$, consulting the following table at each pair of joint vertices, where $X_1, X_2$ denote writers, $X_1?, X_2?$ denote readers, and $f/n$ denotes a non-variable term, a constant when $n=0$ and a compound term when $n>0$:
\begin{center}
\begin{tabular}{l|lll}
$T_1 \backslash T_2$ & Writer $X_2$ & Reader $X_2?$ & Term $f_2/n_2$ \\
\hline
Writer $X_1$ & fail & $X_1 := X_2?$ & $X_1 := T_2$ \\
Reader $X_1?$ & $X_2 := X_1?$ & fail & suspend on $X_1?$\\
Term $f_1/n_1$ & $X_2 := T_1$ & fail & fail if $f_1 \ne f_2$ or $n_1 \ne n_2$\\
\end{tabular}
\end{center}
The writer mgu is the union of all writer assignments if no \emph{fail} was encountered and the suspension set is empty.
\end{definition}

\begin{remark}
In an actual implementation, assuming $T_1$ is a goal term and $T_2$ a head term, the case of $X_1?$ and $T_2$ would add $X_1?$ to the set of readers the goal would suspend upon.
\end{remark}

%------------------------------------------------------------------------------
\subsection{Guards}
\label{sec:guards}
%------------------------------------------------------------------------------

GLP clauses may include \emph{guards}---tests that determine clause applicability.

\begin{definition}[Guarded Clause]
\label{def:guarded-clause}
A \temph{guarded clause} has the form $H$ \verb|:-| $G$ \verb"|" $B$, where $H$ is the head, $G$ is a conjunction of guard predicates, and $B$ is the body. The guard separator ``\verb"|"'' distinguishes guards from the body, and is interpreted logically as a conjunction.  Guard arguments are readers paired to head writers.
\end{definition}

Guards have three-valued semantics. Each guard predicate explicitly defines its \emph{success} condition. A guard \emph{suspends} if it does not succeed but some instance of it under a readers substitution would succeed. A guard \emph{fails} if no such instance exists. A guard conjunction succeeds if all members succeed; it suspends if any member suspends and none fail; it fails if any member fails.

Definition~\ref{def:glp-reduction} of a GLP goal/clause reduction is augmented to succeed if the guard also succeeds.

\begin{remark}[Guards and SRSW]
\label{rem:guards-srsw}
Guard occurrences count toward SRSW satisfaction: if $X?$ occurs in a guard, its paired writer $X$ must occur in the head and $X?$ may additionally occur once in the body.

Furthermore, if the success of a guard implies that $X?$ is ground, then $X?$ as well as $X$ may occur multiple times in the clause. Groundness-implying guards include \verb|ground|, \verb|integer|, \verb|number|, \verb|string|, \verb|constant|, arithmetic comparisons (\verb|<|, \verb|>|, \verb|=<|, \verb|>=|, \verb|=:=|, \verb|=\=|), and ground equality (\verb|=?=|). However,  \verb|known| and \verb|compound| do not imply groundness.
\end{remark}

\begin{remark}[Anonymous Variables]
\label{rem:anonymous-variables}
An \emph{anonymous variable} is any variable whose name begins with \verb|_| (e.g., \verb|_|, \verb|_In?|, \verb|_Out|). Anonymous writers may appear in the head, denoting a fresh writer with no paired reader, so that a value assigned to it is discarded. This provides a controlled exception to the SRSW restriction, allowing a process to abandon an input (e.g.\ an input stream) they are no longer interested in.
\end{remark}

Guard predicates include type tests (\verb|integer|, \verb|number|, \verb|string|, \verb|constant|, \verb|compound|, \verb|ground|, \verb|known|), arithmetic comparisons (\verb|<|, \verb|>|, \verb|=<|, \verb|>=|, \verb|=:=|, \verb|=\=|), and ground equality (\verb|=?=|).

%------------------------------------------------------------------------------
\subsection{Programming Examples}
\label{sec:examples}
%------------------------------------------------------------------------------

\begin{example}[Stream Distribution]
\label{ex:distribute}
Broadcasting to multiple consumers uses the \verb|ground| guard to enable safe replication:
\begin{verbatim}
procedure distribute(Stream?, Stream, Stream).
distribute([X|Xs], [X?|Ys1?], [X?|Ys2?]) :- ground(X?) |
    distribute(Xs?, Ys1, Ys2).
distribute([], [], []).
\end{verbatim}
Since \verb|X?| is ground, its multiple occurrences in the head do not violate SRSW.
\end{example}

\begin{example}[Lookup in Association List]
\label{ex:lookup}
\begin{verbatim}
procedure lookup(_?, _?, _).
lookup(Key, [(K, Value)|_], Value?) :- Key? =?= K? | true.
lookup(Key, [_|Rest], Value?) :- otherwise | lookup(Key?, Rest?, Value).
\end{verbatim}
The \verb|=?=| guard tests ground equality; \verb|otherwise| succeeds when no prior clause applies.
\end{example}

%------------------------------------------------------------------------------
\subsection{The Grassroots Social Graph}
\label{sec:social-graph}
%------------------------------------------------------------------------------

The grassroots social graph is the foundational platform upon which all other grassroots platforms are built. Nodes represent cryptographically-identified agents; edges represent authenticated bidirectional channels; connected components arise spontaneously through befriending.

We present the social graph as a single-agent GLP program, using a \emph{network switch} to simulate communication between agents. This demonstrates the program structure before introducing multiagent GLP in Section~\ref{sec:maglp}. We illustrate with a comprehensive scenario involving three agents---Alice, Bob, and Charlie---that demonstrates cold-call befriending, text messaging, and friend-mediated introduction. The complete program appears in Appendix~\ref{appendix:social-graph-complete}.

\subsubsection{Types and Channels}

The social graph uses the following type definitions:
\begin{verbatim}
Response ::= accept(Channel) ; no.
OutputEntry ::= output(String, Stream?).
OutputsList ::= [] ; [OutputEntry|OutputsList].
NetMsg ::= msg(Constant, _).
NetStream ::= [] ; [NetMsg|NetStream].
AgentId ::= Constant.
Decision ::= yes ; no.
\end{verbatim}

A \verb|Channel| is a pair of streams for bidirectional communication. The \verb|new_channel| guard creates complementary channel endpoints:
\begin{verbatim}
new_channel(ch(Xs?, Ys), ch(Ys?, Xs)).
\end{verbatim}
Both endpoints read from their first stream and write to their second:
The first  reads from \verb|Xs?| and writes to \verb|Ys|; the second reads from \verb|Ys?| and writes to \verb|Xs|.

\subsubsection{Agent Structure}

Each agent processes messages from two separate input streams---user and network---and maintains an outputs list mapping named destinations to output streams:
\begin{verbatim}
procedure agent(AgentId?, Stream?, Stream?, OutputsList?).
\end{verbatim}
The four arguments are: agent identity, user input stream, network input stream, and an outputs list. The outputs list initially contains two entries---\verb|output('_user', ...)|  and \verb|output('_net', ...)|---providing streams to the user interface and the network, respectively. As the agent befriends others, new entries are added.

\subsubsection{Cold-Call Befriending Protocol}

The cold-call protocol enables agents to establish friendship without prior shared variables. Figure~\ref{fig:cold-call} shows the protocol clauses.

\begin{figure*}[t]
\begin{verbatim}
%% User initiates cold call
agent(Id, [msg('_user', Id1, connect(Target))|UserIn], NetIn, Outs) :-
    Id? =?= Id1?, ground(Target?) |
    lookup_send('_net', msg(Target?, intro(Id?, Resp)), Outs?, Outs1),
    inject_msg(Resp?, Target?, Id?, UserIn?, UserIn1),
    agent(Id?, UserIn1?, NetIn?, Outs1?).

%% Received cold-call introduction from net (2-arg msg)
agent(Id, UserIn, [msg(Id1, intro(From, Resp))|NetIn], Outs) :-
    Id? =?= Id1? |
    lookup_send('_user', msg(agent, '_user', befriend(From?, Resp?)), Outs?, Outs1),
    agent(Id?, UserIn?, NetIn?, Outs1?).

%% User decision on cold-call
agent(Id, [msg('_user', Id1, decision(Dec, From, Resp?))|UserIn], NetIn, Outs) :-
    Id? =?= Id1? |
    bind_response(Dec?, From?, Resp, Outs?, Outs1, NetIn?, NetIn1),
    agent(Id?, UserIn?, NetIn1?, Outs1?).

%% Response to sent cold-call (injected into UserIn by inject_msg)
agent(Id, [msg(From, Id1, response(Resp))|UserIn], NetIn, Outs) :-
    Id? =?= Id1? |
    handle_response(Resp?, From?, Outs?, Outs1, NetIn?, NetIn1),
    agent(Id?, UserIn?, NetIn1?, Outs1?).
\end{verbatim}
\caption{Cold-call befriending protocol}
\label{fig:cold-call}
\end{figure*}

The protocol works as follows: (1) Alice sends \verb|connect(bob)| to her agent via the \verb|'_user'| stream; (2) her agent sends a 2-argument \verb|msg(bob, intro(alice, Resp))| via the \verb|'_net'| output, including a fresh response variable \verb|Resp|; (3) Bob receives \verb|msg(bob, intro(alice, Resp))| on his network input and forwards the request to his user interface; (4) Bob decides \verb|yes| or \verb|no|; (5) Bob's agent creates a channel pair and sends \verb|accept(Ch)| back via \verb|Resp|; (6) Alice's agent receives the response and both agents add each other to their outputs lists.

\subsubsection{Channel Establishment}

When a cold-call is accepted, both agents establish symmetric channels (Figure~\ref{fig:channel-establish}).

\begin{figure*}[t]
\begin{verbatim}
procedure bind_response(Decision?, AgentId?, Response,
    OutputsList?, OutputsList, Stream?, Stream).
bind_response(yes, From, accept(RetCh?), Outs, Outs1?, In, In1?) :-
    new_channel(RetCh, LocalCh) |
    handle_response(accept(LocalCh?), From?, Outs?, Outs1, In?, In1).
bind_response(no, _, no, Outs, Outs1?, In, In?) :-
    lookup_send('_user', msg(agent, '_user', rejected), Outs?, Outs1).

procedure handle_response(Response?, AgentId?,
    OutputsList?, OutputsList, Stream?, Stream).
handle_response(accept(ch(FIn, FOut?)), From, Outs, Outs2?, In, In1?) :-
    ground(From?) |
    add_output(From?, FOut, Outs?, Outs1),
    lookup_send('_user', msg(agent, '_user', connected(From?)), Outs1?, Outs2),
    merge(In?, FIn?, In1).
handle_response(no, From, Outs, Outs1?, In, In?) :-
    ground(From?) |
    lookup_send('_user', msg(agent, '_user', rejected(From?)), Outs?, Outs1).
\end{verbatim}
\caption{Channel establishment}
\label{fig:channel-establish}
\end{figure*}

The accepting agent creates a channel pair via \verb|new_channel(RetCh, LocalCh)|. It sends \verb|RetCh| back to the initiator and keeps \verb|LocalCh| for itself. Both channels are complementary: each agent's input is the other's output. Upon acceptance, \verb|add_output| adds the new friend's output stream to the outputs list, and the friend's input stream is merged into the agent's network input via \verb|merge|.

\subsubsection{Text Messaging}

Once agents are friends, they can exchange text messages (Figure~\ref{fig:messaging}).

\begin{figure*}[t]
\begin{multicols}{2}
\begin{verbatim}
%% User sends text to friend
agent(Id,
  [msg('_user', Id1,
    send(Target, Text))|UserIn],
  NetIn, Outs) :-
    Id? =?= Id1?, ground(Target?) |
    lookup_send(Target?,
      msg(Id?, Target?, text(Text?)),
      Outs?, Outs1),
    agent(Id?, UserIn?, NetIn?,
      Outs1?).
\end{verbatim}
\columnbreak
\begin{verbatim}
%% Received text from friend
agent(Id, UserIn,
  [msg(From, Id1, text(Text))|NetIn],
  Outs) :-
    Id? =?= Id1? |
    lookup_send('_user',
      msg(agent, '_user',
        received(From?, Text?)),
      Outs?, Outs1),
    agent(Id?, UserIn?, NetIn?,
      Outs1?).
\end{verbatim}
\end{multicols}
\caption{Text messaging between friends}
\label{fig:messaging}
\end{figure*}

\subsubsection{Friend-Mediated Introduction}

Once agents are friends, they can introduce each other to third parties. The introducer creates a fresh channel pair and sends each half to the respective parties (Figure~\ref{fig:friend-intro}).

\begin{figure*}[t]
\begin{multicols}{2}
\begin{verbatim}
%% User commands: introduce P to Q
agent(Id,
  [msg('_user', Id1,
    introduce(P, Q))|UserIn],
  NetIn, Outs) :-
    Id? =?= Id1?, ground(P?),
    ground(Q?),
    new_channel(PQCh, QPCh) |
    lookup_send(P?,
      msg(Id?, P?, intro(Q?, QPCh?)),
      Outs?, Outs1),
    lookup_send(Q?,
      msg(Id?, Q?, intro(P?, PQCh?)),
      Outs1?, Outs2),
    agent(Id?, UserIn?, NetIn?,
      Outs2?).
\end{verbatim}
\columnbreak
\begin{verbatim}
%% Received introduction from friend
agent(Id, UserIn,
  [msg(From, Id1,
    intro(Other, Ch))|NetIn],
  Outs) :-
    Id? =?= Id1?, ground(Other?) |
    lookup_send('_user',
      msg(agent, '_user',
        befriend_intro(
          From?, Other?, Ch?)),
      Outs?, Outs1),
    agent(Id?, UserIn?, NetIn?,
      Outs1?).

%% User accepts friend introduction
agent(Id,
  [msg('_user', Id1,
    accept_intro(Other, Ch))|UserIn],
  NetIn, Outs) :-
    Id? =?= Id1?, ground(Other?) |
    handle_intro_accept(Ch?, Other?,
      Outs?, Outs1, NetIn?, NetIn1),
    agent(Id?, UserIn?, NetIn1?,
      Outs1?).
\end{verbatim}
\end{multicols}
\caption{Friend-mediated introduction protocol}
\label{fig:friend-intro}
\end{figure*}

When Bob types \verb|introduce(alice, charlie)|, he creates a channel pair via \verb|new_channel(PQCh, QPCh)|. Alice receives \verb|ch(QtoP?, PtoQ)|---she reads from Charlie via \verb|QtoP?| and writes to Charlie via \verb|PtoQ|. Charlie receives the complementary \verb|ch(PtoQ?, QtoP)|. When both accept, they become direct friends without Bob's further involvement.

Note the distinction between cold-call messages and friend-mediated messages: cold-call uses a 2-argument \verb|msg(Target, Content)| sent via the \verb|'_net'| output, while friend-mediated introduction uses a 3-argument \verb|msg(From, To, intro(Other, Ch))| sent directly to a friend via their named output.

\subsubsection{Network Switch Simulation}

In deployment, agents communicate through a physical network. In simulation, a \verb|network3| process routes messages between three agents (Figure~\ref{fig:network-switch}).

\begin{figure*}[t]
\begin{multicols}{2}
\begin{verbatim}
procedure network3(
  Channel?, Channel?, Channel?).

%% Cold-call routing (2-arg msg)

%% Alice cold-calls Bob
network3(
  ch([msg(bob, X)|AliceIn],
     AliceOut?),
  ch(BobIn,
     [msg(bob, X?)|BobOut?]),
  ch(CharlieIn, CharlieOut?)) :-
    network3(
      ch(AliceIn?, AliceOut),
      ch(BobIn?, BobOut),
      ch(CharlieIn?, CharlieOut)).

%% (Five more 2-arg clauses for
%%  other sender/receiver pairs)

%% Friend-to-friend (3-arg msg)

%% Alice -> Bob
network3(
  ch([msg(alice, bob, X)|AliceIn],
     AliceOut?),
  ch(BobIn,
     [msg(alice, bob, X?)
      |BobOut?]),
  ch(CharlieIn, CharlieOut?)) :-
    network3(
      ch(AliceIn?, AliceOut),
      ch(BobIn?, BobOut),
      ch(CharlieIn?, CharlieOut)).
\end{verbatim}
\columnbreak
\begin{verbatim}
%% Bob -> Charlie
network3(
  ch(AliceIn, AliceOut?),
  ch([msg(bob, charlie, X)|BobIn],
     BobOut?),
  ch(CharlieIn,
     [msg(bob, charlie, X?)
      |CharlieOut?])) :-
    network3(
      ch(AliceIn?, AliceOut),
      ch(BobIn?, BobOut),
      ch(CharlieIn?, CharlieOut)).

%% (Four more 3-arg clauses for
%%  other sender/receiver pairs)

%% Termination
network3(
  ch([], []),
  ch([], []),
  ch([], [])).
\end{verbatim}
\end{multicols}
\caption{Network switch simulation for three agents}
\label{fig:network-switch}
\end{figure*}

The network switch routes both 2-argument cold-call messages (where \verb|msg(Target, Content)| is addressed by target name) and 3-argument friend-to-friend messages (where \verb|msg(From, To, Content)| carries both sender and receiver identities). In deployment, the network switch is replaced by maGLP's Cold-call transaction (Section~\ref{sec:maglp}).

\subsubsection{The Scenario and Actors}

The complete scenario demonstrates all three protocols:
\begin{enumerate}
\item Alice cold-calls Bob (Bob accepts) --- Alice and Bob become friends
\item Alice sends Bob: ``Hi Bob, this is Alice''
\item Bob cold-calls Charlie (Charlie accepts) --- Bob and Charlie become friends
\item Charlie sends Bob: ``Hi Bob, this is Charlie''
\item Bob introduces Alice to Charlie (both accept) --- Alice and Charlie become direct friends
\item Alice sends Charlie: ``Hi Charlie, this is Alice''
\item Charlie responds: ``Hi Alice, this is Charlie''
\end{enumerate}

Each agent is driven by an \emph{actor}---a GLP procedure that implements a state machine, reacting to messages from the agent and producing commands. The actor's state is encoded in procedure names (e.g., \verb|alice_wait_bob_connected|, \verb|bob_wait_charlie_msg|), with transitions via recursive calls. See Appendix~\ref{appendix:social-graph-complete} for the complete program: helper predicates, actor implementations, the play that ties everything together, and the multiagent boot variants.

%==============================================================================
\section{Multiagent GLP}\label{sec:maglp}
%==============================================================================

This section extends GLP to multiple agents. We first define multiagent transition systems via multiagent atomic transactions, then define multiagent GLP (maGLP) as a transactions-based multiagent transition system, and finally show how the social graph operates in the multiagent setting.

%------------------------------------------------------------------------------
\subsection{Multiagent Transition Systems}
\label{sec:mts}
%------------------------------------------------------------------------------

We assume a potentially infinite set of \emph{agents} $\Pi$, but consider only finite subsets of it, so when we refer to a particular set of agents $P \subset \Pi$ we assume $P$ to be nonempty and finite. We use $\subset$ to denote the strict subset relation and $\subseteq$ when equality is also possible.

We use $S^P$ to denote the set $S$ indexed by the set $P$, and if $c\in S^P$ we use $c_p$ to denote the member of $c$ indexed by $p\in P$. Intuitively, think of such a $c\in S^P$ as an array of cells indexed by members of $P$ with cell values in $S$.

\begin{definition}[Local States, Configuration, Transaction, Participants]\label{definition:at}
Given agents $Q \subset \Pi$ and an arbitrary set $S$ of \temph{local states}, a \temph{configuration} over $Q$ and $S$ is a member of $C:= S^Q$. An \temph{atomic transaction}, or just \emph{transaction}, over $Q$ and $S$ is any pair of configurations $t=c\rightarrow c' \in C^2$ such that $c\ne c'$, with $t_p := c_p \rightarrow c'_p$ for any $p\in Q$, and with $p$ being an \temph{active participant} in $t$ if $c_p\ne c'_p$, \temph{stationary participant} otherwise.
\end{definition}

\begin{definition}[Degree]\label{definition:degree}
The \temph{degree} of a transaction $t$ (unary, binary, \ldots, $k$-ary) is the number of active participants in $t$, and the \temph{degree} of a set of transactions $T$ is the maximal degree of any $t\in T$.
\end{definition}

\begin{definition}[Multiagent Transition System]\label{definition:mts}
Given agents $P \subset \Pi$ and an arbitrary set $S$ of \temph{local states} with a designated \temph{initial local state} $s0\in S$, a \temph{multiagent transition system} over $P$ and $S$ is a transition system $TS= (C,c0,T)$ with \temph{configurations} $C:= S^P$, \temph{initial configuration} $c0:= \{s0\}^P$, and \temph{transitions} $T\subseteq C^2$ being a set of transactions over $P$ and $S$, with the \temph{degree} of $TS$ being the degree of $T$.
\end{definition}

Rather than specifying a multiagent transition system over a set of agents $P$ directly, we specify it via atomic transactions, which are typically of bounded degree smaller than $|P|$.

\begin{definition}[Transaction Closure]\label{definition:closure}
Let $P\subset \Pi$, $S$ a set of local states, and $C:=S^P$.
For a transaction $t=(c\rightarrow c')$ over local states $S$ with participants $Q \subseteq P$, the \temph{$P$-closure of $t$}, $t{\uparrow}P$, is the set of transitions over $P$ and $S$ defined by:
$$
t{\uparrow}P := \{ t' \in C^2 :
\forall q\in Q.(t_q = t'_q) \wedge \forall p\in P\setminus Q.(p\text{ is stationary in }t')\}
$$
If $R$ is a set of transactions, each $t\in R$ over some $Q\subseteq P$ and $S$, then the \temph{$P$-closure of $R$}, $R{\uparrow}P$, is the set of transitions over $P$ and $S$ defined by:
$$
R{\uparrow}P := \bigcup_{t\in R} t{\uparrow}P
$$
\end{definition}

Namely, the closure over $P\supseteq Q$ of a transaction $t$ over $Q$ includes all transitions $t'$ over $P$ in which members of $Q$ do the same in $t$ and in $t'$, and the rest remain in their current (arbitrary) state.

\begin{definition}[Transactions-Based Multiagent Transition System]\label{definition:tbmts}
Given agents $P \subset \Pi$, local states $S$ with initial local state $s0\in S$, and a set of transactions $R$, each $t\in R$ over some $Q\subseteq P$ and $S$, the \temph{transactions-based multiagent transition system} over $P$, $S$, and $R$ is the multiagent transition system $TS= (S^P,\{s0\}^P,R{\uparrow}P)$.
\end{definition}

In other words, one can fully specify a multiagent transition system over $S$ and $P$ simply by providing a set of atomic transactions over $S$, each with participants $Q\subseteq P$.

%------------------------------------------------------------------------------
\subsection{From GLP to Multiagent GLP}
\label{sec:glp-to-maglp}
%------------------------------------------------------------------------------

In extending GLP to multiple agents, each agent maintains its own asynchronous resolvent as its local state. The key insight is that GLP's variable pairs provide natural binary communication channels: when agent $p$ assigns a writer $X$ for which the paired reader $X?$ is held by agent $q$, the assignment $X := T$ must be communicated to $q$.

A key difference between single-agent GLP and multiagent GLP is in the initial state. In a multiagent transition system all agents must have the same initial local state $s0$ (Definition~\ref{definition:mts}). This precludes setting up an initial configuration in which agents share logic variables, as this would imply different initial states for different agents.

We resolve this in two steps. First, we employ only anonymous logic variables ``\verb|_|'' in the initial local states of agents: Anonymous variables are, on the one hand, syntactically identical, hence allow all initial states to be syntactically identical, and on the other hand represent unique variables, hence semantically all initial goals have unique, local, non-shared variables. The initial state of all agents is the atomic goal \verb|agent(ch(_?,_),ch(_?,_))|, with the first channel serving communication with the user and the second with the network.

Second, the Cold-call transaction enables agents to bootstrap communication by establishing shared variables through the network infrastructure, realizing the cold-call protocol for connecting previously-disconnected agents.

%------------------------------------------------------------------------------
\subsection{Multiagent GLP Definition}
\label{sec:maglp-def}
%------------------------------------------------------------------------------

\begin{definition}[Multiagent GLP]\label{definition:maGLP}
Given agents $P\subset \Pi$ and GLP program $M$, the \temph{maGLP transition system} over $P$ and $M$ is the transactions-based multiagent transition system (Definition~\ref{definition:tbmts}) over $P$, local states being asynchronous resolvents over $M$, initial local state $s0 = (\{\verb|agent(ch(_?,_),ch(_?,_))|\}, \emptyset)$, and the following transactions $c\rightarrow c'$:
\begin{enumerate}
    \item \textbf{Reduce $p$:} A unary transaction with participant $p$ where $c_p\rightarrow c'_p$ is a GLP Reduce transition (Definition~\ref{def:glp-ts}).
    
    \item \textbf{Communicate $p$ to $q$:} A transaction with participants $p,q\in P$ where $c_p=(G_p,\sigma_p)$, $c_q=(G_q,\sigma_q)$, $\{X?:=T\} \in \sigma_p$, $X?$ occurs in $G_q$, $c'_p=(G_p,\sigma_p \setminus \{X?:=T\})$, and $c'_q=(G_q\{X?:=T\},\sigma_q)$. 
    
    \item \textbf{Cold-call $p$ to $q$:} A binary transaction with participants $p\ne q \in P$ where the network output stream in $c_p$ has a new message \verb|msg|$(q,X)$, $c'_p$ is the result of advancing the network output stream in $c_p$, and $c'_q$ is the result of adding $X?$ to the network input stream in $c_q$.
\end{enumerate}
\end{definition}

Note that Communicate maybe unary or binary, depending on whether $p=q$.  Both Communicate and Cold-call transfer assignments from writers to readers: Communicate operates between agents sharing a paired reader and writer, while Cold-call operates through the network streams established in each agent's initial configuration, enabling the creation of paired variables among previously-disconnected agents.
Naturally, Cold-call would be the exception, as once agents share a paired variable they do not need to communicate; moreover, an agent with two friends (with which it shares channels) may introduce them to each other, also saving the need for a Cold-call.
%------------------------------------------------------------------------------
\subsection{The Multiagent Social Graph}
\label{sec:ma-social-graph}
%------------------------------------------------------------------------------

The social graph agent code presented in Section~\ref{sec:social-graph} runs unchanged in maGLP. The only difference is the boot code: the network switch is replaced by maGLP's Cold-call transaction, and each agent runs on a separate isolate. Two boot variants are presented in \ref{appendix:social-graph-complete}: a headless boot with actors (Appendix~\ref{app:maglp-boot}) and an interactive UI boot (Appendix~\ref{app:maglp-ui-boot}).

When Alice executes the cold-call protocol to befriend Bob, her agent sends \verb|msg(bob, intro(alice, Resp))| on the \verb|'_net'| output stream. The Cold-call transaction (Definition~\ref{definition:maGLP}) transfers this message to Bob's network input stream, adding the reader \verb|Resp?| to Bob's resolvent. When Bob accepts, assigning \verb|Resp| to \verb|accept(ch(FIn?, FOut))|, the Communicate transaction transfers this assignment back to Alice.

Similarly, friend-mediated introduction works through Communicate transactions: when Bob introduces Alice to Charlie, the channel pair he creates contains readers that are transferred to Alice and Charlie via Communicate, establishing their direct connection.

The network switch in the concurrent GLP version (Section~\ref{sec:social-graph}) is thus a faithful simulation of maGLP's Cold-call transaction, allowing single-process testing of multiagent protocols.

%------------------------------------------------------------------------------
\subsection{Safety Properties of maGLP}
\label{sec:maglp-safety}
%------------------------------------------------------------------------------

The safety properties established for single-agent GLP extend to maGLP.

\begin{restatable}[Safety Properties of maGLP]{proposition}{propMaGLPSafety}
\label{proposition:maGLP-safety}
The safety properties established for GLP in Section~\ref{sec:glp} extend to maGLP:
\begin{enumerate}
\item \textbf{SO Preservation} (cf.\ Proposition~\ref{prop:so-preservation}): If the initial goals of all agents satisfy SO, then every goal in every agent's resolvent throughout the run satisfies SO.
\item \textbf{Monotonicity} (cf.\ Proposition~\ref{prop:glp-monotonicity}): If unit goal $A$ in agent $p$'s resolvent can reduce with clause $C$ at step $i$, then at any step $j > i$, either $A$ has been reduced or there exists $A'$ in $p$'s resolvent where $A' = A\tau$ for some reader substitution $\tau$, and $A'$ can reduce with $C$.
\end{enumerate}
\end{restatable}

\begin{restatable}[maGLP Computation is Deduction]{proposition}{propMaGLPDeduction}
\label{prop:maglp-deduction}
Let $L$ be the transition system whose resolvent is the union of all local resolvents, the initial goal includes a \texttt{network} goal with channels paired to each agent's network channels, and the program is $M$ augmented with the GLP definition of \texttt{network}. Then maGLP runs correspond to GLP runs of $L$, and their outcomes are logical consequences of the augmented program.
\end{restatable}

\begin{restatable}[maGLP Persistence]{lemma}{lemmaGLPPersistence}
\label{lem:maglp-persistence}
maGLP is persistent.
\end{restatable}

\begin{definition}[maGLP Fair Run]
\label{def:maglp-fair-run}
A complete maGLP run is \temph{fair} if every transaction that becomes enabled is eventually taken.
\end{definition}

\mypara{Implementation}
The implementation-ready variants of GLP and maGLP, along with formal correctness proofs that they implement their respective specifications, are presented in a companion paper~\citep{shapiro2026implementing}.

%==============================================================================
\section{Multiagent GLP is Grassroots}\label{sec:grassroots}
%==============================================================================

This section establishes that maGLP is grassroots, following the framework of~\citep{shapiro2023grassrootsBA,shapiro2025atomic}. Formal definitions appear in Appendix~\ref{appendix:grassroots-defs}.

Informally, a protocol defined via atomic transactions is \emph{grassroots} if its transactions are \emph{interactive}: for any group of agents $P$ running within a larger group $P'$, it is always possible for an outsider in $P' \setminus P$ to interact with a member of $P$ in a way that leaves an ``alien trace'' in $P$'s state---a configuration that $P$ could not have produced on its own.

The Cold-call transaction (Definition~\ref{definition:maGLP}) is the fundamental interactive transaction of maGLP. It allows an agent $q$ outside a group $P$ to connect to an agent $p \in P$, establishing a pair of shared variables: a writer held by one agent and its paired reader held by the other. The resulting shared variable constitutes an alien trace in $p$'s state---a reference that could not have arisen from $P$ running alone.

\begin{restatable}[maGLP is Grassroots]{theorem}{thmMaGLPGrassroots}
\label{theorem:maGLP-grassroots}
The maGLP protocol is grassroots.
\end{restatable}

The grassroots property of maGLP extends to applications built on top of it, provided they use the cold-call mechanism.

\begin{definition}[GLP Application, Cold Call]
A \temph{GLP application} is a GLP program $M$ together with the maGLP infrastructure. An application \temph{uses cold calls} if agents can execute the Cold-call transaction to establish communication with previously-disconnected agents.
\end{definition}

\begin{restatable}[GLP Applications Using Cold Calls are Grassroots]{proposition}{propAppGrassroots}
\label{prop:app-grassroots}
Any GLP application that uses cold calls is grassroots.
\end{restatable}

\begin{restatable}[The Grassroots Social Graph is Grassroots]{corollary}{corSocialGraphGrassroots}
\label{corollary:social-graph-grassroots}
The GLP implementation of the grassroots social graph (Section~\ref{sec:social-graph}) is grassroots.
\end{restatable}

\section{Related Work}\label{section:related-work}

Grassroots platforms require agents to verify cryptographic identity and protocol compatibility upon contact, form authenticated channels, and coalesce spontaneously without global coordination. The language must support multiple concurrent platform instances and metaprogramming for tooling development. We examine how existing systems address these requirements.

\mypara{Concurrent Logic Programming}
GLP belongs to the family of concurrent logic programming (CLP) languages that emerged in the 1980s: Concurrent Prolog~\cite{shapiro1983subset}, GHC~\cite{ueda1986guarded}, and PARLOG~\cite{clark1986parlog}. These languages interpret goals as concurrent processes communicating through shared logical variables, using committed-choice execution with guarded clauses. Shapiro's comprehensive survey~\cite{shapiro1989family} documents this family and its design space.

A key evolution was \emph{flattening}: restricting guards to primitive tests only. Flat Concurrent Prolog (FCP)~\cite{mierowsky1985fcp} and Flat GHC~\cite{ueda1986guarded} demonstrated that flat guards suffice for practical parallel programming while dramatically simplifying semantics and implementation.

GLP can be understood as \textbf{Flat Concurrent Prolog with the Single-Reader Single-Writer (SRSW) constraint}. FCP introduced read-only annotations (\verb|?|) distinguishing readers from writers of shared variables, enabling dataflow synchronization. However, read-only unification proved semantically problematic: Levi and Palamidessi~\cite{levi1985readonly} showed it is order-dependent, and Mierowsky et al.~\cite{mierowsky1985fcp} documented non-modularity issues. GHC dispensed with read-only annotations entirely, relying on guard suspension semantics.

GLP's SRSW constraint---requiring that each variable has exactly one writer and one reader occurrence---resolves these difficulties by ensuring that (1) no races occur on variable binding, and (2) term matching suffices, eschewing unification entirely. The result is a cleaner semantic foundation while preserving the expressiveness of stream-based concurrent programming. GLP retains logic programming's metaprogramming capabilities~\cite{safra1988meta,lichtenstein1988concurrent,shapiro1984systems}, essential for platform tooling development.

\mypara{Modes in Concurrent Logic Programming}
Mode systems for CLP have a rich history. PARLOG used mode declarations at the predicate level, with input modes enforcing one-way matching. Ueda's work on moded Flat GHC~\cite{ueda1994moded,ueda1995io} is most directly relevant: his mode system assigns polarity to every variable occurrence (positive for input/read, negative for output/write), with the \emph{well-modedness} property guaranteeing each variable is written exactly once. Ueda's subsequent linearity analysis~\cite{ueda2001resource} identifies variables read exactly once, enabling compile-time garbage collection. GLP enforces both single-reader and single-writer universally as a syntactic restriction, whereas Ueda's system guarantees single-writer with single-reader as an optional refinement.

\mypara{Distributed actor and process languages}
Actor-based languages (Erlang/OTP~\cite{armstrong2013programming}, Akka~\cite{akka2022}, Pony~\cite{clebsch2015deny}) and active object languages~\cite{boer2017survey,boer2024active} provide message-passing concurrency and fault isolation. However, their security models operate at the transport layer (TLS in Akka Remote~\cite{akka2022}, Erlang's cookie-based authentication~\cite{armstrong2013programming}) rather than integrating cryptographic identity and code attestation into language primitives. Orleans~\cite{orleans2022} assumes trusted runtime environments, lacking the attestation mechanisms required for grassroots platforms where participants must verify code integrity without central coordination.

\mypara{Capability security}
E~\cite{miller2006robust} provides capability-based security through unforgeable object references with automatic encryption. While ensuring object uniqueness and access control, E does not address verifying real-world identity or protocol implementation attestation---distinct requirements for grassroots platforms.

\mypara{Linear logic and futures/promises}
GLP's SRSW constraint has deep connections to two foundational ideas. Girard's linear logic~\cite{girard1987linear} introduced the principle that resources must be used exactly once; GLP's single-writer/single-reader restriction enforces this at the variable level, ensuring each assignment is produced exactly once and consumed exactly once. Friedman and Wise's futures and promises~\cite{friedman1976impact} introduced placeholders for values computed asynchronously; GLP's reader/writer pairs realize this pattern, with the writer as promise (single producer) and reader as future (single consumer). The Caires-Pfenning-Toninho line of work~\cite{caires2010session,caires2016linear} established deep connections between linear logic and session types; GLP achieves similar resource-discipline properties through syntactic restrictions rather than type systems.

\mypara{Linear types and session types}
Linear types~\cite{wadler1990linear} ensure single-use of resources, similar to GLP's single-writer constraint. However, GLP's SRSW mechanism provides bidirectional pairing---each writer has exactly one reader---enabling authenticated channels without type-level tracking. Session types~\cite{honda1993types} specify communication protocols statically, with implementations in Links~\cite{cooper2007links,lindley2017lightweight}, Rust~\cite{jespersen2015session}, Scala~\cite{scalas2016lightweight}, and Go~\cite{castro2019distributed}. While these verify protocol conformance at compile time, GLP's reader/writer synchronization enforces protocol dynamically through suspension and resumption, and runtime attestation enables participants to verify protocol compatibility when establishing connections between independently-deployed agents.

\mypara{Concurrent coordination languages}
Concurrent ML~\cite{reppy1999concurrent} provides first-class synchronous channels and events. The Join Calculus~\cite{fournet1996reflexive} offers pattern-based synchronization through join patterns. GLP's SRSW variables provide asynchronous communication through reader/writer pairs with the monotonicity property (Proposition~\ref{prop:glp-monotonicity}) ensuring suspended goals remain reducible once readers are instantiated. However, neither provides mechanisms for cryptographic identity verification or authenticated channel establishment required for grassroots platforms.

\mypara{Blockchain programming languages}
Smart contract languages like Solidity~\cite{mukhopadhyay2018ethereum} and Move~\cite{move2022} provide deterministic execution and asset safety but assume blockchain infrastructure for identity and consensus. While Scilla~\cite{scilla2018} separates computation from communication similar to GLP's message-passing model, it targets on-chain state transitions rather than peer-to-peer authenticated channels. GLP achieves security properties through the language-level SRSW invariant and attestations, without requiring global consensus.

\mypara{Authorization languages}
OPA/Rego~\cite{opa2021} and Cedar~\cite{hicks2023cedar} provide declarative policy specification but are specialized for policy evaluation. They consume authentication tokens as inputs but do not integrate attestation as first-class primitives for verifying remote code execution.

\mypara{Grassroots platforms}
The mathematical foundations for grassroots platforms were established in~\cite{shapiro2023grassrootsBA}, with atomic transactions introduced in~\cite{shapiro2025atomic}. Grassroots social networks~\cite{shapiro2023gsn}, cryptocurrencies~\cite{shapiro2024gc,lewis2023grassroots}, and federations~\cite{halpern2024federated,shapiro2025GF} have been formally specified and proven grassroots. GLP provides the programming language to implement these specifications.

\section{Conclusion}\label{section:conclusion}

We have presented GLP, a secure, multiagent, concurrent logic programming language designed for implementing grassroots platforms. The key contributions are:

\begin{enumerate}
\item \textbf{Mathematical Foundations:} We recalled the framework of multiagent transition systems with atomic transactions and the key theorem that protocols over interactive transactions are grassroots.

\item \textbf{GLP Definition:} We defined GLP as an extension of logic programs with reader/writer variable pairs satisfying the Single-Reader/Single-Writer (SRSW) restriction, providing asynchronous communication through term matching rather than unification.

\item \textbf{Grassroots Proof:} We proved that multiagent GLP (maGLP) is grassroots by showing it is a transactions-based protocol with interactive transactions (via the Network transaction for cold calls).

\item \textbf{Application Criterion:} We established that any GLP application using cold calls inherits the grassroots property from maGLP.

\item \textbf{Grassroots Social Graph:} We demonstrated the utility of GLP by implementing the grassroots social graph protocol and proving it grassroots.
\end{enumerate}

The grassroots social graph serves as the infrastructure layer for other grassroots platforms. Future work includes implementing grassroots social networks, grassroots cryptocurrencies, and grassroots federations in GLP, as well as developing the Dart/Flutter implementation for smartphone deployment.


\bibliographystyle{tlplike}
\bibliography{bib}

\appendix

\section{Logic Programs}\label{appendix:lp}

We recall standard Logic Programs (LP) notions of syntax, most-general unifier (mgu), and semantics via goal reduction.

\begin{definition}[Logic Programs Syntax]
\label{def:lp-syntax}
We employ standard LP notions. Let $\calV$ denote the set of \temph{variables} (identifiers beginning with uppercase). A \temph{term} is a variable, a constant (numbers, strings, or the empty list \verb|[]|), or a compound term $f(T_1,\ldots,T_n)$ with functor $f$ and subterms $T_i$. Let $\calT$ denote the set of all terms. We use standard list notation: \verb=[X|Xs]= for a list cell, \verb|[X1,...,Xn]| for finite lists. A term is \temph{ground} if it contains no variables.

A \temph{unit goal} is a compound term, also commonly referred to as an \temph{atom}. A \temph{goal} is a multiset of unit goals; the empty goal is written \verb|true|. A \temph{clause} $A$~\verb|:-|~$B$ has head $A$ (a unit goal) and body $B$ (a goal); a \temph{unit clause} has empty body. A \temph{logic program} is a finite set of clauses; clauses for the same predicate form a \temph{procedure}. Let $\calG(P)$ denote the set of goals over the vocabulary of the program $P$.
\end{definition}

A \emph{substitution} $\sigma$ is an idempotent function $\sigma: \calV \to \calT$, a mapping from variables to terms applied to a fixed point. By convention, $\sigma(x)=x\sigma$. Let $\Sigma$ denote the set of all substitutions. We assume standard notions of instance, ground, renaming, renaming apart, unifier, and most-general unifier (mgu).

\begin{definition}[LP Goal/Clause Reduction]
\label{def:lp-reduction}
Given an LP unit goal $A$ and clause $C$, with $H$ \verb|:-| $B$ being the result of renaming $C$ apart from $A$, the \temph{LP reduction} of $A$ with $C$ \temph{succeeds with} $(B,\sigma)$ if $A$ and $H$ have an mgu $\sigma$.
\end{definition}

\begin{definition}[Logic Programs Transition System]
\label{def:lp-ts}
A transition system $LP(P) = (C, c_0, T)$ is a \temph{Logic Programs transition system} for a logic program $P$ and initial goal $G_0 \in \mathcal{G}(P)$, if $C=\mathcal{G}(P)\times \Sigma$, $c_0=(G_0,\emptyset)$, and $T$ is the set of all transitions $(G,\sigma) \rightarrow (G',\sigma')$ such that for some unit goal $A \in G$ and clause $C \in P$ the LP reduction of $A$ with $C$ succeeds with $(B,\hat\sigma)$, $G' = (G \setminus \{A\} \cup B)\hat\sigma$, and $\sigma'=\sigma\circ\hat\sigma$.
\end{definition}

LP has two forms of nondeterminism: the choice of $A \in G$, called \emph{and-nondeterminism}, and the choice of $C \in P$, called \emph{or-nondeterminism}, and as such are closely-related to Alternating Turing Machines~\citep{shapiro1984alternation}.

\begin{definition}[Proper and Successful Run, Outcome]
\label{def:proper-run}
A run $\rho: (G_0,\sigma_0) \rightarrow \cdots \rightarrow (G_n, \sigma_n)$ of $LP(P)$ is \temph{proper} if for any $1\le i< n$, a variable that occurs in $G_{i+1}$ but not in $G_i$ also does not occur in any $G_j$, $j<i$. If proper, the \temph{outcome} of $\rho$ is $(G_0$ \verb|:-| $G_n)\sigma_n$. Such a run is \temph{successful} if $G_n=\emptyset$.
\end{definition}
The following proposition justifies the computation-as-deduction view of LP~\citep{kowalski1974predicate},
calling a proper LP run a \emph{derivation} and a complete proper run ending in the empty goal a \emph{successful derivation}.

\begin{restatable}[LP Computation is Deduction]{proposition}{propLPDeduction}
\label{prop:lp-deduction}
The outcome $(G_0$ \verb|:-| $G_n)\sigma$ of a proper run of $LP(P)$ is a logical consequence of $P$.
\end{restatable}

The $LP(P)$ transition system also supports denotational semantics, including clause semantics~\citep{gaifman1989fully}, atom semantics, and ground atom semantics~\citep{lloyd1987foundations}; see \ref{appendix:denotational}.

\section{Term Matching}\label{appendix:term-matching}

If two terms $T_1$ and $T_2$ that jointly satisfy SO are unifiable with an mgu $\sigma$, then $\sigma$ maps any variable in $T_1$ to a subterm of $T_2$ and vice versa. Hence, the SO invariant of GLP allows eschewing unification in favour of \emph{term matching} that performs joint term-tree traversal and collects variable assignments along the way.

\begin{definition}[Term Matching]
\label{def:term-matching}
Given two terms $T_1$ and $T_2$ that jointly satisfy SO, their \temph{term matching} proceeds via the joint traversal of the term-trees of $T_1$ and $T_2$, consulting the following table at each pair of joint vertices, where $X_1, X_2$ denote writers, $X_1?, X_2?$ denote readers, and $f/n$ denotes a non-variable term, a constant when $n=0$ and a compound term when $n>0$:
\begin{center}
\begin{tabular}{l|lll}
$T_1 \backslash T_2$ & Writer $X_2$ & Reader $X_2?$ & Term $f_2/n_2$ \\
\hline
Writer $X_1$ & fail & $X_1 := X_2?$ & $X_1 := T_2$ \\
Reader $X_1?$ & $X_2 := X_1?$ & fail & suspend on $X_1?$\\
Term $f_1/n_1$ & $X_2 := T_1$ & fail & fail if $f_1 \ne f_2$ or $n_1 \ne n_2$\\
\end{tabular}
\end{center}
The writer mgu is the union of all writer assignments if no \emph{fail} was encountered and the suspension set is empty.
\end{definition}

\begin{remark}
In an actual implementation, assuming $T_1$ is a goal term and $T_2$ a head term, the case of $X_1?$ and $T_2$ would add $X_1?$ to the set of readers the goal would suspend upon.
\end{remark}

\section{Denotational Semantics}\label{appendix:denotational}

The $LP(P)$ transition system allows defining several denotational semantic notions for a program $P$: (1) the \emph{clause semantics} is the set of all outcomes of all proper runs with an initial most-general unit goal (arguments are distinct variables), closely related to the fully-abstract compositional semantics of LP~\citep{gaifman1989fully}; (2) the \emph{atom semantics} is the set of all outcomes of all successful derivations with an initial most-general unit goal; (3) the \emph{ground atom semantics} is the standard model-theoretic semantics, the set of ground instances of the atom semantics over the Herbrand universe of $P$~\citep{lloyd1987foundations}.

\section{Grassroots Definitions}\label{appendix:grassroots-defs}

This appendix contains the full definitions for protocols and the grassroots property referenced in Section~\ref{sec:grassroots}.

%------------------------------------------------------------------------------
\subsection{Protocols and the Grassroots Property}
%------------------------------------------------------------------------------

A protocol is a family of multiagent transition systems, one for each set of agents $P\subset \Pi$, which share an underlying set of local states $\calS$ with a designated initial state $s0$.

\begin{definition}[Local-States Function]\label{definition:local-states-function}
A \temph{local-states function} $S: 2^\Pi \to 2^\calS$ maps every set of agents $P \subset \Pi$ to a set of local states $S(P) \subset \calS$ that includes $s0$ and satisfies $P \subset P' \subset \Pi \implies S(P) \subset S(P')$.
\end{definition}

Given a local-states function $S$, we use $C$ to denote configurations over $S$, with $C(P) := S(P)^P$ and $c0(P):= \{s0\}^P$.

\begin{definition}[Protocol]\label{definition:protocol}
A \temph{protocol} $\calF$ over a local-states function $S$ is a family of multiagent transition systems that has exactly one transition system $\calF(P) = (C(P),c0(P),T(P))$ for every $P \subset \Pi$.
\end{definition}

\begin{definition}[Projection]\label{definition:projection}
Let $\emptyset \subset P \subset P' \subset \Pi$.
If $c'$ is a configuration over $P'$ then $c'/P$, the \temph{projection of $c'$ over $P$}, is the configuration $c$ over $P$ defined by $c_p := c'_p$ for every $p\in P$.
\end{definition}

Note that in the definition above, $c_p$, the state of $p$ in $c$, is in $S(P')$, not in $S(P)$, and hence may include elements ``alien'' to $P$, e.g., logic variables shared with $q\in P'\setminus P$.

Being oblivious implies that if a run of $\calF(P')$ reaches some configuration $c'$, then anything $P$ could do on their own in the configuration $c'/P$ (with a transition from $T(P)$), they can still do in the larger configuration $c'$ (with a transition from $T(P')$), effectively being oblivious to members of $P'\setminus P$.

Being interactive is a weak liveness requirement: no matter what members of $P$ do, if they run within a larger set of agents it is always the case that they can eventually interact with non-$P$'s in a way that leaves ``alien traces'' in the local states of $P$, so that the resulting configuration $c'/P$ could not have been produced by $P$ running on their own.

%------------------------------------------------------------------------------
\subsection{Transactions-Based Protocols}
%------------------------------------------------------------------------------

\begin{definition}[Transactions Over a Local-States Function]\label{definition:transactions-lsf}
Let $S$ be a local-states function. A set of transactions $R$ is \temph{over $S$} if every transaction $t\in R$ is a multiagent transition over $Q$ and $S(Q')$ for some $Q \subseteq Q'\subset \Pi$. Given such a set $R$ and $P\subset \Pi$,
$R(P) := \{ t\in R : t \text{ is over } Q \text{ and } S(Q'), Q \subseteq Q'\subseteq P\}$.
\end{definition}

\begin{definition}[Transactions-Based Protocol]\label{definition:tb-protocol}
Let $S$ be a local-states function and $R$ a set of transactions over $S$.
Then a \temph{protocol $\calF$ over $R$ and $S$} includes for each set of agents $P\subset \Pi$ the transactions-based multiagent transition system $\calF(P)$ over $P$, $S(P)$, and $R(P)$: $\calF(P) := (S(P)^P,\{s0\}^P,R(P){\uparrow}P)$.
\end{definition}

\begin{definition}[Interactive Transactions]\label{definition:interactive-transactions}
A set of transactions $R$ over a local-states function $S$ is \temph{interactive} if for every $\emptyset \subset P \subset P' \subset \Pi$ and every configuration $c\in C(P')$ such that $c/P\in C(P)$, there is a computation $(c\xrightarrow{*} c')\subseteq R(P'){\uparrow}P'$ for which $c'/P\notin C(P)$.
\end{definition}

\section{Deferred Proofs}\label{appendix:proofs}

This appendix contains deferred proofs from Sections~\ref{sec:glp}, \ref{sec:maglp}, and~\ref{sec:grassroots}.

%------------------------------------------------------------------------------
\subsection{GLP Proofs}
\label{app:glp-proofs}
%------------------------------------------------------------------------------

\propGLPDeduction*

\begin{proof}
The $/\!?$ operator replaces every reader $X?$ by its paired writer $X$, transforming GLP terms into LP terms. We show that the GLP run $\rho$ corresponds to an LP run $\rho/?$ of $LP(P/?)$.

Consider a GLP transition $(G, \sigma) \rightarrow (G', \sigma')$:
\begin{itemize}
    \item \emph{Reduce transition}: Goal $A$ reduces with clause $C$ via writer mgu $\hat\sigma$. Applying $/\!?$, the clause $C/?$ is an LP clause, and $A/?$ unifies with the head $H/?$ via the mgu $\hat\sigma/?$ (since writers map to writers). This is a valid LP reduction.
    \item \emph{Communicate transition}: A reader $X? \in G$ is replaced by the value $T$ assigned to its paired writer. Under $/\!?$, both $X?$ and $X$ map to $X$, so this transition becomes the identity---the variable $X$ is already assigned $T$ in the LP view.
\end{itemize}

Thus each GLP transition corresponds to zero or one LP transitions, and the GLP run $\rho$ projects to an LP run $\rho/?$ of $LP(P/?)$. By standard LP soundness, the outcome of $\rho/?$ is a logical consequence of $P/?$.
\end{proof}

\propSOPreservation*

\begin{proof}
By induction on the length of the run. The base case is immediate: $G_0$ satisfies SO by assumption.

For the inductive step, assume $G$ satisfies SO and consider a transition $(G, \sigma) \rightarrow (G', \sigma')$:
\begin{itemize}
    \item \emph{Reduce transition}: Goal $A \in G$ reduces with clause $C = (H \mathrel{\mbox{\texttt{:-}}} B)$ via writer mgu $\hat\sigma$, yielding $G' = (G \setminus \{A\} \cup B)\hat\sigma$. Since $C$ satisfies SRSW, it satisfies SO. Since $C$ is renamed apart from $G$, the variables in $B$ are fresh. The writer mgu $\hat\sigma$ maps writers in $A$ to subterms of $H$ and vice versa; by SO of both $G$ and $C$, each variable is assigned at most once. Applying $\hat\sigma$ to $(G \setminus \{A\} \cup B)$ replaces each variable by a term containing fresh variables (from $B$) or ground subterms. Since no variable in $G \setminus \{A\}$ occurs in $A$ (by SO of $G$), and no variable in $B$ occurs in $G$ (by renaming apart), $G'$ satisfies SO.

    \item \emph{Communicate transition}: $G' = G\hat\sigma?$ where $\hat\sigma? = \{X? := T\}$. Since $G$ satisfies SO, $X?$ occurs at most once in $G$. Replacing this single occurrence by $T$ (which satisfies SO by Definition~\ref{def:writers-assignment}) preserves SO, provided $T$ shares no variables with the rest of $G$. By the proper run condition, variables in $T$ are fresh, so $G'$ satisfies SO.
\end{itemize}
\end{proof}

\propMonotonicity*

\begin{proof}
Suppose goal $A$ can reduce with clause $C$ at step $i$, meaning the writer mgu of $A$ and the head $H$ of (a renaming of) $C$ succeeds. Consider what can change between steps $i$ and $j > i$:
\begin{itemize}
    \item \emph{Reduce transitions on other goals}: These do not affect $A$ directly. By SO, no other goal shares a writer with $A$, so no other reduction can assign a writer in $A$.

    \item \emph{Communicate transitions}: These assign readers, not writers. A communicate transition $X? := T$ may instantiate a reader $X? \in A$, yielding $A' = A\{X? := T\}$. We show $A'$ can still reduce with $C$:

    The original writer mgu succeeded, meaning at position $p$ where $X?$ occurred in $A$, either (a) $H$ had a writer $Y$ at position $p$, yielding assignment $Y := X?$, or (b) $H$ had a reader $Y?$ at position $p$, which would have caused failure (reader-reader), contradiction.

    In case (a), after the communicate transition, $A'$ has $T$ at position $p$. The clause $C$ (renamed apart for $A'$) has a fresh writer $Y'$ at position $p$. The writer mgu now yields $Y' := T$, which succeeds.

    \item \emph{Reduce transition on $A$}: If $A$ itself is reduced at some step $k$ with $i < k \le j$, then an instance of $A$ has been reduced, satisfying the proposition.
\end{itemize}

Thus, if $A$ has not been reduced by step $j$, the (possibly instantiated) goal $A'$ at step $j$ can still reduce with a fresh renaming of $C$.
\end{proof}

\lemPersistence*

\begin{proof}
For Reduce: if a Reduce transition is enabled for goal $A$ in configuration $(G, \sigma_r)$, it remains enabled in any configuration $(G, \sigma'_r)$ with $\sigma_r \subseteq \sigma'_r$, since other Reduce transitions assign disjoint writers (by SO) and Communicate transitions only instantiate readers. For Communicate: applying $\{X? := T\} \in \sigma$ to $X? \in G$ remains enabled since Reduce transitions do not remove assignments from $\sigma$ and other Communicate transitions apply different assignments.
\end{proof}

%------------------------------------------------------------------------------
\subsection{maGLP Proofs}
\label{app:maglp-proofs}
%------------------------------------------------------------------------------

\propMaGLPSOPreservation*

\begin{proof}
Identical to single-agent GLP (Proposition~\ref{prop:so-preservation}), substituting ``agent $p$'s resolvent'' for ``resolvent'' and noting that Communicate and Cold-call transitions transfer only reader assignments, which do not affect SO.
\end{proof}

\propMaGLPMonotonicity*

\begin{proof}
Identical to single-agent GLP (Proposition~\ref{prop:glp-monotonicity}), substituting ``agent $p$'s resolvent'' for ``resolvent'' and noting that Reduce transitions operate locally within each agent whilst Communicate and Cold-call transitions preserve reducibility through reader assignment transfer.
\end{proof}

\lemmaGLPPersistence*

\begin{proof}
Reduce persistence follows from GLP persistence (Lemma~\ref{lem:persistence}). Communicate and Cold-call transactions remain enabled since Reduce transitions do not remove reader assignments or network messages.
\end{proof}

%------------------------------------------------------------------------------
\subsection{Grassroots Proofs}
\label{app:grassroots-proofs}
%------------------------------------------------------------------------------

\propOblivious*

\begin{proof}
Let $\calF$ be a protocol over transactions $R$ and local-states function $S$. Let $P \subset P' \subset \Pi$. We must show $T(P){\uparrow}P' \subseteq T(P')$.

By Definition~\ref{definition:tb-protocol}, $T(P) = R(P){\uparrow}P$. A transition $t' \in T(P){\uparrow}P'$ is derived from some $t \in R(P)$ by extending to agents in $P'$ who remain stationary. Since $t \in R(P)$ has participants $Q \subseteq P \subset P'$ and is over $S(Q')$ for some $Q' \subseteq P \subset P'$, we have $t \in R(P')$. Hence $t' \in R(P'){\uparrow}P' = T(P')$.
\end{proof}

\thmInteractiveGrassroots*

\begin{proof}
Let $\calF$ be a protocol over a set of transactions $R$, where $R$ is interactive.
Since $\calF$ is a transactions-based protocol then, according to Proposition~\ref{proposition:oblivious}, $\calF$ is oblivious. And since $\calF$ is over an interactive set of transactions, $\calF$ is interactive.
Therefore, by Definition~\ref{definition:grassroots}, $\calF$ is grassroots.
\end{proof}

\thmMaGLPGrassroots*

\begin{proof}
By Theorem~\ref{theorem:interactive-grassroots}, it suffices to show that maGLP is a transactions-based protocol with interactive transactions.

\emph{maGLP is transactions-based}: By Definition~\ref{definition:maGLP}, maGLP is defined via a set of transactions (Reduce, Communicate, Cold-call) over a local-states function (asynchronous resolvents) with a common initial state. Hence maGLP is a transactions-based protocol (Definition~\ref{definition:tb-protocol}).

\emph{maGLP transactions are interactive}: Let $\emptyset \subset P \subset P' \subseteq \Pi$ and let $c \in C(P')$ be a configuration such that $c/P \in C(P)$. We must show there exists a computation $c \xrightarrow{*} c'$ of maGLP$(P')$ such that $c'/P \notin C(P)$.

Since $c/P \in C(P)$, the local states of agents in $P$ contain no variables shared with agents in $P' \setminus P$---all their reader/writer pairs are ``internal'' to $P$.

Consider any agent $p \in P$ and any agent $q \in P' \setminus P$. Agent $p$ can execute a Reduce transaction that sends a message \verb|msg|$(q,X)$ on its network output stream, where $X$ is a fresh writer. Then the Cold-call transaction from $p$ to $q$ adds $X?$ to agent $q$'s network input stream. Agent $q$ can then execute Reduce transactions that assign $X?$ some term $T$, creating a reader assignment $X? := T$. Finally, the Communicate transaction from $q$ to $p$ applies this assignment to $p$'s resolvent.

The result is that agent $p$'s local state now contains a term $T$ that originated from agent $q \in P' \setminus P$. This constitutes an ``alien trace''---the configuration $c'/P$ contains references to variables or terms that could not have been produced by agents in $P$ running alone. Hence $c'/P \notin C(P)$.

Since such a computation exists for any valid configuration $c$, the maGLP transactions are interactive (Definition~\ref{definition:interactive-transactions}).

By Theorem~\ref{theorem:interactive-grassroots}, maGLP is grassroots.
\end{proof}

\propAppGrassroots*

\begin{proof}
A GLP application using cold calls is a restriction of maGLP to a specific program $M$. The Cold-call transaction remains available, as cold calls are used. The proof of Theorem~\ref{theorem:maGLP-grassroots} relies only on the availability of the Cold-call transaction to establish interactivity. Since cold calls provide this mechanism, the application inherits the interactive property.

The application is transactions-based by construction (being a restriction of maGLP). By Proposition~\ref{proposition:oblivious}, it is oblivious. By the interactivity argument above, it is interactive. Therefore, by Definition~\ref{definition:grassroots}, it is grassroots.
\end{proof}

\corSocialGraphGrassroots*

\begin{proof}
The social graph uses cold calls for initial contact between disconnected agents. By Proposition~\ref{prop:app-grassroots}, it is grassroots.
\end{proof}

\section{Guards and System Predicates}\label{appendix:guards-system}

Guards and system predicates extend GLP programs with access to the GLP runtime state, operating system and hardware capabilities.

\mypara{Guard predicates}
Guards provide read-only access to the runtime state of GLP computation. A guard appears after the clause head, separated by \verb=|=, and must be satisfied for the clause to be selected. The following builtin guards are fundamental for concurrent GLP programming:

\begin{itemize}
\item \verb|ground(X)| succeeds if \verb|X| is ground (contains no variables).
\item \verb|known(X)| succeeds if \verb|X| is not a variable, though it may not be ground.
\item \verb|writer(X)| and \verb|reader(X)| succeed if \verb|X| is an uninstantiated writer or reader respectively. Note that \verb|reader(X)| is non-monotonic.
\item \verb|otherwise| succeeds if all previous clauses for this procedure failed.
\item \verb|X =?= Y| succeeds if both arguments are ground and equal. For example, \verb|f(a,X?) =?= f(b,Z?)| fails (not suspends) because the ground subterms \verb|a| and \verb|b| already differ. Note that the implementation checks terms left-to-right and does not guarantee early failure detection; \verb|f(X?,a) =?= f(Y?,b)| may suspend rather than fail.
\item Arithmetic comparison guards \verb|X < Y|, \verb|X > Y|, \verb|X =< Y|, \verb|X >= Y|, \verb|X =:= Y|, \verb|X =\= Y| succeed if both arguments are ground numbers satisfying the relation.
\end{itemize}

When a guard's success condition implies that a variable is ground, the clause body may contain multiple occurrences of that variable's reader without violating the single-writer requirement, enabling safe replication of ground terms to multiple concurrent consumers.

\mypara{Defined guard predicates}
To support abstract data types and cleaner code organization, GLP provides for user-defined guards via unit clauses. A unit clause \verb|p(T1,...,Tn).| defines a guard predicate; the call \verb|p(S1,...,Sn)| in guard position is unfolded to the term matching of \verb|T1| with \verb|S1|, ..., \verb|Tn| with \verb|Sn|. For example, the equality guard is defined by the clause \verb|X = X.|, so the guard \verb|A = B| unfolds to matching both \verb|A| and \verb|B| against the same variable \verb|X|.

\mypara{System predicates}
System predicates execute atomically with goal/clause reduction and provide access to underlying runtime services:
\begin{itemize}
\item \verb|evaluate(Expr?,Result)| evaluates ground arithmetic expressions.
\item \verb|current_time(T)| provides system timestamps for temporal coordination.
\item \verb|variable_name(X,Name)| returns a unique identifier for variable \verb|X| and its pair.
\end{itemize}

\mypara{Arithmetic evaluation in assignments}
Arithmetic expressions are defined by the following clause:
\begin{verbatim}
X? := E :- ground(E) | evaluate(E?,X).
\end{verbatim}
Ensuring the expression is ground before calling the system evaluator, maintaining program safety whilst providing convenient notation for mathematical computations.


\input{sections/appendix-additional-techniques}
\section{Social Graph Walkthrough}\label{appendix:social-graph-walkthrough}

This appendix presents the detailed walkthrough of the grassroots social graph program introduced in Section~\ref{sec:social-graph}.

%------------------------------------------------------------------------------
\subsection{Types and Channels}
%------------------------------------------------------------------------------

The social graph uses the following type definitions:
\begin{verbatim}
Response ::= accept(Channel) ; no.
OutputEntry ::= output(String, Stream?).
OutputsList ::= [] ; [OutputEntry|OutputsList].
NetMsg ::= msg(Constant, _).
NetStream ::= [] ; [NetMsg|NetStream].
AgentId ::= Constant.
Decision ::= yes ; no.
\end{verbatim}

A \verb|Channel| is a pair of streams for bidirectional communication. The \verb|new_channel| guard creates complementary channel endpoints:
\begin{verbatim}
new_channel(ch(Xs?, Ys), ch(Ys?, Xs)).
\end{verbatim}
Both endpoints read from their first stream and write to their second:
The first  reads from \verb|Xs?| and writes to \verb|Ys|; the second reads from \verb|Ys?| and writes to \verb|Xs|.

%------------------------------------------------------------------------------
\subsection{Agent Structure}
%------------------------------------------------------------------------------

Each agent processes messages from two separate input streams---user and network---and maintains an outputs list mapping named destinations to output streams:
\begin{verbatim}
procedure agent(AgentId?, Stream?, Stream?, OutputsList?).
\end{verbatim}
The four arguments are: agent identity, user input stream, network input stream, and an outputs list. The outputs list initially contains two entries---\verb|output('_user', ...)|  and \verb|output('_net', ...)|---providing streams to the user interface and the network, respectively. As the agent befriends others, new entries are added.

%------------------------------------------------------------------------------
\subsection{Cold-Call Befriending Protocol}
%------------------------------------------------------------------------------

The cold-call protocol enables agents to establish friendship without prior shared variables. Figure~\ref{fig:cold-call} shows the protocol clauses.

\begin{figure*}[t]
\begin{verbatim}
%% User initiates cold call
agent(Id, [msg('_user', Id1, connect(Target))|UserIn], NetIn, Outs) :-
    Id? =?= Id1?, ground(Target?) |
    lookup_send('_net', msg(Target?, intro(Id?, Resp)), Outs?, Outs1),
    inject_msg(Resp?, Target?, Id?, UserIn?, UserIn1),
    agent(Id?, UserIn1?, NetIn?, Outs1?).

%% Received cold-call introduction from net (2-arg msg)
agent(Id, UserIn, [msg(Id1, intro(From, Resp))|NetIn], Outs) :-
    Id? =?= Id1? |
    lookup_send('_user', msg(agent, '_user', befriend(From?, Resp?)), Outs?, Outs1),
    agent(Id?, UserIn?, NetIn?, Outs1?).

%% User decision on cold-call
agent(Id, [msg('_user', Id1, decision(Dec, From, Resp?))|UserIn], NetIn, Outs) :-
    Id? =?= Id1? |
    bind_response(Dec?, From?, Resp, Outs?, Outs1, NetIn?, NetIn1),
    agent(Id?, UserIn?, NetIn1?, Outs1?).

%% Response to sent cold-call (injected into UserIn by inject_msg)
agent(Id, [msg(From, Id1, response(Resp))|UserIn], NetIn, Outs) :-
    Id? =?= Id1? |
    handle_response(Resp?, From?, Outs?, Outs1, NetIn?, NetIn1),
    agent(Id?, UserIn?, NetIn1?, Outs1?).
\end{verbatim}
\caption{Cold-call befriending protocol}
\label{fig:cold-call}
\end{figure*}

The protocol works as follows: (1) Alice sends \verb|connect(bob)| to her agent via the \verb|'_user'| stream; (2) her agent sends a 2-argument \verb|msg(bob, intro(alice, Resp))| via the \verb|'_net'| output, including a fresh response variable \verb|Resp|; (3) Bob receives \verb|msg(bob, intro(alice, Resp))| on his network input and forwards the request to his user interface; (4) Bob decides \verb|yes| or \verb|no|; (5) Bob's agent creates a channel pair and sends \verb|accept(Ch)| back via \verb|Resp|; (6) Alice's agent receives the response and both agents add each other to their outputs lists.

%------------------------------------------------------------------------------
\subsection{Channel Establishment}
%------------------------------------------------------------------------------

When a cold-call is accepted, both agents establish symmetric channels (Figure~\ref{fig:channel-establish}).

\begin{figure*}[t]
\begin{verbatim}
procedure bind_response(Decision?, AgentId?, Response,
    OutputsList?, OutputsList, Stream?, Stream).
bind_response(yes, From, accept(RetCh?), Outs, Outs1?, In, In1?) :-
    new_channel(RetCh, LocalCh) |
    handle_response(accept(LocalCh?), From?, Outs?, Outs1, In?, In1).
bind_response(no, _, no, Outs, Outs1?, In, In?) :-
    lookup_send('_user', msg(agent, '_user', rejected), Outs?, Outs1).

procedure handle_response(Response?, AgentId?,
    OutputsList?, OutputsList, Stream?, Stream).
handle_response(accept(ch(FIn, FOut?)), From, Outs, Outs2?, In, In1?) :-
    ground(From?) |
    add_output(From?, FOut, Outs?, Outs1),
    lookup_send('_user', msg(agent, '_user', connected(From?)), Outs1?, Outs2),
    merge(In?, FIn?, In1).
handle_response(no, From, Outs, Outs1?, In, In?) :-
    ground(From?) |
    lookup_send('_user', msg(agent, '_user', rejected(From?)), Outs?, Outs1).
\end{verbatim}
\caption{Channel establishment}
\label{fig:channel-establish}
\end{figure*}

The accepting agent creates a channel pair via \verb|new_channel(RetCh, LocalCh)|. It sends \verb|RetCh| back to the initiator and keeps \verb|LocalCh| for itself. Both channels are complementary: each agent's input is the other's output. Upon acceptance, \verb|add_output| adds the new friend's output stream to the outputs list, and the friend's input stream is merged into the agent's network input via \verb|merge|.

%------------------------------------------------------------------------------
\subsection{Text Messaging}
%------------------------------------------------------------------------------

Once agents are friends, they can exchange text messages (Figure~\ref{fig:messaging}).

\begin{figure*}[t]
\begin{multicols}{2}
\begin{verbatim}
%% User sends text to friend
agent(Id,
  [msg('_user', Id1,
    send(Target, Text))|UserIn],
  NetIn, Outs) :-
    Id? =?= Id1?, ground(Target?) |
    lookup_send(Target?,
      msg(Id?, Target?, text(Text?)),
      Outs?, Outs1),
    agent(Id?, UserIn?, NetIn?,
      Outs1?).
\end{verbatim}
\columnbreak
\begin{verbatim}
%% Received text from friend
agent(Id, UserIn,
  [msg(From, Id1, text(Text))|NetIn],
  Outs) :-
    Id? =?= Id1? |
    lookup_send('_user',
      msg(agent, '_user',
        received(From?, Text?)),
      Outs?, Outs1),
    agent(Id?, UserIn?, NetIn?,
      Outs1?).
\end{verbatim}
\end{multicols}
\caption{Text messaging between friends}
\label{fig:messaging}
\end{figure*}

%------------------------------------------------------------------------------
\subsection{Friend-Mediated Introduction}
%------------------------------------------------------------------------------

Once agents are friends, they can introduce each other to third parties. The introducer creates a fresh channel pair and sends each half to the respective parties (Figure~\ref{fig:friend-intro}).

\begin{figure*}[t]
\begin{multicols}{2}
\begin{verbatim}
%% User commands: introduce P to Q
agent(Id,
  [msg('_user', Id1,
    introduce(P, Q))|UserIn],
  NetIn, Outs) :-
    Id? =?= Id1?, ground(P?),
    ground(Q?),
    new_channel(PQCh, QPCh) |
    lookup_send(P?,
      msg(Id?, P?, intro(Q?, QPCh?)),
      Outs?, Outs1),
    lookup_send(Q?,
      msg(Id?, Q?, intro(P?, PQCh?)),
      Outs1?, Outs2),
    agent(Id?, UserIn?, NetIn?,
      Outs2?).
\end{verbatim}
\columnbreak
\begin{verbatim}
%% Received introduction from friend
agent(Id, UserIn,
  [msg(From, Id1,
    intro(Other, Ch))|NetIn],
  Outs) :-
    Id? =?= Id1?, ground(Other?) |
    lookup_send('_user',
      msg(agent, '_user',
        befriend_intro(
          From?, Other?, Ch?)),
      Outs?, Outs1),
    agent(Id?, UserIn?, NetIn?,
      Outs1?).

%% User accepts friend introduction
agent(Id,
  [msg('_user', Id1,
    accept_intro(Other, Ch))|UserIn],
  NetIn, Outs) :-
    Id? =?= Id1?, ground(Other?) |
    handle_intro_accept(Ch?, Other?,
      Outs?, Outs1, NetIn?, NetIn1),
    agent(Id?, UserIn?, NetIn1?,
      Outs1?).
\end{verbatim}
\end{multicols}
\caption{Friend-mediated introduction protocol}
\label{fig:friend-intro}
\end{figure*}

When Bob types \verb|introduce(alice, charlie)|, he creates a channel pair via \verb|new_channel(PQCh, QPCh)|. Alice receives \verb|ch(QtoP?, PtoQ)|---she reads from Charlie via \verb|QtoP?| and writes to Charlie via \verb|PtoQ|. Charlie receives the complementary \verb|ch(PtoQ?, QtoP)|. When both accept, they become direct friends without Bob's further involvement.

Note the distinction between cold-call messages and friend-mediated messages: cold-call uses a 2-argument \verb|msg(Target, Content)| sent via the \verb|'_net'| output, while friend-mediated introduction uses a 3-argument \verb|msg(From, To, intro(Other, Ch))| sent directly to a friend via their named output.

%------------------------------------------------------------------------------
\subsection{Network Switch Simulation}
%------------------------------------------------------------------------------

In deployment, agents communicate through a physical network. In simulation, a \verb|network3| process routes messages between three agents (Figure~\ref{fig:network-switch}).

\begin{figure*}[t]
\begin{multicols}{2}
\begin{verbatim}
procedure network3(
  Channel?, Channel?, Channel?).

%% Cold-call routing (2-arg msg)

%% Alice cold-calls Bob
network3(
  ch([msg(bob, X)|AliceIn],
     AliceOut?),
  ch(BobIn,
     [msg(bob, X?)|BobOut?]),
  ch(CharlieIn, CharlieOut?)) :-
    network3(
      ch(AliceIn?, AliceOut),
      ch(BobIn?, BobOut),
      ch(CharlieIn?, CharlieOut)).

%% (Five more 2-arg clauses for
%%  other sender/receiver pairs)

%% Friend-to-friend (3-arg msg)

%% Alice -> Bob
network3(
  ch([msg(alice, bob, X)|AliceIn],
     AliceOut?),
  ch(BobIn,
     [msg(alice, bob, X?)
      |BobOut?]),
  ch(CharlieIn, CharlieOut?)) :-
    network3(
      ch(AliceIn?, AliceOut),
      ch(BobIn?, BobOut),
      ch(CharlieIn?, CharlieOut)).
\end{verbatim}
\columnbreak
\begin{verbatim}
%% Bob -> Charlie
network3(
  ch(AliceIn, AliceOut?),
  ch([msg(bob, charlie, X)|BobIn],
     BobOut?),
  ch(CharlieIn,
     [msg(bob, charlie, X?)
      |CharlieOut?])) :-
    network3(
      ch(AliceIn?, AliceOut),
      ch(BobIn?, BobOut),
      ch(CharlieIn?, CharlieOut)).

%% (Four more 3-arg clauses for
%%  other sender/receiver pairs)

%% Termination
network3(
  ch([], []),
  ch([], []),
  ch([], [])).
\end{verbatim}
\end{multicols}
\caption{Network switch simulation for three agents}
\label{fig:network-switch}
\end{figure*}

The network switch routes both 2-argument cold-call messages (where \verb|msg(Target, Content)| is addressed by target name) and 3-argument friend-to-friend messages (where \verb|msg(From, To, Content)| carries both sender and receiver identities). In deployment, the network switch is replaced by maGLP's Cold-call transaction (Section~\ref{sec:maglp}).

%------------------------------------------------------------------------------
\subsection{The Scenario and Actors}
%------------------------------------------------------------------------------

The complete scenario demonstrates all three protocols:
\begin{enumerate}
\item Alice cold-calls Bob (Bob accepts) --- Alice and Bob become friends
\item Alice sends Bob: ``Hi Bob, this is Alice''
\item Bob cold-calls Charlie (Charlie accepts) --- Bob and Charlie become friends
\item Charlie sends Bob: ``Hi Bob, this is Charlie''
\item Bob introduces Alice to Charlie (both accept) --- Alice and Charlie become direct friends
\item Alice sends Charlie: ``Hi Charlie, this is Alice''
\item Charlie responds: ``Hi Alice, this is Charlie''
\end{enumerate}

Each agent is driven by an \emph{actor}---a GLP procedure that implements a state machine, reacting to messages from the agent and producing commands. The actor's state is encoded in procedure names (e.g., \verb|alice_wait_bob_connected|, \verb|bob_wait_charlie_msg|), with transitions via recursive calls. See \ref{appendix:social-graph-complete} for the complete program: helper predicates, actor implementations, the play that ties everything together, and the multiagent boot variants.

\input{sections/appendix-social-graph-complete}

\end{document}
